\section{Discussions and Limitations}
While recent works have reported lots
of potential side-channel vulnerabilities, we find most of them are not patched by
developers because they do not think those vulnerabilities are severe enough.
However, side-channels are inevitable in software and it is hard to fix all of them. 
Addressing old vulnerabilities may also introduce new leakage
sites. We need a
tool that can automatically estimate the sensitivity level of each vulnerability.
So software engineers can focus on
``severe'' leakages. For example, our tool will report that 
the modular exponentiation using square and multiply algorithms can
leak more information than a key validation function.

\tool{} can be used by software developers to find sensitive vulnerabilities
and reason about countermeasures.
\tool{} estimates the amount of leaked information for each side-channel leakage
in one execution trace. \tool{} is useful for software
engineers to test programs and fix vulnerabilities.
\tool{} works on native x86 execution traces. The design, which is very
precise in terms of true leakages compared to other static source code
method~\cite{197207,BacelarAlmeida:2013:FVS:2483313.2483334}, can omit some
leakages on other traces. The amount of leaked information also depends on the secret key.
However, as the tool is intended for debugging and testing,
we think it is software engineers' responsibility to select the input key and trigger 
the path in which they are interested. It is not a problem for crypto software 
since virtually all keys follow the same or similar computational paths.

We use the amount of leaked information to represent the sensitivity level of 
each side-channel vulnerability. Although imperfect, \tool{} produces a reasonable 
measurement for each leak. For example, the simple modular exponentiation is 
notoriously famous for various side-channel attacks~\cite{kocher1996timing}. 
During the execution, the single leak points will be executed multiple times, 
and each time it leaks one different bit. Therefore, \tool{} reports that the 
vulnerability can leak the whole key. However, not every leak point inside a 
loop is severe. Because very often the leakages may leak the same bit in the 
original key, and those leaks are not independent. \tool{} can capture the most 
fine-grained information by modeling each leak during the execution as a math 
formula and by using the conjunction of those formulas to describe the total effect. 
There could be multiple leakage sites in the program. While some leakages are independent, 
some other leakage sites are dependent. Some leakage sites (e.g., square and multiply) 
can leak one particular bit of the original key, but some leakage sites leak one bit 
from several bytes in the original key. Those constraints can precisely model the 
relationship between the key buffers and leakage sites.

\tool{} reaches full precision if the number of estimated leaked bits 
equals to Definition~\ref{def}. According to the threat model, the 
length of the secret is known, so the only estimated value is the number
of possible secrets under the attacker's observation ($|K^o|$). 
During the symbolic execution, \tool{} may lose precision from the 
memory model it uses in theory. However, we do not find false positives 
caused by the imprecise memory model during the evaluation. 
During the symbolic execution, for each constraint it generates,
the tool will check the correctness of the constraint by giving 
the constraint real values (the value of sensitive keys). Sampling 
can also introduce imprecision but with a probabilistic guarantee, 
as long as the sampling can terminate under the Central Limit Theorem (CLT) condition 
in probability theory. 
However, during the evaluation, we find that \tool{} cannot estimate 
the amount of leakage for some leakage sites in a reasonable time, 
which means the number of $K^o$ is very small. According to Definition~\ref{def}, 
it means the leakage is very severe.

