\section{Discussions and Limitations}
While recent work found many side-channel vulnerabilities, 
we note that many of them have not been patched by developers.
Side-channels are inevitable in software and it would be difficult to fix all of them. 
We need a
tool that estimates the sensitivity of each vulnerability
so software engineers can focus on
``severe'' leakages. For example, \tool{} reports that 
the modular exponentiation using square and multiply algorithms can
leak more information than a key validation function.

Software developers can use \tool{} to find severe vulnerabilities
and reason about countermeasures.
\tool{} estimates the amount of leaked information for each side-channel leakage
in one execution trace. \tool{} is useful for software
engineers to test programs and fix vulnerabilities.
The design, which is more precise in reporting true leakages as compared with other static
methods~\cite{197207,BacelarAlmeida:2013:FVS:2483313.2483334}, obviously misses
leakages on unexplored traces. The amount of leaked information also depends on the secret key.
However, as the tool is intended for debugging and testing,
we think it is a software engineer's responsibility to select the input key and trigger 
the path in which they are interested. It is not a problem for crypto software 
since virtually all keys follow similar computational paths.

We use the amount of leaked information to represent the sensitivity level of 
each side-channel vulnerability. Although imperfect, \tool{} produces a reasonable 
measurement for each leak. For example, the simple modular exponentiation is 
notoriously famous for multiple side-channel attacks~\cite{kocher1996timing}. 
During the execution, a single leak point may execute multiple times
and each time leak a different bit. In this case, \tool{} reports that the 
vulnerability can leak the whole key. However, not every leak point inside a 
loop is severe. If a site in the loop leaks the same bit from the 
original key, and these leaks are not independent. \tool{} captures most 
fine-grained information by modeling each leak during the execution as a 
formula and the conjunction of the formulas to describe its total effect. 
Some leakage sites (e.g., square and multiply) 
can leak one particular bit of the original key, but some leakage sites leak one bit 
from several bytes in the original key. \tool{} can capture the dependency among the leaks and
reports more precise leakage information.

\tool{} reaches full precision if the number of estimated leaked bits 
equals to Definition~\ref{def}. 
\tool{} may lose precision from the 
memory model it uses in theory. However, we did not find false positives 
caused by the imprecise memory model during our evaluation. 
Sampling introduces imprecision but with a probabilistic guarantee. 
However, during the evaluation, we find that \tool{} cannot estimate 
the amount of leakage for some leakage sites in a reasonable time, 
which means the number of $K^o$ is very small. According to Definition~\ref{def}, 
it means the leakage is very severe. The sampling method in \S\ref{sec:scala} seems
simple and may miss some leakages (e.g., chosen ciphertext attacks) in theory. 
However, the evaluation result 
shows \tool{} can identify all leakages found by the previous work~\cite{203878,236338,Brotzman19Casym}.
