\section{Scaling to Real-world Crypto Systems}
\label{sec:scala}

In the previous section, we propose an information leakage definition for
realistic attack scenarios to model two types of address-based side-channel
leakages and showed how to quantify them by calculating the number of input
keys ($K^o$) that satisfy the formulas. Intuitively, we can use
symbolic execution to capture math formulas and model counting
to obtain the number of satisfying input keys ($K^o$). However, preliminary
experiments showed that this approach was far too expensive to use with real-world
applications. In this section, we discuss the bottlenecks in
this approach and propose a practical solution.

In general, \tool{} faces the following performance challenges in
\emph{scaling to production-system crypto analysis}.
\begin{itemize}
      \item Symbolic execution (\textbf{Challenge C2})
      \item Counting the number of items in $K^o$ (\textbf{Challenge C3})
\end{itemize}

\subsection{Trace-oriented Symbolic Execution}
Symbolic execution is notorious for its high performance cost. 
Previous trace-oriented
symbolic execution
work~\cite{203878,Chattopadhyay:2017:QIL:3127041.3127044} has serious
performance bottlenecks. As a result, these approaches either apply only to
small programs~\cite{Chattopadhyay:2017:QIL:3127041.3127044} or require
domain knowledge~\cite{Wang:2007:NCD:1250662.1250723} to simplify the analysis. 
%Those tools interpret each
%instruction and update memory cells and registers with formulas that
%captured the semantics of the execution and search different input values that
%can lead to different execution behaviors using constraint solver. 
We implement
the approach presented in \S\ref{sec:trace-qif} and model the side-channels as
formulas. While the tools can analyze some simple cases such as AES, it cannot handle complicated examples such as RSA.
We observe that finding side-channels using symbolic execution differs from
traditional symbolic execution, and it can be optimized to be as efficient
as other methods.

\subsubsection{Interpret Instructions Symbolically}
Existing binary analysis \cready{tools}{frameworks}~\cite{shoshitaishvili2016state,
10.1007/978-3-642-22110-1_37, song2008bitblaze} translate machine instructions into
intermediate languages (IR) to simplify analysis since
the variety of machine instructions is
enormous, and their semantics is complex. The Intel Developer
Manual~\cite{intelsys} documents more than 1000 different x86 instructions. 
Unfortunately, the IR layer, which
reduces the workload of these tools, is not suitable for side-channels 
analysis because
IR-based or source code side-channels analyses do not represent the executed instructions accurate enough to analyze fully their control and memory accesses.
For example, a compiler may use conditional moves or bitwise operations to eliminate
branches. Also, as some IRs are not a superset or a subset of ISA, 
it is hard to rule out conditional jumps introduced by IR and add real branches 
eliminated by IR transformations.

Moreover, the IR causes significant overhead~\cite{217563}.
Translating machine instructions into IR is time-consuming. For example,
REIL IR~\cite{dullien2009reil}, adopted in CacheS~\cite{236338}, has multiple
transform processes, from binary to VEX IR, BAP IR, and finally REIL IR\@. 
Also, IR increases the total number of instructions. For example, x86
instruction \textit{test eax, eax} transfers into 18 REIL IR instructions. 

%\vspace*{2pt}
\textbf{Our Solution:}
We abandoned IR and expended the effort to implement 
symbolic execution directly on x86 instructions. 
Table~\ref{scala:ir} shows that eliminating the IR reduces the number 
of instructions examined during analysis. Previous works~\cite{217563} also 
adopted a similar approach to speed up fuzzing. Our implementation differs
from that work in two aspects: 1) We use complete constraints. 2) We run the
symbolic execution on one execution path each time. Our approach is approximately 30 times faster than using an IR (transferring ISA into IR and
symbolically executing it).

\begin{table}%[ht]
      \centering\small\footnotesize
      \caption{The number of x86,  % instructions and the number of 
             REIL IR, and VEX IR instructions on the traces of crypto programs.}
      \label{scala:ir}
\vspace*{-5pt}
      \resizebox{\columnwidth}{!}{%

            \begin{tabular}{cccc}
                  \hline
                                    & \begin{tabular}[c]{@{}c@{}}Number of\\ x86 Instructions\end{tabular} & \begin{tabular}[c]{@{}c@{}}Number of\\ VEX IR\end{tabular} & \begin{tabular}[c]{@{}c@{}}Number of\\ REIL IR\end{tabular} \\ \hline
                  AES OpenSSL 0.9.7 & $1,704$                   & $23,938$ (15x)            & $62,045$ (36x)            \\
                  DES OpenSSL 0.9.7 & $2,976$                   & $41,897$ (15x)            & $100,365$ (33x)           \\
                  RSA OpenSSL 0.9.7 & $1.6*10^7$                & $2.4*10^8$ (15x)          & $5.9*10^8$ (37x)          \\
                  RSA mbedTLS 2.5  & $2.2*10^7$                & $3.1*10^8$ (15x)          & $8.6*10^8$  (39x)         \\ \hline
            \end{tabular}
      }
      \vspace*{-16pt}
\end{table}

\subsubsection{Constraint Solving}
As discussed in \S\ref{side-channel:condition}, the problem of identifying
side-channels can be reduced to the question:

\begin{quote}
      \textit{Can we find two different input variables $k_1, k_2 \in K$ that
            satisfy the formula $f_a(k_1) \neq f_a(k_2)$?}
\end{quote}

Existing approaches rely on satisfiability modulo theories (SMT) solvers 
(e.g., Z3~\cite{DeMoura:2008:ZES:1792734.1792766}) to find satisfying assignments to
$k_1$ and $k_2$.
While this is a universal approach to solving constraints, 
for constraints of this form, using custom 
heuristics and testing is much more efficient in practice. Constraint 
solving is a decision problem expressed in logic formulas. SMT solvers 
transfer the SMT formula into the boolean conjunctive normal 
form (CNF) and feed it into the internal boolean satisfiability 
problem (SAT) solver. The translation process, called ``bit blasting'', 
is time-consuming. Also, as the SAT problem is a well-known NP-complete 
problem, it is hard to deal when it comes to
practical uses with huge formulas. Despite the rapid improvement 
in SMT solvers in recent years, constraint solving remains one of 
the obstacles to scaling the analysis of real-world crypto-systems.

%\vspace*{2pt}
\textbf{Our Solution:}
Instead of feeding the formula $f_a(k_1) \neq f_a(k_2)$ into a SMT solver, we
randomly pick $k_1, k_2 \in K$ and test them if they satisfy the
formula. Our solution is based on the following intuition. For most combination
of $(k_{1}, k_{2} )$, $f_a(k_1) \neq f_a(k_2)$. As long as
$f_a$ is not a constant function, such $k_1, k_2$ must exist. For example,
suppose each time we only have 5\% chance to find such $k_1, k_2$, then after we
test with different input combination with 100 times, we have $1 -
(1-0.05)^{100} = 99.6\%$ chance find such $k_1, k_2$. This type of random algorithm
works well for our problem.

\subsection{Counting the Solutions}
\label{MCreasons}
%According to
%Definition~\ref{def} introduced in \S\ref{sec:trace-qif},
%the problem of quantifying the amount of leaked information can be reduced to
%the problem of computing the number of items in $K^o$. However, we find that while
%there are various propositional model counters (e.g., \#SAT), they are not
%sufficient scalable for production cryptosystem analysis.
%there is no open source modulo theories counter (\#SMT) available.

%One straightforward method approximating the number of solutions is based on Monte Carlo
%sampling. However, the number of satisfying values could be exponentially small.
%Consider the formula $f_i\equiv{k_1} = 1\land{k_2} = 2\land{k_3} = 3\land{k_4} =
%4$, where $k_1$, $k_2$, $k_3$, and $k_4$ each represents one byte in the
%original sensitive input buffer, there is only one satisfying solution of total
%$2^{32}$ possible values, which requires exponentially many samples to get a
%tight bound. Monte Carlo method also suffers from the curse of dimensionality.
%For example, the length of an RSA private key can be as long as 4096 bits. If we
%take each byte (8 bits) in the original buffer as one symbol, the formula can
%have as many as 512 symbols.

%\vspace*{2pt}
%\textbf{Our Solution to Challenge C3:}
%We adopt multiple-step Monte Carlo sampling methods to count the number of
%possible inputs that satisfy the logic formula groups. The key idea is to split
%those constraints into several small formulas and sample them independently.
%We will introduce the method in the following subsection.

%\subsection{Information Leakage Estimation}

\newcommand{\addr}[1]{{l}_{#1}}
\renewcommand{\addr}[1]{{\gamma}_{#1}}
\renewcommand{\addr}[1]{{\zeta}_{#1}}
\renewcommand{\addr}[1]{{\xi}_{#1}}

In this section, we present the algorithm to calculate the information leakage
based on Definition~\ref{def} (\S\ref{sec:trace-qif}), answering to
\textbf{Challenge C3}.

\subsubsection{Problem Statement}
For each leakage site, we model it with a constraint using the
method presented in~\S\ref{side-channel:condition}. Suppose the address of the
leakage site is $\addr{i}$, we use $c_{\addr{i}}$ to denote the constraint
that models its side-channel leakage. 
%For
%multiple leakage sites, we take the conjunction of those constraints to
%represent those leakage sites.

According to the Definition~\ref{def}, to calculate the amount of leaked
information, the key is to calculate the cardinality
of $K^o$. Suppose an attacker can observe $n$ leakage sites, and each leakage
site has the following constraints: $c_{\addr{1}}, c_{\addr{2}}, \ldots,
c_{\addr{n}}$ respectively. The total leakage can be calculated from the constraint
$c_t({\addr{1}},{\addr{2}},\ldots,{\addr{n}}) = c_{\addr{1}} \land c_{\addr{2}}
\land \ldots \land c_{\addr{n}}$. 
%%The problem of estimating the total leaked
%%information can be reduced to the problem of counting the number of different
%%solutions that satisfies the constraint
%%$c_t({\addr{1}},{\addr{2}},\ldots,{\addr{n}})$. 
A simple method is to pick elements $k$ from $K$ and check if an
element is also contained in $K^o$. Assume $q$ elements satisfy this condition. In
expectation, we can use $\frac{k}{q}$ to approximate the value of
$\frac{|K|}{|K^o|}$.

However, the above sampling method fails in practice due to the following two problems:

\begin{enumerate}
      \item The curse of dimensionality problem. $c_t({\addr{1}},\ldots,{\addr{n}})$ is
            the conjunction of many constraints. Therefore, the input variables
            of each constraints will also be the input variables of the
            $c_t({\addr{1}},\ldots,{\addr{n}})$. The sampling method fails
            as $n$ grows. 

            %For example, if the program has $2$ input whose size is 
            %one byte, the whole search space is a $256^2$ cube. If we want
            %the sampling distance between each point equals to $d$, we need
            %$256^2d$ points. If the program has $10$ byte input, we need
            %$256^{10}d$ points if we still we want the sampling distance equals
            %to $d$.

      \item The number of satisfying assignments could be exponentially small.
            According to Chernoff bound, we need exponentially many samples to
            get a tight bound. 
            %On an extreme situation, if the constraint only
            %has one unique satisfying solution, the simple Monte Carlo method
            %cannot find the satisfying assignment even after sampling many
            %points.
\end{enumerate}

However, despite above two problems, we also observe two characteristics of the
problem:
\begin{enumerate}
      \item $c_t({\addr{1}},{\addr{2}},\ldots,{\addr{n}})$ is the conjunction of
            several short constraints $c_{\addr{i}}$. The set containing the
            input variables of $c_{\addr{i}}$ is the subset of the input
            variables of $c_t({\addr{1}},{\addr{2}},\ldots,{\addr{n}})$. Some
            constraints have completely different input variables from other
            constraints.

            \item Each time when we sample $c_t({\addr{1}},{\addr{2}},\ldots,{\addr{n}})$
            with a point, the sampling result is \emph{Satisfied} or not \emph{Not Satisfied}.
            The outcome does not depend on the result of 
            previous experiments. Also, as the amount of leaked information is calculated
            by a $\log$ function, we need not exactly count the number of solutions for 
            a given constraint.
            %\item For each constraint $F(C_{{addr}_i})$, the satisfying assignments
            %are close to each other, which means if we find one satisfying assignment, we 
            %are more likely to find other satisfying assignments nearby than randomly
            %pick one point in the whole searching space.

\end{enumerate}

In regard to the above problems, we present our methods. First, we split
$c_t(\addr{1},\addr{2},\ldots,\addr{n})$ into several independent constraint
groups. After that, we run a multi-step sampling method for each constraint.

\subsubsection{Maximum Independent Partition}

For a constraint $c_{\addr{i}}$, we define function $\pi$, which maps the
constraint into a set of different input symbols. For example, $\pi(k1 + k2 >
128) = \{k1, k2\}$.

\begin{mydef}[]
      \label{independentC}
      Given two constraints $c_m$ and $c_n$, we call them independent iff
      $$\pi(c_m) \cap \pi(c_n) = \emptyset$$
\end{mydef}

Based on the Definition~\ref{independentC}, we can split the constraint
$c_t(\addr{1},\addr{2},\ldots,\addr{n})$ into several independent constraints.
There are many partitions. For our project, we are interested in the following
one.

\begin{mydef}\label{Goodpartition}
      For the constraint $c_t(\addr{1},\addr{2},\ldots,\addr{n})$,
      we call the constraint group
      $g_{1}, g_{2}, \ldots, g_{m}$
      the maximum independent partition of $c_t(\addr{1},\addr{2},\ldots,\addr{n})$ iff
      \begin{enumerate}
            \item $g_{1} \land g_{2} \land \ldots \land g_{m} = c_t(\addr{1},\addr{2},\ldots,\addr{n})$
            \item $\forall i, j \in \{1, \ldots, m\} \quad \textrm{and} \quad
                        i \neq j,\quad\pi(g_{i}) \cap \pi(g_{j}) = \emptyset $
            \item For any other partitions  $h_{1}, h_{2}, \ldots, h_{m'}$ satisfy 1) and
                  2), $m \geq m'$
      \end{enumerate}

\end{mydef}

The reason we want a good partition of constraints is we want to reduce
the dimensions. For example,
a good partition of $F: ({k_1} = 1)\land({k_2} = 2)\land({k_3} > 4)\land({k_3} - {k_4} > 10)$ would be
$g_{1}: ({k_1} = 1)\quad g_{2}: ({k_2} = 2)\quad g_{3}: ({k_3} > 4) \land
({k_3} - {k_4} > 10)$  
We can sample each constraint independently and combine them
with Theorem~\ref{IndependentConstraint}.

\begin{theorem}
      \label{IndependentConstraint}
      Let $g_{1}, g_{2}, \ldots, g_{m}$ be a maximum independent partition of
      $c_t(\addr{1},\addr{2},\ldots,\addr{n})$.
      $K_c$ is the input set that satisfies constraint $c$. We have the following
      equation in regard to the size of $K_c$
      $$|K_{c_t(\addr{1},\addr{2},\ldots,\addr{n})}| = |K_{g_{1}}|*|K_{g_{2}}|*\ldots*|K_{g_{n}}|$$
\end{theorem}
\vspace{-3pt}
With Theorem~\ref{IndependentConstraint}, we change the problem of
counting the number of solutions to a complicated constraint in a high-dimension
space into counting solutions to several small constraints. We compute the
maximum independent partition by iterating each $\addr{i}$ and applying the function
$\pi$ over the constraint $\addr{i}$.

%% We apply the following
%% algorithm~\ref{algo:max-inde} to get the Maximum Independent Partition of the
%% $c_t(\addr{1},\addr{2},\ldots,\addr{n})$.

%% \IncMargin{1em}
%% \begin{algorithm}[h]
%%       \DontPrintSemicolon
%%       \SetKwInOut{Input}{input}\SetKwInOut{Output}{output}
%%       \Input{$c_t(\addr{1},\addr{2},\ldots,\addr{n}) = c_{\addr{1}} \land c_{\addr{2}} \land \ldots \land c_{\addr{m}}$}
%%       \Output{The Maximum Independent Partition of $G = \{g_{1}, g_{2}  , \ldots,  g_{m} \}$ }
%%       \For{$i\leftarrow 1$ \KwTo $n$}
%%       {
%%             $S_{c_i}$ $\leftarrow$ $\pi(c_{\addr{i}})$ \;
%%             \For{$g_{i} \in G$}
%%             {
%%                   $S_{g_i}$ $\leftarrow$ $\pi(g_{i})$ \;
%%                   $S$ $\leftarrow$ $S_{C_i} \cap S_{G_i}$  \;
%%                   \If{$S \neq \emptyset$}
%%                   {
%%                         $g_{i} \leftarrow g_{i} \land g_{\addr{i}}$ \;
%%                         \textbf{break} \;
%%                   }
%%                   Insert $c_{\addr{i}}$ to $G$
%%             }
%%       }
%%       \caption{The Maximum Independent Partition}
%%       \label{algo:max-inde}
%% \end{algorithm}
%% \DecMargin{1em}

\subsubsection{Multiple-step Monte Carlo Sampling}

After we split those constraints into several small constraints, we count the
number of solutions for each constraint. Even though the dimension has been
significantly reduced by the previous step, this is still a \#P problem. 
%For our
%project, we apply the approximate counting instead of exact counting for two
%reasons. First, we do not need to have a very precise result of the exact number
%of total solutions since the information is defined with a logarithmic function.
%We do not need to distinguish between constraints having $10^{10}$ or $10^{10} +
%10$ solutions; they are very close to after taking logarithmic. Second, the
%precise model counting approaches, like Davis-Putnam-Logemann-Loveland (DPLL)
%search, have difficulty scaling up to large problem sizes.

We apply the ``counting by sampling'' method. For
the constraint $g_{i}= c_{i_1} \land c_{i_2} \land ,\ldots, \land c_{i_j} \land
,\ldots, \land c_{i_m}$, if the solution satisfies $g_{i}$, it should also
satisfy any constraint from $c_{i_1}$ to $c_{i_m}$. In other words, $K_{c_gi}$
should be the subset of $K_{c_1}$, $K_{c_2}$, \ldots , $K_{c_m}$. We notice that
$c_i$ usually has fewer inputs than $g_{i}$. For example, if
$c_{i_j}$ has only one 8-bit input variable, we can find the exact solution set
$K_{c_{i_j}}$ of $c_{i_j}$ by trying every possible 256 solution. After that,
we only generate random input numbers for the other input variables in
constraint $g_{i}$. With this simple yet effective trick, we reduce the number of input
while still ensuring accuracy.

%% the algorithm
%\IncMargin{1em}
%\begin{algorithm}
%\SetAlgoLined
%\DontPrintSemicolon

%\KwIn{{The constraint $G_{i}= C_{i_1} \land C_{i_2}
%\land \ldots \land C_{i_m}$}}    
%\KwOut{{The number of assignments that satisfy $G_{i}$ $|K_{G_{i}}|$}}

%\SetKwProg{RW}{RandomWalk}{}{}
%\SetKwProg{MM}{MetropolisMove}{}{} 
%$n$: the number of sampling times \;
%$P$: a probability generator \;
%$k$: the input assignment \;
%$n_{s}$: the number of satisfying assignments \;
%$\#k$: the satisfying number of k \fixme{this number is not used syntactically} \; 
%Initialization: \;
%$\#{k_0}$ $\leftarrow$ $\sum_{j=1}^{m}C_{i_j}(k_0)$ \;
%\For{$t\leftarrow 1$ \KwTo $n$} {
%      $p$ $\leftarrow$ $P$ \;
%      \If{$p \geq 0.5$}
%      {
%        $v$ $\leftarrow$ \RW{$v$} {}
%      }
%      \Else{
%            $v$ $\leftarrow$ \MM{$v$} {}
%      }
%      \If{$v$ satisfies $G_{i}$}
%      {$n_{s}$ $\leftarrow$ $n_{s} + 1$}
%}
%$|K_{G_{i}}|$ $\leftarrow$ $n_s|K| / n$
%\caption{Metropolis Sampling}
%\end{algorithm}
%\DecMargin{1em}

\subsubsection{Error Estimation}
\label{sssec:errest}
Our result has a probabilistic guarantee that the error of the estimated amount of leaked 
information is less than 1 bit under the Central Limit Theorem (CLT) and uncertainty
propagation theorem.

Let $n$ be the number of samples and $n_s$ be the number of samples that satisfy
the constraint $C$. Then we get $\hat{p} = \frac{n_s}{n}$. If we repeat the
experiment multiple times, each time we get a $\hat{p}$. As each
$\hat{p}$ is independent and identically distributed, according to the central limit
theorem, the mean value should follow the normal distribution
$ \frac{\bar{p}-E(p)}{\sigma\sqrt{n}} \rightarrow N(0,1) $. Here $E(p)$ is the
mean value of $p$, and $\sigma$ is the standard variance of $p$. If we use the
observed value $\hat{p}$ to calculate the standard deviation, we can claim that
we have 95\%\footnote{For a normal distribution, 95\% of variable $\Delta p$ fall within two sigmas of the mean.} 
confidence that the error $\Delta p= \bar{p} - E(p)$ falls in the interval:
$$ |\Delta p| \leq 1.96\sqrt{\frac{ \hat{p} (1- \hat{p} )}{n}}$$

Since we use $L = \log_{2}p$ to estimate the amount of leaked information, we
can have the following error propagation formula $\Delta L = \frac{\Delta
p}{p\ln2}$ by taking the derivative from Definition~\ref{def}. For \tool, we want the error of estimated leaked
information ($\Delta L$) to be less than 1 bit. So we get $\frac{\Delta
p}{p\ln2} \leq 1$. Therefore, as long as $ n \geq \frac{1.96^2(1-p)}{p(\ln2)^2}$, we have
95\% confidence that the error of estimated leaked information is less than 1 bit.
During the simulation, if $n$ and $p$ satisfy this inequality, the Monte Carlo
simulation will terminate.
