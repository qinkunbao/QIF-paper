
\newcommand{\vvv}{\vspace*{-3pt}}

\section{Exisiting Information Leakage Quantification}

Given an event $e$ that occurs with the probability $p(e)$, we receive
\begin{displaymath}
    I = - \log_2p(e)
\end{displaymath}
bits of information by knowing the event $e$ happens according to information theory~\cite{shannon1948mathematical}. 
Considering a char variable $a$
with one byte size in a C program, its value ranges from 0 to 255.
If we observe $a$ equals $1$, without any domain knowledge, 
the probability of this observation is $\frac{1}{256}$. 
So we get $-\log(\frac{1}{256}) = 8$ bits information, which is exactly the size
of a char variable in the C program.
Existing work on information leakage quantification typically use Shannon
entropy~\cite{clark2007static,Wichelmann:2018:MFF:3274694.3274741},
min-entropy~\cite{10.1007/978-3-642-00596-1_21}, and max-entropy~\cite{182946,
Doychev:2017:RAS:3062341.3062388}. In these frameworks, the input sensitive
information $K$ is considered as a random variable.

Let $k$ be one of the possible
value of $K$. The Shannon entropy $H(K)$ is defined as
\begin{displaymath}
    H(K) = - \sum_{k {\in} K}p(k)\log_2(p(k))
\end{displaymath}

Shannon entropy can be used to quantify the initial uncertainty about
sensitive information. It measures the amount of information in a system.

Min-entropy describes the information leaks for a program with the most likely input. 
For example, min-entropy can be used to describe the
best chance of success in guessing one's password using the
most common password. %, which is defined as
\begin{displaymath}
    \mathit{min\text{-}entropy} = - \log_2(p_{\mathit{max}})
\end{displaymath}

Max-entropy is defined solely on the number of possible observations.
%It is equal to $-\log_2{n}$.
\begin{displaymath}
    \mathit{max\text{-}entropy} = -\log_2{n}
\end{displaymath}
As it is easy to compute, most recent works~\cite{182946,Doychev:2017:RAS:3062341.3062388} use max-entropy as the definition of
the amount of leaked information.

To illustrate how these definitions work, we consider the code
fragment in Figure~\ref{fig:side-channel}. It has two possible leakage sites, A and B.

\begin{figure}[h!]
\centering
\begin{lstlisting}[xleftmargin=.03\textwidth,xrightmargin=.01\textwidth]
uint8_t key[2], t1, t2;
get_key(key);              // 0 <= key[0], key[1] < 256
t1 = key[0] + key[1];
t2 = key[0] - key[1];
if (t1 < 4)                // leakage site A
    foo();                             
if (t2 > 0)                // leakage site B     
    bar();                             
\end{lstlisting}
\vspace*{-1pt}
    \caption{Side-channel leakage}
    \label{fig:side-channel}
\end{figure}
In this paper we assume an attacker can observe the secret-dependent control-flows in 
Figure~\ref{fig:side-channel}.
Therefore, an attacker can have two different observations for each leak site
depending on the value of the $\mathit{key}$: $A$ for function \textsf{foo} is executed, 
$\neg A$ for function \textsf{foo} is not executed, $B$ for function \textsf{bar} is
executed, and $\neg B$ for function \textsf{bar} is not executed.

\begin{table}[ht]
    \centering\small\footnotesize
    \caption{The distribution of observation}\label{shtable}
    \vspace*{-0pt}
%    \resizebox{\columnwidth}{!}{
    \begin{tabular}{l|cc|cc}
        \hline
        Observation ($o$)   & $A$ & $\neg A$ & $B$ & $\neg B$ \\ \hline
        Number of Solutions & 65526       & 10        & 32768     & 32768           \\ \hline
        Possibility (p)     & 0.9998      & 0.0002    & 0.5    & 0.5       \\
        \hline
    \end{tabular}
%        }
\end{table}

Assuming $\mathit{key}$ is uniformly distributed, we can calculate the corresponding
possibility by counting the number of possible inputs. Table~\ref{shtable}
describes the probability of each observation. We use the above three types of 
leakage metrics to quantify the leaked information for leak A and leak B.

\vspace{3pt}
\textbf{Min Entropy.}
As $p_{A\mathit{max}} = 0.9998$ and $p_{B\mathit{max}} = 0.5$, 
with the definition, $\mathit{min\text{-}entropy_A}$ equals to
$0.000\ \mathrm{bits}$ and $\mathit{min\text{-}entropy_B}$ equals to
$1.000\ \mathrm{bits}$.

\vspace{3pt}
\textbf{Max Entropy.}
Depending on the value of key, the code have two branches for each leakage site. 
Therefore, with the max entropy definition, both leakage sites leak $1.000\ \mathrm{bits}$.

\vspace{3pt}
\textbf{Shannon Entropy.}
Based on Shannon entropy, the respective amount of information in A and B equals to
{\footnotesize
\begin{align*}
    \mathit{Shannon\text{-}entropy_A} & = -(0.9998*\log_{2}0.9998      \\
                                    & \qquad+ 0.0002*\log_{2}0.0002)  \\
                                    & = 0.000\ \mathrm{bits}         \\
    \mathit{Shannon\text{-}entropy_B} & = -(0.5*\log_{2}0.5      \\
                                    & \qquad+ 0.5*\log_{2}0.5)        \\
                                    & = 1.000\ \mathrm{bits}                             
\end{align*}
}

\section{\tool{}' main components}

\label{appendix:code}
\begin{table}[h!]
    \centering
    %    \resizebox{.8\columnwidth}{!}{
    \caption{\tool{}' main components and sizes}\label{tbl:implementation}
    \begin{tabular}{lr@{~}@{}l}
        \hline
        Component            & \multicolumn{2}{c}{Lines of Code (LOC)}             \\ \hline
        Trace Logging        & 501 lines                               & of C++    \\
        Symbolic Execution   & 14,963 lines                            & of C++    \\
        Data Flow            & 451 lines                               & of C++    \\
        Monte Carlo Sampling & 603 lines                               & of C++    \\
        Others               & 211 lines                               & of Python \\ \hline
        Total                & 16,729 lines                            &           \\\hline
    \end{tabular}
    %    }
\end{table}



\section{Algorithm to Compute the Maximum Independent Partition}
\renewcommand{\baselinestretch}{0.97}\selectfont

\label{appendix:partition}
{\small
\IncMargin{1em}
\begin{algorithm}[h]\small
    \DontPrintSemicolon
    \SetKwInOut{Input}{input}\SetKwInOut{Output}{output}
    \Input{$c_t(\addr{1},\addr{2},\ldots,\addr{n}) = c_{\addr{1}} \land c_{\addr{2}} \land \ldots \land c_{\addr{m}}$}
    \Output{The Maximum Independent Partition of $G = \{g_{1}, g_{2}  , \ldots,  g_{m} \}$ }
    Insert $c_{\addr{1}}$ to $G$ as a new entry \;
    \For{$i\leftarrow 2$ \KwTo $n$}
    {
        $S_{c_{\addr{i}}}$ $\leftarrow$ $\pi(c_{\addr{i}})$ \;
        \For{$g_{i} \in G$}
        {
            $S_{g_j}$ $\leftarrow$ $\pi(g_{j})$ \;
            $S$ $\leftarrow$ $S_{c_{\addr{i}}} \cap S_{g_j}$  \;
            \If{$S \neq \emptyset$}
            {
                $g_{j} \leftarrow g_{i} \land g_{\addr{i}}$ \;
                \textbf{continue} \;
            }
            Insert $c_{\addr{i}}$ to $G$ as a new entry \;
        }
    }
    \caption{The Maximum Independent Partition}
    \label{algo:max-inde}
\end{algorithm}
\DecMargin{1em}
}

\renewcommand{\baselinestretch}{1}\selectfont

\section{Algorithm to Compute the Number of Satisfying Assignments}
\label{appendix:montecarlo}
~
{\small
\IncMargin{1em}
\begin{algorithm}\small
    \SetAlgoLined
    \DontPrintSemicolon

    \KwIn{{The constraint $g_{i}= c_{i_1} \land c_{i_2}
                    \land \ldots \land c_{i_m}$}}
    \KwOut{{The number of assignments that satisfy $g_{i}$ $|K_{g_{i}}|$}}

    $n$: the number of sampling times \;
    $S_{c_i}$: the set contains input variables for $c_{i}$ \;
    $n_{s}$: the number of satisfying assignments \;
    $N_{c_t}$: the set contains all solution for $c_t$ \;
    $r$: times of reducing $g$\;
    $k$: the input variable \;
    $R$: a function that produces a random point from $S_{c_i}$\;
    %$\#k$: the satisfying number of k \fixme{this number is not used syntactically} \;
    %Initialization: \;
    $r$ $\leftarrow$ $1$,
    $n$ $\leftarrow$ $0$ \;
    \For{$t$ $\leftarrow$ $1$ \KwTo $m$} {
        $S_{c_t}$ $\leftarrow$ $\pi(c_t)$ \;
        \If{$|S_{c_t}| = 1$}
        {
            $N_{c_t}$ $\leftarrow$ Compute all solutions of $c_i$ \;
            $N_{c_t} = \{n_1, \ldots, n_m\},\ S_{c_t} = \{k\}  $ \;
            $g_{i} = $ $g_i(k=n_1) \land \ldots \land g_i(k=n_m)$ \;
            $r \leftarrow r+1$ \;

        }
    }
    \While{$n \leq \frac{8p}{1-p}$} {
        $S_{g_i}$ $\leftarrow$ $\pi(g_i)$ \;
        $v \leftarrow R(S_{g_i})$ 
        \If{$v$ satisfies $g_i$}
        {
           $n_s \leftarrow n_s + 1$
        }
        $n \leftarrow n +1,\ p = \frac{n_s}{n}$
    }

    $|K_{g_{i}}|$ $\leftarrow$ $n_s|K| / (n * r * range(k))$
    \caption{Multiple Step Monte Carlo Sampling}
\end{algorithm}
\DecMargin{1em}
}

\section{AES Lookup Tables Leakage}
\label{appendix:mbedtls_aes}
Shown in Figure~\ref{fig:lookuptable}, usually a smaller look up table 
leaks less amount of information.
\begin{figure}[h]
    \centering
    \begin{lstlisting}[xleftmargin=.01\textwidth,xrightmargin=.01\textwidth]
int mbedtls_internal_aes_encrypt( mbedtls_aes_context *ctx,
const unsigned char input[16],
unsigned char output[16] )
{
uint32_t *RK, X0, X1, X2, X3, Y0, Y1, Y2, Y3;
...
for( i = ( ctx->nr >> 1 ) - 1; i > 0; i-- )
{
  AES_FROUND( Y0, Y1, Y2, Y3, X0, X1, X2, X3 ); // Leakage 1
  AES_FROUND( X0, X1, X2, X3, Y0, Y1, Y2, Y3 ); // Leakage 2
}
AES_FROUND( Y0, Y1, Y2, Y3, X0, X1, X2, X3 );   // Leakage 3
X0 = *RK++ ^ \                                  // Leakage 4
    ( (uint32_t) FSb[ ( Y0       ) & 0xFF ]       ) ^
    ( (uint32_t) FSb[ ( Y1 >>  8 ) & 0xFF ] <<  8 ) ^
    ( (uint32_t) FSb[ ( Y2 >> 16 ) & 0xFF ] << 16 ) ^
    ( (uint32_t) FSb[ ( Y3 >> 24 ) & 0xFF ] << 24 );
// X1, X2, X3 do the same computation as X0
...                                          // Leakage 5,6,7
PUT_UINT32_LE( X0, output,  0 );
...
return( 0 );
}
\end{lstlisting}
    \vspace*{-6pt}
    \caption{Function \textsf{mbedtls\_internal\_aes\_encrypt}}
    \label{fig:lookuptable}
    \vspace*{-9pt}
\end{figure}


\section{Minor side-channels vulnerability}
\label{appendix:minor:vul}
Here we present a side-channel vulnerability that leaks
less than one bit information by~\tool{}. The vulnerability exists
from OpenSSL 0.9.7 to OpenSSL 1.1.1. Shown in Figure~\ref{appen:fig:exp} 
and Figure~\ref{appen:fig:exp_new}, \textbf{m} is a big number that
derives from the private key.
At line 6, it can leak the last bit of \textbf{m} by observing the branch. 
As the leak is tiny,
we think developers do not have enough motivations
to fix the vulnerability.

\begin{figure}[h]
\centering
\begin{lstlisting}[xleftmargin=.02\textwidth,xrightmargin=.01\textwidth]
int BN_mod_exp_mont_consttime(BIGNUM *rr, 
const BIGNUM *a, const BIGNUM *p,
const BIGNUM *m, BN_CTX *ctx, BN_MONT_CTX *in_mont)
{
... 
if (!(m->d[0] & 1)) {
    ... 
    return 0;
}
bits = BN_num_bits(p);
if (bits == 0) 
...
}
\end{lstlisting}
    \vspace*{-6pt}
    \caption{\textsf{BN\_mod\_exp\_mont\_consttime} in OpenSSL 0.9.7}
    \label{appen:fig:exp}
    \vspace*{-6pt}
\end{figure}

\begin{figure}[h]
    \centering
    \begin{lstlisting}[xleftmargin=.02\textwidth,xrightmargin=.01\textwidth]
int BN_mod_exp_mont_consttime(BIGNUM *rr, 
const BIGNUM *a, const BIGNUM *p,
const BIGNUM *m, BN_CTX *ctx, BN_MONT_CTX *in_mont)
{
...
if (!BN_is_odd(m)) {
    ...
    return 0;
}
bits = BN_num_bits(p);
if (bits == 0) 
...
}
\end{lstlisting}
    \vspace*{-6pt}
    \caption{\textsf{BN\_mod\_exp\_mont\_consttime} in OpenSSL 1.1.1}
    \label{appen:fig:exp_new}
    \vspace*{-6pt}
\end{figure}

\section{Unknown Leaks in OpenSSL 1.1.1}\label{unknown:leak}
Shown in Table~\ref{tab:RSAOpenSSL1.1.1g}, \tool{} discovers a series
of side-channel vulnerabilities in the up-to-date version of OpenSSL library. 
However, many of them are negligible quantified by \tool{}.
Here we present a few vulnerabilitites in Figure~\ref{fig:unknown1} and Figure~\ref{fig:unknown2}. 

\begin{figure}[h]
\centering
\begin{lstlisting}[xleftmargin=.02\textwidth,xrightmargin=.01\textwidth]
BN_ULONG bn_sub_words(BN_ULONG *r, const BN_ULONG *a, 
const BN_ULONG *b, int n)
{
BN_ULONG t1, t2;
int c = 0;
... 
while (n) {
    t1 = a[0];
    t2 = b[0];
    r[0] = (t1 - t2 - c) & BN_MASK2;
    if (t1 != t2)         // leakage         
        c = (t1 < t2);
    a++;
    b++;
    r++;
    n--;
}
return c;
}
\end{lstlisting}
        \vspace*{-6pt}
        \caption{Unknown sensitive secret-dependent branch leaks from function 
                 \textsf{bn\_sub\_words} in OpenSSL 1.1.1g.}
        \label{fig:unknown1}
        \vspace*{-6pt}
 \end{figure}

\begin{figure}[h]
\centering
\begin{lstlisting}[xleftmargin=.02\textwidth,xrightmargin=.01\textwidth]
int bn_div_fixed_top(BIGNUM *dv, BIGNUM *rm, 
const BIGNUM *num,
const BIGNUM *divisor, BN_CTX *ctx)
{
... 
t2 = (BN_ULLONG) d1 *q;
for (;;) {
    if(t2 <= ((((BN_ULLONG) rem) << BN_BITS2) | n2)) //leakage
        break;
    q--;
    rem += d0;
    if (rem < d0)
        break;      /* don't let rem overflow */
    t2 -= d1;
}
... 
}
\end{lstlisting}
    \vspace*{-6pt}
    \caption{Unknown sensitive secret-dependent branch leaks from function 
             \textsf{bn\_div\_fixed\_top} in OpenSSL 1.1.1g.}
    \label{fig:unknown2}
    \vspace*{-6pt}
\end{figure}


\section{Detailed Experimental Results}
\label{sec:result-table}

Here we present the detailed experimental results.
Due to space limitation, we select the representative implementations of
AES, DES, RSA, and ECDSA in
mbedTLS 2.5,
OpenSSL 1.1.0f,  and
OpenSSL 1.1.1.  
%For RSA, we also include OpenSSL 1.0.2f and OpenSSL 1.0.2k.
The results are representative to other versions.
All the results will be made available in electronic format online
when the paper is published. %at \fixme{http://tinyurl}.

In all the tables presented in this appendix, the mark ``$*$'' means timeout,
which indicates more severe leakages. See \S\ref{loc:timeout} for the details.
Also note that we round the calculated numbers of leaked bits to include one digit
after the decimal point, so $0.0$ really means very small amount of leakage, 
but not exactly zero. See \S\ref{sssec:errest} for the details of error estimate.

 \begin{table}[!ht]
\centering\tiny\scriptsize
\caption{Leakages in DES implemented by mbed TLS 2.5}\label{tab:DESmbed TLS2.5}
%\resizebox{\columnwidth}{!}{
\begin{tabular}{lrlrr}
\hline
\textbf{File} & \textbf{Line No.} & \textbf{Function} & \textbf{\# Leaked Bits} & \textbf{Type} \\\hline
des.c& 441&mbedtls\_des\_setkey&0.9 &DA\\
des.c& 438&mbedtls\_des\_setkey&1.0 &DA\\
des.c& 438&mbedtls\_des\_setkey&1.0 &DA\\
des.c& 439&mbedtls\_des\_setkey&1.1 &DA\\
des.c& 439&mbedtls\_des\_setkey&1.0 &DA\\
des.c& 440&mbedtls\_des\_setkey&1.0 &DA\\
des.c& 446&mbedtls\_des\_setkey&0.9 &DA\\
des.c& 446&mbedtls\_des\_setkey&1.0 &DA\\
des.c& 444&mbedtls\_des\_setkey&1.0 &DA\\
des.c& 444&mbedtls\_des\_setkey&1.0 &DA\\
des.c& 443&mbedtls\_des\_setkey&1.0 &DA\\
des.c& 443&mbedtls\_des\_setkey&1.0 &DA\\
des.c& 444&mbedtls\_des\_setkey&1.0 &DA\\
des.c& 445&mbedtls\_des\_setkey&1.1 &DA\\
des.c& 448&mbedtls\_des\_setkey&0.9 &DA\\
\hline
\end{tabular}
%}
\renewcommand{\baselinestretch}{1.0}\selectfont
\end{table}

 \begin{table}[h!]
\centering\tiny\scriptsize
\caption{Leakages in DES implemented by OpenSSL 1.1.0f}\label{tab:DESOpenSSL1.1.0f}
%\resizebox{\columnwidth}{!}{
\begin{tabular}{lrlrr}
\hline
\textbf{File} & \textbf{Line No.} & \textbf{Function} & \textbf{\# Leaked Bits} & \textbf{Type} \\\hline
set\_key.c& 351&DES\_set\_key\_unchecked&7.1 &DA\\
set\_key.c& 353&DES\_set\_key\_unchecked&8.8 &DA\\
set\_key.c& 361&DES\_set\_key\_unchecked&8.0 &DA\\
set\_key.c& 362&DES\_set\_key\_unchecked&5.7 &DA\\
set\_key.c& 362&DES\_set\_key\_unchecked&2.0 &DA\\
set\_key.c& 364&DES\_set\_key\_unchecked&3.5 &DA\\
set\_key.c& 364&DES\_set\_key\_unchecked&4.9 &DA\\
set\_key.c& 365&DES\_set\_key\_unchecked&0.4 &DA\\
\hline
\end{tabular}
%}
\renewcommand{\baselinestretch}{1.0}\selectfont
\end{table}

 \begin{table}[!ht]
\centering\tiny\scriptsize
\caption{Leakages in DES implemented by OpenSSL 1.1.1}\label{tab:DESOpenSSL1.1.1}
%\resizebox{\columnwidth}{!}{
\begin{tabular}{lrlrr}
\hline
\textbf{File} & \textbf{Line No.} & \textbf{Function} & \textbf{\# Leaked Bits} & \textbf{Type} \\\hline
set\_key.c& 350&DES\_set\_key\_unchecked&5.8 &DA\\
set\_key.c& 350&DES\_set\_key\_unchecked&6.6 &DA\\
set\_key.c& 350&DES\_set\_key\_unchecked&7.5 &DA\\
set\_key.c& 350&DES\_set\_key\_unchecked&6.4 &DA\\
set\_key.c& 355&DES\_set\_key\_unchecked&1.9 &DA\\
set\_key.c& 355&DES\_set\_key\_unchecked&3.1 &DA\\
\hline
\end{tabular}
%}
\renewcommand{\baselinestretch}{1.0}\selectfont
\end{table}

 
 
 \begin{table}[!ht]
\centering\tiny\scriptsize
\caption{Leakages in RSA implemented by mbed TLS 2.5}\label{tab:RSAmbed TLS2.5}
%\resizebox{\columnwidth}{!}{
\begin{tabular}{lrlrr}
\hline
\textbf{File} & \textbf{Line No.} & \textbf{Function} & \textbf{\# Leaked Bits} & \textbf{Type} \\\hline
bignum.c& 1617&mbedtls\_mpi\_exp\_mod&0.9 &CF\\
bignum.c& 861&mbedtls\_mpi\_cmp\_mpi&8.5 &CF\\
bignum.c& 862&mbedtls\_mpi\_cmp\_mpi&7.7 &CF\\
bignum.c& 1167&mpi\_mul\_hlp&*&CF\\
bignum.c& 828&mbedtls\_mpi\_cmp\_abs&9.6 &CF\\
bignum.c& 829&mbedtls\_mpi\_cmp\_abs&9.5 &CF\\
\hline
\end{tabular}
%}
\renewcommand{\baselinestretch}{1.0}\selectfont
\end{table}

 \begin{table}[h!]
\centering\tiny\scriptsize
\caption{Leakages in RSA implemented by mbed TLS 2.15.1}\label{tab:RSAmbed TLS2.15.1}
%\resizebox{\columnwidth}{!}{
\begin{tabular}{lrlrr}
\hline
\textbf{File} & \textbf{Line No.} & \textbf{Function} & \textbf{\# Leaked Bits} & \textbf{Type} \\\hline
bignum.c& 855&mbedtls\_mpi\_cmp\_mpi&*&CF\\
rsa.c& 184&rsa\_check\_context.isra.0&1.0 &CF\\
bignum.c& 825&mbedtls\_mpi\_cmp\_abs&*&CF\\
bignum.c& 197&mbedtls\_mpi\_copy&*&CF\\
bignum.c& 1629&mbedtls\_mpi\_exp\_mod&0.9 &CF\\
bignum.c& 829&mbedtls\_mpi\_cmp\_abs&*&CF\\
bignum.c& 859&mbedtls\_mpi\_cmp\_mpi&*&CF\\
bignum.c& 873&mbedtls\_mpi\_cmp\_mpi&2.8 &CF\\
bignum.c& 874&mbedtls\_mpi\_cmp\_mpi&2.7 &CF\\
bignum.c& 840&mbedtls\_mpi\_cmp\_abs&8.2 &CF\\
bignum.c& 841&mbedtls\_mpi\_cmp\_abs&8.7 &CF\\
bignum.c& 1201&mbedtls\_mpi\_mul\_mpi&*&CF\\
\hline
\end{tabular}
%}
\renewcommand{\baselinestretch}{1.0}\selectfont
\end{table}

 \begin{table}[h!]
\centering\tiny\scriptsize
\caption{Leakages in RSA implemented by OpenSSL 1.0.2f}\label{tab:RSAOpenSSL1.0.2f}
%\resizebox{\columnwidth}{!}{
\begin{tabular}{lrlrr}
\hline
\textbf{File} & \textbf{Line No.} & \textbf{Function} & \textbf{\# Leaked Bits} & \textbf{Type} \\\hline
bn\_lib.c& 199&BN\_num\_bits&*&CF\\
bn\_lib.c& 200&BN\_num\_bits&15.6 &CF\\
bn\_lib.c& 201&BN\_num\_bits&16.2 &DA\\
bn\_lib.c& 673&BN\_ucmp&*&CF\\
bio\_asn1.c& 482&\_\_udivdi3&8.5 &CF\\
bn\_div.c& 381&BN\_div&*&CF\\
bn\_div.c& 456&BN\_div&*&CF\\
bn\_gcd.c& 279&BN\_mod\_inverse&1.0 &CF\\
bn\_gcd.c& 302&BN\_mod\_inverse&5.0 &CF\\
bn\_gcd.c& 324&BN\_mod\_inverse&6.6 &CF\\
bn\_add.c& 255&BN\_usub&*&CF\\
bn\_gcd.c& 305&BN\_mod\_inverse&12.8 &CF\\
bn\_gcd.c& 327&BN\_mod\_inverse&15.6 &CF\\
bn\_lib.c& 203&BN\_num\_bits&15.2 &DA\\
bn\_lib.c& 208&BN\_num\_bits&12.7 &CF\\
bn\_lib.c& 209&BN\_num\_bits&*&DA\\
bn\_lib.c& 212&BN\_num\_bits&*&DA\\
bn\_gcd.c& 515&BN\_mod\_inverse&*&CF\\
bn\_div.c& 381&BN\_div&2.7 &CF\\
bn\_div.c& 439&BN\_div&14.2 &CF\\
bn\_div.c& 385&BN\_div&11.4 &CF\\
bn\_div.c& 381&BN\_div&1.1 &CF\\
bn\_div.c& 381&BN\_div&1.0 &CF\\
bn\_div.c& 469&BN\_div&0.9 &CF\\
bn\_exp.c& 676&BN\_mod\_exp\_mont\_consttime&1.0 &CF\\
bn\_exp.c& 796&BN\_mod\_exp\_mont\_consttime&1.0 &CF\\
bn\_mont.c& 262&BN\_from\_montgomery\_word&*&DA\\
bn\_mont.c& 263&BN\_from\_montgomery\_word&*&DA\\
bn\_mont.c& 264&BN\_from\_montgomery\_word&*&DA\\
bn\_mont.c& 266&BN\_from\_montgomery\_word&*&DA\\
bn\_mont.c& 276&BN\_from\_montgomery\_word&*&DA\\
bn\_mont.c& 276&BN\_from\_montgomery\_word&0.2 &DA\\
bn\_mont.c& 276&BN\_from\_montgomery\_word&0.2 &DA\\
bn\_mont.c& 275&BN\_from\_montgomery\_word&0.1 &CF\\
bn\_mont.c& 282&BN\_from\_montgomery\_word&*&CF\\
bn\_asm.c& 787&bn\_sqr\_comba8&*&CF\\
bn\_asm.c& 646&bn\_mul\_comba8&*&CF\\
bn\_mont.c& 201&BN\_from\_montgomery\_word&0.0 &CF\\
\hline
\end{tabular}
%}
\renewcommand{\baselinestretch}{1.0}\selectfont
\end{table}

 \begin{table}[h!]
\centering\tiny\scriptsize
\caption{Leakages in RSA implemented by OpenSSL 1.0.2k}\label{tab:RSAOpenSSL1.0.2k}
%\resizebox{\columnwidth}{!}{
\begin{tabular}{lrlrr}
\hline
\textbf{File} & \textbf{Line No.} & \textbf{Function} & \textbf{\# Leaked Bits} & \textbf{Type} \\\hline
bn\_lib.c& 199&BN\_num\_bits&*&CF\\
bn\_lib.c& 200&BN\_num\_bits&9.8 &CF\\
bn\_lib.c& 201&BN\_num\_bits&12.1 &DA\\
bn\_shift.c& 168&BN\_lshift&5.3 &CF\\
bn\_lib.c& 673&BN\_ucmp&*&CF\\
bio\_asn1.c& 484&\_\_udivdi3&6.4 &CF\\
bn\_div.c& 381&BN\_div&*&CF\\
bn\_div.c& 456&BN\_div&*&CF\\
bn\_gcd.c& 279&BN\_mod\_inverse&1.0 &CF\\
bn\_gcd.c& 302&BN\_mod\_inverse&8.4 &CF\\
bn\_gcd.c& 324&BN\_mod\_inverse&8.6 &CF\\
bn\_add.c& 255&BN\_usub&*&CF\\
bn\_gcd.c& 327&BN\_mod\_inverse&14.2 &CF\\
bn\_gcd.c& 305&BN\_mod\_inverse&13.8 &CF\\
bn\_lib.c& 203&BN\_num\_bits&14.2 &DA\\
bn\_lib.c& 208&BN\_num\_bits&13.5 &CF\\
bn\_lib.c& 209&BN\_num\_bits&*&DA\\
bn\_lib.c& 212&BN\_num\_bits&*&DA\\
bn\_gcd.c& 515&BN\_mod\_inverse&*&CF\\
bn\_div.c& 381&BN\_div&8.2 &CF\\
bn\_div.c& 439&BN\_div&*&CF\\
bn\_div.c& 385&BN\_div&3.9 &CF\\
bn\_div.c& 381&BN\_div&1.0 &CF\\
bn\_div.c& 381&BN\_div&0.8 &CF\\
bn\_div.c& 469&BN\_div&1.6 &CF\\
bn\_exp.c& 716&BN\_mod\_exp\_mont\_consttime&1.0 &CF\\
bn\_exp.c& 836&BN\_mod\_exp\_mont\_consttime&1.1 &CF\\
bn\_mont.c& 262&BN\_from\_montgomery\_word&*&DA\\
bn\_mont.c& 263&BN\_from\_montgomery\_word&*&DA\\
bn\_mont.c& 264&BN\_from\_montgomery\_word&*&DA\\
bn\_mont.c& 266&BN\_from\_montgomery\_word&*&DA\\
bn\_mont.c& 276&BN\_from\_montgomery\_word&*&DA\\
bn\_mont.c& 282&BN\_from\_montgomery\_word&*&CF\\
bn\_mont.c& 201&BN\_from\_montgomery\_word&0.0 &CF\\
bn\_asm.c& 646&bn\_mul\_comba8&*&CF\\
bn\_asm.c& 787&bn\_sqr\_comba8&*&CF\\
\hline
\end{tabular}
%}
\renewcommand{\baselinestretch}{1.0}\selectfont
\end{table}

 \begin{table}[h!]
\centering\tiny\scriptsize
\caption{Leakages in RSA implemented by OpenSSL 1.1.0f}\label{tab:RSAOpenSSL1.1.0f}
%\resizebox{\columnwidth}{!}{
\begin{tabular}{l@{~~}rlr@{~~}r}
\hline
\textbf{File} & \textbf{Line No.} & \textbf{Function} & \textbf{\# Leaked Bits} & \textbf{Type} \\\hline
bn\_lib.c& 143&BN\_num\_bits\_word&*&CF\\
bn\_lib.c& 144&BN\_num\_bits\_word&*&CF\\
bn\_lib.c& 145&BN\_num\_bits\_word&17.2 &DA\\
bn\_lib.c& 1029&bn\_correct\_top&*&CF\\
bn\_lib.c& 639&BN\_ucmp&*&CF\\
ct\_b64.c& 164&\_\_udivdi3&5.9 &CF\\
bn\_div.c& 330&BN\_div&*&CF\\
bn\_gcd.c& 192&int\_bn\_mod\_inverse&1.0 &CF\\
bn\_gcd.c& 215&int\_bn\_mod\_inverse&7.9 &CF\\
bn\_gcd.c& 237&int\_bn\_mod\_inverse&8.2 &CF\\
bn\_gcd.c& 218&int\_bn\_mod\_inverse&14.9 &CF\\
bn\_gcd.c& 240&int\_bn\_mod\_inverse&9.2 &CF\\
bn\_lib.c& 147&BN\_num\_bits\_word&*&DA\\
bn\_lib.c& 152&BN\_num\_bits\_word&12.6 &CF\\
bn\_lib.c& 153&BN\_num\_bits\_word&*&DA\\
bn\_lib.c& 156&BN\_num\_bits\_word&*&DA\\
bn\_div.c& 384&BN\_div&17.2 &CF\\
bn\_div.c& 330&BN\_div&11.9 &CF\\
bn\_div.c& 334&BN\_div&3.8 &CF\\
bn\_exp.c& 622&BN\_mod\_exp\_mont\_consttime&1.0 &CF\\
bn\_exp.c& 741&BN\_mod\_exp\_mont\_consttime&1.0 &CF\\
bn\_mont.c& 138&BN\_from\_montgomery\_word&*&DA\\
bn\_mont.c& 139&BN\_from\_montgomery\_word&*&DA\\
bn\_mont.c& 140&BN\_from\_montgomery\_word&*&DA\\
bn\_mont.c& 142&BN\_from\_montgomery\_word&*&DA\\
bn\_mont.c& 152&BN\_from\_montgomery\_word&*&DA\\
bn\_asm.c& 733&bn\_sqr\_comba8&*&CF\\
bn\_asm.c& 592&bn\_mul\_comba8&*&CF\\
bn\_mont.c& 98&BN\_from\_montgomery\_word&0.0 &CF\\
bn\_div.c& 330&BN\_div&0.3 &CF\\
bn\_div.c& 330&BN\_div&0.3 &CF\\
\hline
\end{tabular}
%}
\renewcommand{\baselinestretch}{1.0}\selectfont
\end{table}

 \begin{table}[h!]
\centering\tiny\scriptsize
\caption{Leakages in RSA implemented by OpenSSL 1.1.1}\label{tab:RSAOpenSSL1.1.1}
%\resizebox{\columnwidth}{!}{
\begin{tabular}{l@{~~}rlr@{~~}r}
\hline
\textbf{File} & \textbf{Line No.} & \textbf{Function} & \textbf{\# Leaked Bits} & \textbf{Type} \\\hline
%\input{figures/form/latex/RSA-openssl-1-1-1-data.tex}
rsa\_ossl.c& 649&rsa\_ossl\_mod\_exp&1.0 &CF\\
bn\_asm.c& 592&bn\_mul\_comba8&0.3 &CF\\
bn\_exp.c& 613&BN\_mod\_exp\_mont\_consttime&1.0 &CF\\
bn\_exp.c& 745&BN\_mod\_exp\_mont\_consttime&1.0 &CF\\
\hline
\end{tabular}
%}
\renewcommand{\baselinestretch}{1.0}\selectfont
\end{table}

 \begin{table}[h!]
\centering\tiny\scriptsize
\caption{Leakages in RSA implemented by OpenSSL 1.1.1g}\label{tab:RSAOpenSSL1.1.1g}
%\resizebox{\columnwidth}{!}{
\begin{tabular}{l@{~~}rlr@{~~}r}
\hline
\textbf{File} & \textbf{Line No.} & \textbf{Function} & \textbf{\# Leaked Bits} & \textbf{Type} \\\hline
& 29& \_\_udivdi3&1.3 &CF\\
bn\_div.c& 374&bn\_div\_fixed\_top&10.0 &CF\\
bn\_asm.c& 413&bn\_sub\_words&*&CF\\
bn\_div.c& 374&bn\_div\_fixed\_top&5.7 &CF\\
rsa\_ossl.c& 654&rsa\_ossl\_mod\_exp&1.7 &CF\\
bn\_exp.c& 625&BN\_mod\_exp\_mont\_consttime&2.1 &CF\\
bn\_exp.c& 745&BN\_mod\_exp\_mont\_consttime&2.0 &CF\\
bn\_div.c& 378&bn\_div\_fixed\_top&0.0 &CF\\
\hline
\end{tabular}
%}
\renewcommand{\baselinestretch}{1.0}\selectfont
\end{table}


 \begin{table}[!ht]
\centering\tiny\scriptsize
\caption{Leakages in ECDSA implemented by OpenSSL 1.1.0f}\label{tab:t_ECDSAOpenSSL1.1.0f-nonce}
%\resizebox{\columnwidth}{!}{
\begin{tabular}{l@{~~}rlr@{~~}r}

\hline
\textbf{File} & \textbf{Line No.} & \textbf{Function} & \textbf{\# Leaked Bits} & \textbf{Type} \\\hline
bn\_lib.c& 1029&bn\_correct\_top&*&CF\\
bn\_div.c& 330&BN\_div&8.4 &CF\\
bn\_div.c& 330&BN\_div&3.2 &CF\\
bn\_lib.c& 144&BN\_num\_bits\_word&0.0 &CF\\
bn\_lib.c& 145&BN\_num\_bits\_word&2.0 &DA\\
\hline
\end{tabular}
%}
\renewcommand{\baselinestretch}{1.0}\selectfont
\end{table}

 \begin{table}[!ht]
\centering\tiny\scriptsize
\caption{Leakages in ECDSA implemented by mbedTLS 2.5}\label{tab:t_ECDSALibgcrypt2.5-nonce}
%\resizebox{\columnwidth}{!}{
\begin{tabular}{lrlrr}
\hline
\textbf{File} & \textbf{Line No.} & \textbf{Function} & \textbf{\# Leaked Bits} & \textbf{Type} \\\hline
bignum.c& 369&mbedtls\_mpi\_bitlen&7.5 &CF\\
bignum.c& 1167&mpi\_mul\_hlp&1.8 &CF\\
bignum.c& 861&mbedtls\_mpi\_cmp\_mpi&1.1 &CF\\
bignum.c& 862&mbedtls\_mpi\_cmp\_mpi&0.8 &CF\\
bignum.c& 861&mbedtls\_mpi\_cmp\_mpi&1.8 &CF\\
bignum.c& 862&mbedtls\_mpi\_cmp\_mpi&12.0 &CF\\
\hline
\end{tabular}
%}
\renewcommand{\baselinestretch}{1.0}\selectfont
\end{table}


 \begin{table}[h!]
\centering\tiny\scriptsize
\caption{Leakages in ECDSA implemented by mbedTLS 2.15.1}\label{tab:t_ECDSALibgcrypt2.15.1-nonce}
%\resizebox{\columnwidth}{!}{
\begin{tabular}{l@{~~}rlr@{~~}r}

\hline
\textbf{File} & \textbf{Line No.} & \textbf{Function} & \textbf{\# Leaked Bits} & \textbf{Type} \\\hline
bignum.c& 395&mbedtls\_mpi\_bitlen&11.7 &CF\\
bignum.c& 840&mbedtls\_mpi\_cmp\_abs&0.4 &CF\\
bignum.c& 1179&mpi\_mul\_hlp&9.7 &CF\\
bignum.c& 841&mbedtls\_mpi\_cmp\_abs&0.5 &CF\\
\hline
\end{tabular}
%}
\renewcommand{\baselinestretch}{1.0}\selectfont
\end{table}


 
 
