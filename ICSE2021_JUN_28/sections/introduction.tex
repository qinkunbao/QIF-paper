\section{Introduction} 
%% side channels are important
Side channels are ubiquitous in modern computer systems as sensitive
information can leak through many mechanisms such as power,
electromagnetic radiation, and even
sound~\cite{agrawal2002side,kar20178,chari1999towards,217605,genkin2014rsa}.
Among them, software side-channel attacks, such as cache attacks, memory page
attacks, and controlled-channel attacks, are especially problematic as they do not require physical proximity~\cite{7163052,217543,217589,lee2017inferring,191010,liu2015last}. These
attacks arises from shared micro-architectural components in a computer processor.
By observing inadvertent interactions between two programs, attackers can infer program
execution flows that manipulate secrets and guess secrets such as encryption
keys~\cite{Osvik2006,Gullasch:2011:CGB:2006077.2006784,203878,10.1007/978-3-540-45238-6_6}.

%% to deal with side channels, we can protect or detect them and detection is better
%%arious countermeasures have been proposed to defend against software-based
%%side-channel attacks. Hardware-level solutions, such as reducing shared
%%resources, adopting oblivious RAM, and using transnational
%%memory~\cite{203878,217537,shih2017t,Zhang:2015:HDL:2775054.2694372}, need new
%%hardware features and changes in modern complex computer systems, which are
%%impractical and hard to adopt in reality. Therefore, a promising and
%%universal direction is software countermeasures, detecting and eliminating
%%side-channel vulnerabilities from code base.

Many side-channel attacks originate
in two code patterns: data flow from secrets to load
addresses and data flow from secrets to branch conditions. We refer to them as
secret-dependent data accesses and control flows, respectively. 

Recent work~\cite{203878,217537,Wichelmann:2018:MFF:3274694.3274741,Brotzman19Casym,236338,182946}
can detect many side-channel vulnerabilities. For example,
DATA~\cite{217537} reports 2,248 potential leakage sites for the RSA
implementation in OpenSSL 1.1.0f\@. 
Further analysis shows 1,510 leaks can be dismissed. But that
leaves 460 data-access leaks and 278 control-flow leaks. 
Many of these vulnerabilities have not been fixed by developers for a variety of reasons.
First, some vulnerable implementations perform better. For example,
RSA implementations usually adopt the CRT optimization,
which is faster but vulnerable to fault attacks~\cite{aumuller2002fault}.
Second, fixing vulnerabilities can introduce new 
weaknesses.
Third, most vulnerabilities pose a negligible risk. 
Although some vulnerabilities result in the key being 
entirely compromised~\cite{184415, aumuller2002fault}, 
many others are less
severe in reality. Therefore, we need a proper quantification metric to 
assess the sensitivity of side-channel vulnerabilities,
so a developer can efficiently triage them.

Previous work~\cite{182946,5207642} can identify numerous leakages or even provide an upper bound on the amount of leakage, which is useful to verify that an implementation is secure if it incurs zero leakage.
However, these techniques cannot quantify the severity of a leak because they over approximate the leakage. For example, CacheAudit~\cite{182946} 
reports that the upper-bound leakage of AES-128 exceeds the original key size. Besides, existing side-channel quantification work~\cite{182946,5207642} assumes an attacker runs the target program multiple times with different
input secrets and calculates an ``average'' estimation, which is different from real attack scenarios when the secret that an attacker wants to retrieve is fixed. As a consequence, those results are less useful to assess the severity level of each leakage site.

To overcome these limitations, we propose a novel method to quantify information 
leakage precisely. We quantify the number of bits that can leak during a real 
execution and define the amount of leaked information as the cardinality of 
possible secrets based on an attacker's observation. Before an attack, an adversary has a large but finite input space. 
Whenever the adversary observes a leakage site, they can eliminate some impossible 
inputs and reduce the input space's size. In an extreme case, if the input space's size reduces to one, an adversary has determined the input, which means all secret information (e.g., the entire secret key) is
leaked. By counting the number of inputs~\cite{10.1007/11499107_24}, we can quantify the information leakage precisely.
We use symbolic execution to generate constraints to model the relation 
between the original sensitive input and an attacker's observations. 
Symbolic execution provides fine-grained information, but it is expensive
to compute. Therefore, prior symbolic
execution work~\cite{203878,236338,Brotzman19Casym} either analyzes only
small programs or applies domain knowledge~\cite{203878} to simplify the analysis. We
examine the bottleneck of the trace-oriented symbolic execution and optimize it
to work for real-world crypto-systems.

We have implemented the proposed technique in a tool called \tool{} and demonstrated 
it on real-world crypto libraries, including OpenSSL, 
mbedTLS, Libgcrypt, and Monocypher.
We collect execution traces of these libraries and apply 
symbolic execution to each instruction. We model
each side-channel leak as a logic formula. These
formulas precisely model side-channel vulnerabilities. 
Then we use the conjunction of those formulas to model the 
leaks at a statement that appears in different location in
the execution trace file (e.g., leaks inside a loop).
Finally, we introduce a Monte Carlo sampling method to estimate 
the information leakage. 
The experimental results confirm
that \tool{} precisely identifies previously known vulnerabilities and 
reports how much information is leaked and which byte in the original sensitive 
buffer is leaked. We also test \tool{} on side-channel-free algorithms. 
\tool{} produces no false positives.
The result also shows the newer version of crypto libraries leak less information 
than earlier versions.
\tool{} also discovers new vulnerabilities. With the help of \tool{}, 
we confirm that some of these vulnerabilities are severe.


%Based on the fixed attack target, we classify the software-based side-channel
%vulnerabilities into two categories: 1.\textit{secret-dependent control-flow
%transfers} and 2.\textit{secret-dependent data accesses} and model them with
%math formulas which constrain the value of sensitive information. We quantify
%the amount of leaked information as the number of possible solutions that are
%reduced after applying each constrains.


%Our method can identify and quantify address-based sensitive information
%leakage sites in real-world applications automatically. Adversaries can exploit
%different control-flow transfers and data-access patterns when the program
%processes different sensitive data. We refer them as the potential information
%leakage sites. Our tool can discover and estimate those potential information
%leakage sites as well as how many bits they can leak. We are also able to
%report precisely how many bits can be leaked in total if an attacker observes
%more than one site. We run symbolic execution on execution traces. We model
%each side-channel leakage as a math formula. The sensitive input is divided
%into several independent bytes and each byte is regarded as a unique symbol.
%Those formulas can precisely model every the side-channel vulnerability. In
%other words, if the application has a different sensitive input but still
%satisfies the formula, the code can still leak the same information.  
%Those information leakage sites may spread in the whole program and their
%leakages may not be dependent. Simply adding them up can only get a coarse
%upper bound estimate. In order to accurately calculate the total information
%leakage, we must know the dependent relationships among those multiple leakages
%sites. Therefore, we introduce a monte carlo sampling method to estimate the
%total information leakage.

In summary, we make the following contributions:

\begin{itemize}
      \item We propose a novel method that can quantify fine-grained leaked
            information from side-channel vulnerabilities that result from actual attack
            scenarios. Our approach differs from previous ones in that we
            model real attack scenarios for one execution. 
            We model the information quantification problem as a counting problem 
            and use a Monte Carlo sampling method to estimate the information leakage.
   
      \item We implement the method into a tool and apply it
            to several pieces of real software. \tool{} successfully identifies
            previous unknown and known side-channel vulnerabilities and calculates the corresponding information leakage. 
            Our results are useful in practice.
            The leakage estimates and the corresponding trigger inputs can 
            help developers to triage and fix the vulnerabilities.
\end{itemize}


