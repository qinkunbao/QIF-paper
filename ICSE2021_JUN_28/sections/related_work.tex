\section{Related Work}

There is a vast amount of work on 
side channel 
detection~\cite{182946, 236338, Brotzman19Casym, 203878,217537,Wichelmann:2018:MFF:3274694.3274741,langley2010ctgrind}, 
mitigation~\cite{Page2005PartitionedCA,
Wang:2007:NCD:1250662.1250723,Zhang:2015:HDL:2775054.2694372,Li:2014:SLH:2541940.2541947,
236344,shih2017t,Coppens:2009:PMT:1607723.1608124,
brickell2006software,crane2015thwarting}, 
information
quantification~\cite{biondi2018scalable,10.1007/978-3-642-31424-7_40,McCamantE2008,5207642,Phan:2012:SQI:2382756.2382791,Chattopadhyay:2017:QIL:3127041.3127044,zhang2010sidebuster,zhou2018static,pasareanu2016multi}, 
and model counting~\cite{wei2005new, gomes2007sampling, gomes2006model, kroc2008leveraging, Chattopadhyay:2017:QIL:3127041.3127044}.
Here we only present only work closely related to ours.
Due to space limit, we do not discuss related work on side-channel attacks.

\subsection{Detection and Mitigation}

CacheAudit~\cite{182946} uses abstract domains to compute an
over approximation of cache-based side-channel information leakage upper bound.
However, it is difficult to judge the sensitive level of the side-channel leakage based on
the leakage provided by CacheAudit. CacheS~\cite{236338} improves on
CacheAudit with new abstract domains that only track
secret-related code. Like CacheAudit, CacheS cannot
indicate the sensitive level of side-channel vulnerabilities.
CaSym~\cite{Brotzman19Casym} introduces a static cache-aware symbolic
reasoning technique to cover multiple paths in a target program. Again, their
approaches cannot evaluate the sensitive level for each side-channel
vulnerability, and it only work on small code snippets.

The dynamic approach, usually consists of taint analysis and symbolic execution,
can perform a very precise analysis. CacheD~\cite{203878} takes a concrete execution trace and runs symbolic execution on the trace
to get the formula of each memory address. Therefore, CacheD is
quite precise in avoiding false positives. However, CacheD is not able to detect secret-dependent control-flows. We adopted a similar approach to model the secret-dependent data accesses, but \tool{} also finds secret-dependent control-flows and give a precise quantification of the leakage.
DATA~\cite{217537} detects address-based side-channel vulnerabilities by comparing different execution traces under various test inputs. After collecting execution traces, DATA aligns them and finds the differences. It uses statistical hypothesis testing to find true leakages. However, both imperfect trace alignment and statistical testing result that DATA can produce false positives.
MicroWalk~\cite{Wichelmann:2018:MFF:3274694.3274741} uses
mutual information (MI) between sensitive input and execution state to detect side-channels. 

Both hardware~\cite{Page2005PartitionedCA,
Wang:2007:NCD:1250662.1250723,Zhang:2015:HDL:2775054.2694372,Li:2014:SLH:2541940.2541947,
236344, 236334} and software~\cite{shih2017t,Coppens:2009:PMT:1607723.1608124,
brickell2006software,crane2015thwarting, 197207} side-channels mitigation techniques have
been proposed recently. Hardware countermeasures, including partitioning hardware resources~\cite{Page2005PartitionedCA}, randomizing cache
accesses~\cite{Wang:2007:NCD:1250662.1250723, 236344}, and designing new
architecture~\cite{tiwari2011crafting}, require changes to complex processors and are complex to adopt. On the contrary, software approaches are
usually easy to implement. Coppens et
al.~\cite{Coppens:2009:PMT:1607723.1608124} uses a compiler
to eliminate key-dependent control-flow transfers. Crane et
al.~\cite{crane2015thwarting} mitigated side-channels by randomizing software.
As for crypto libraries, the basic idea is to eliminate key-dependent
control-flow transfers and data accesses. Common approaches include
bit-slicing~\cite{konighofer2008fast,rebeiro2006bitslice} and unifying
control-flows~\cite{Coppens:2009:PMT:1607723.1608124}.

\subsection{Quantification}

Proposed by Denning~\cite{robling1982cryptography} and Gray~\cite{gray1992toward}, 
Quantitative Information Flow (QIF) aims at providing an estimation of the amount of leaked
information from the sensitive information given the public output. If zero bits
of the information are leaked, the program is called non-interference. McCamant
and Ernst~\cite{McCamantE2008} quantify the information leakage as the network
flow capacity. Backes et al.~\cite{5207642} propose an automated method for QIF
by computing an equivalence relation on the set of input keys. But the approach
cannot handle real-world programs with bitwise operations. 
Phan et al.~\cite{Phan:2012:SQI:2382756.2382791} propose symbolic QIF. The goal of their
work is to ensure a program is non-interference. They adopt an over
approximation method to estimate the total information leakage and their method
does not work for secret-dependent memory access side-channels.
\cready{}{Pasareanu et al.~\cite{pasareanu2016multi} combine symbolic analysis and Max-SMT solving to synthesize the concrete public input ($m \in M$ in this paper) that can lead to the worst case leakage. They assume the target problem has multiple different input secrets and calculate the average leakage for one-fixed public input.}
CHALICE~\cite{Chattopadhyay:2017:QIL:3127041.3127044} quantifies the leaked
information for a given cache behavior. 
It symbolically reasons about cache
behavior and estimates the amount of leaked information based on cache miss/hit.
Their approach only scale to small programs, which limits its usage in
real-world applications. On the contrary, \tool{} assesses the sensitive level
of side-channels with different granularities. It can also analyze side-channels
in real-world crypto libraries.

\subsection{Model Counting}
Model counting refers to the problem of computing the number of 
models for a propositional formula (\#SAT). There are two approaches to
solving the problem, exact model counting and approximate model 
counting. We focus on approximate model counting since it is our approach. Wei and Selman~\cite{wei2005new} introduce
ApproxCount, a local search based method using Markov Chain Monte 
Carlo (MCMC). ApproxCount has the better scalability than 
exact model counters. Other approximate model counter includes 
SampleCount~\cite{gomes2007sampling},
Mbound~\cite{gomes2006model}, and MiniCount~\cite{kroc2008leveraging}. 
Unlike ApproxCount,
these model counters can give lower or upper bounds with guarantees.
Despite the rapid development of model counters for SAT and some 
research~\cite{chistikov2017approximate,phan2015model} on Modulo Theories model counting (\#SMT),
they cannot be directly applied to 
side channel leakage quantification.
ApproxFlow~\cite{biondi2018scalable} uses ApproxMC~\cite{chakraborty2016algorithmic} for information flow quantification,
but it has only been tested with small programs.
