\subsection{Implementation}
We implement \tool{} with 16,729 lines of code in C++ and Python. It has three
components, Intel Pin tool that can collect the execution trace, the
instruction-level symbolic execution engine, and the backend that can estimate
the information leakage. The breakdown is shown in
Table~\ref{appendix:code}. The dynamic symbolic execution engine is built from scratch.
The tool can also report the memory address of
the leakage site. To assist developers to fix the bugs, we also have several
Python scripts that can report the leakage location in the source code with the
debug information and the symbol information. A sample report can be found in
the appendix.

\label{appendix:code}
\begin{table}[h!]
    \centering
    %    \resizebox{.8\columnwidth}{!}{
    \caption{\tool{}' main components and sizes}\label{tbl:implementation}
    \begin{tabular}{lr@{~}@{}l}
        \hline
        Component            & \multicolumn{2}{c}{Lines of Code (LOC)}             \\ \hline
        Trace Logging        & 501 lines                               & of C++    \\
        Symbolic Execution   & 14,963 lines                            & of C++    \\
        Data Flow            & 451 lines                               & of C++    \\
        Monte Carlo Sampling & 603 lines                               & of C++    \\
        Others               & 211 lines                               & of Python \\ \hline
        Total                & 16,729 lines                            &           \\\hline
    \end{tabular}
    %    }
\end{table}

Our current implementation supports most Intel 32-bit instructions,
including bitwise operations, control transfer, data movement, and logic
instructions, which are essential in finding memory-based side-channel
vulnerabilities. For other instructions that the current implementation does not
support, the tool will use the real values to update the registers and memory
cells. Therefore, the tool may miss some leakages but will not give us any new
false positives. As our proposed method is architecture-independent,
there is no fundamental difficulties to extend the tool to support 64-bit
architecture. 
