\section{Background and Threat Model}

\subsection{Address-based Side-channels}
Side channels leak sensitive information
unconsciously through different execution behaviors caused by shared 
hardware components (e.g., CPU cache, TLB, and
DRAM) in modern computer systems~\cite{ge2018survey,szefer2019survey,191010,184415,Osvik2006,liu2015last,184415}. 

%For example, cached-based
%side-channels~\cite{191010,184415,Osvik2006,liu2015last,184415}
%rely on the time difference between cache misses and cache hits. We introduce two
%common attack strategies, namely Prime+Probe~\cite{liu2015last} and
%Flush+Reload~\cite{184415}. Prime+Probe targets a single cache set. An
%attacker preloads the cache set with its own data and waits until the victim
%executes the program. If the victim accesses the cache set and evicts part of
%the data, the attacker will experience a slow measurement. 
%While Flush+Reload targets a
%single cache line, it requires the attacker and victim share some memory. 
%During the ``flush'' stage, the attacker flushes the ``monitored
%memory'' from the cache and waits for the victim to access the memory,
%who will load the sensitive information to the cache line. In the next phase,
%the attacker reloads the ``monitored memory''. By measuring the time difference
%brought by cache hit and miss, the attacker can further infer the sensitive information.
%Some other types of side-channels target different hardware
%layers other than CPU cache. For example, the controlled-channel
%attack~\cite{7163052}, where an attacker works in the kernel space, can infer
%sensitive data in shielding systems by observing the page fault sequences
%after restricting some code and data pages.

\begin{figure}[]
    \noindent\begin{minipage}{0.45\linewidth}
        \noindent
        \begin{lstlisting}[numbers = none]
unsigned long long r;
int secret[32];
...
while(i>0){
    r = (r * r) % n;
    if(secret[--i] == 1)
        r = (r * x) % n;   
}
        \end{lstlisting}
\vspace*{-9pt}
        \caption{Secret-dependent control-flow transfers}
        \label{fig:secret:cf}
    \end{minipage}
    \hfill
    \begin{minipage}{0.45\linewidth}
        \begin{lstlisting}[numbers = none]
static char Fsb[256] = {...}
... 
uint32_t a = *RK++ ^ \ 
(FSb[(secret)) ^
(FSb[(secret >> 8)] << 8 ) ^
(FSb[(secret >>16)] << 16 ) ^
(FSb[(secret >>24)] << 24 );
...
        \end{lstlisting}
\vspace*{-9pt}
        \caption{Secret-dependent memory accesses}
        \label{fig:secret:da}
    \end{minipage}
\vspace*{-12pt}
\end{figure}

The key intuition is that many of those side-channel attacks happen when a
program accesses different memory addresses if the program has different
sensitive inputs. As shown in Figure~\ref{fig:secret:cf} and Figure~\ref{fig:secret:da}, 
if a program shows different patterns in
control transfers or data accesses when the program processes different
sensitive inputs, the program are vulnerable to side-channel attacks.
Different kinds of side-channels can be exploited to retrieve information at
various granularities. For example, cache channels can observe cache
accesses at the level of cache sets~\cite{liu2015last}, cache lines~\cite{184415}, 
or other granularities. 
Other kinds of side-channels like controlled-channel attack~\cite{7163052},
can observe the memory access at the level of memory pages.

%\lstinputlisting[language=c, 
%                 numbers=left,
%                 caption={Sample code shows secret-dependent memory access and 
%                          secret-dependent control-flow transfer.},
%                 captionpos=b,
%                 label={code:background},
%                 frame=single,
%                 basicstyle=\fontsize{7}{9}\selectfont\ttfamily]
%                 {sample_code/background.c}

%For example, the above code~\ref{code:background} show a simple encryption
%function that has the two kinds of side-channels. At line 11, depending on the
%value of a key, the code will access the different entry in the predefined
%table. At the line 13, the code will do a series of computation and determine
%if the code in the if branch is executed or not. Such vulnerabilities are
%called the memory-based side-channles. We identify and quantify the leakage of
%the two kinds of vulnerabilities in the paper.

\cut{

\subsection{Exisiting Information Leakage Quantification}\label{sec:background_leak}

Given an event $e$ that occurs with the probability $p(e)$, we receive
\begin{displaymath}
    I = - \log_2p(e)
\end{displaymath}
bits of information by knowing the event $e$ happens according to information theory~\cite{shannon1948mathematical}. 
Considering a char variable $a$
with one byte size in a C program, its value ranges from 0 to 255.
If we observe $a$ equals $1$, without any domain knowledge, 
the probability of this observation is $\frac{1}{256}$. 
So we get $-\log(\frac{1}{256}) = 8$ bits information, which is exactly the size
of a char variable in the C program.
Existing works on information leakage quantification typically use Shannon
entropy~\cite{clark2007static,Wichelmann:2018:MFF:3274694.3274741},
min-entropy~\cite{10.1007/978-3-642-00596-1_21}, and max-entropy~\cite{182946,
Doychev:2017:RAS:3062341.3062388}. In these frameworks, the input sensitive
information $K$ is considered as a random variable.

Let $k$ be one of the possible
value of $K$. The Shannon entropy $H(K)$ is defined as
\begin{displaymath}
    H(K) = - \sum_{k {\in} K}p(k)\log_2(p(k))
\end{displaymath}

Shannon entropy can be used to quantify the initial uncertainty about
sensitive information. It measures the amount of information in a system.

Min-entropy describes the information leaks for a program with the most likely input. 
For example, min-entropy can be used to describe the
best chance of success in guessing one's password using the
most common password. %, which is defined as
\begin{displaymath}
    \mathit{min\text{-}entropy} = - \log_2(p_{\mathit{max}})
\end{displaymath}

Max-entropy is defined solely on the number of possible observations.
%It is equal to $-\log_2{n}$.
\begin{displaymath}
    \mathit{max\text{-}entropy} = -\log_2{n}
\end{displaymath}
As it is easy to compute, most recent works~\cite{182946,Doychev:2017:RAS:3062341.3062388} use max-entropy as the definition of
the amount of leaked information.

To illustrate how these definitions work, we consider the code
fragment in Figure~\ref{fig:side-channel}. It has two possible leakage sites, A and B.

\begin{figure}[h!]
\centering
\begin{lstlisting}[xleftmargin=.03\textwidth,xrightmargin=.01\textwidth]
uint8_t key[2], t1, t2;
get_key(key);              // 0 <= key[0], key[1] < 256
t1 = key[0] + key[1];
t2 = key[0] - key[1];
if (t1 < 4)                // leakage site A
    foo();                             
if (t2 > 0)                // leakage site B     
    bar();                             
\end{lstlisting}
\vspace*{-1pt}
    \caption{Side-channel leakage}
    \label{fig:side-channel}
\end{figure}
In this paper we assume an attacker can observe the secret-dependent control-flows in 
Figure~\ref{fig:side-channel}.
Therefore, an attacker can have two different observations for each leak site
depending on the value of the $\mathit{key}$: $A$ for function \textsf{foo} is executed, 
$\neg A$ for function \textsf{foo} is not executed, $B$ for function \textsf{bar} is
executed, and $\neg B$ for function \textsf{bar} is not executed. Now the
question is how much information can be leaked from the above code if an
attacker knows which branch is executed?

\begin{table}[ht]
    \centering\small\footnotesize
    \caption{The distribution of observation}\label{shtable}
    \vspace*{-0pt}
%    \resizebox{\columnwidth}{!}{
    \begin{tabular}{l|cc|cc}
        \hline
        Observation ($o$)   & $A$ & $\neg A$ & $B$ & $\neg B$ \\ \hline
        Number of Solutions & 65526       & 10        & 32768     & 32768           \\ \hline
        Possibility (p)     & 0.9998      & 0.0002    & 0.5    & 0.5       \\
        \hline
    \end{tabular}
%        }
\end{table}

Assuming $\mathit{key}$ is uniformly distributed, we can calculate the corresponding
possibility by counting the number of possible inputs. Table~\ref{shtable}
describes the probability of each observation. We use the above three types of 
leakage metrics to quantify the leaked information for leak A and leak B.

\vspace{3pt}
\textbf{Min Entropy.}
As $p_{A\mathit{max}} = 0.9998$ and $p_{B\mathit{max}} = 0.5$, 
with the definition, $\mathit{min\text{-}entropy_A}$ equals to
$0.000\ \mathrm{bits}$ and $\mathit{min\text{-}entropy_B}$ equals to
$1.000\ \mathrm{bits}$.

\textbf{Max Entropy.}
Depending on the value of key, the code have two branches for each leakage site. 
Therefore, with the max entropy definition, both leakage sites leak $1.000\ \mathrm{bits}$.

\textbf{Shannon Entropy.}
Based on Shannon entropy, the respective amount of information in A and B equals to
{\footnotesize
\begin{align*}
    \mathit{Shannon\text{-}entropy_A} & = -(0.9998*\log_{2}0.9998      \\
                                    & \qquad+ 0.0002*\log_{2}0.0002)  \\
                                    & = 0.000\ \mathrm{bits}         \\
    \mathit{Shannon\text{-}entropy_B} & = -(0.5*\log_{2}0.5      \\
                                    & \qquad+ 0.5*\log_{2}0.5)        \\
                                    & = 1.000\ \mathrm{bits}                             
\end{align*}
}
In the next section, we will show that these measures work well only
theoretically in a static analysis setting. 
Generally, they do not apply to dynamic analysis or real
settings. We will present that the static or theoretical results could be
dramatically different from the real world, and we do need a better method to
quantify the information leakage from a practical point of view.

}

\subsection{Threat Model}
We consider an attacker shares the same hardware resource with the victim.
The attacker attempts to retrieve sensitive information via address-based
side-channel attacks. The attacker has no direct access to the memory or cache
but can probe the memory or cache at each program point. In reality, the
attacker will face many possible obstacles like the noisy observations 
on the memory or cache. However, for this project, we assume
the attacker can have noise-free observations like previous works~\cite{203878,182946,Brotzman19Casym}. 
The threat model captures most of the cache-based and memory-based side-channel attacks. 
We only consider the deterministic program for the project and assume an attacker 
has access to the source code of the target program.
