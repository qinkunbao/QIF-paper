\documentclass[10pt,conference]{IEEEtran}
\usepackage{adjustbox}
\usepackage{booktabs}

\usepackage{listings}
\usepackage{epsfig}
\usepackage{url}
\usepackage{cite}
\usepackage{fancybox}
\usepackage{amsmath}
\usepackage{amssymb}
\usepackage{amsfonts}
\usepackage{amsthm}
\usepackage{tikz}
\usepackage{multirow}
\usepackage{balance}
\usepackage{graphicx}
\usepackage{pdfpages}
\usepackage{subfig}
\usepackage[hidelinks]{hyperref}
\usepackage{xcolor}
\usepackage[ruled,linesnumbered]{algorithm2e}
\usepackage[para,online,flushleft]{threeparttable}
\usepackage{xspace}
\usepackage[scaled]{beramono}
\usepackage[T1]{fontenc}



\usepackage[english]{babel} % handle hyphenation

\usepackage{amssymb}
\usepackage{ulem}
\normalem


\newcommand{\cready}[2]{{\color{red}{\ifx&#1&\else- #1\fi}} {\color{green}{\ifx&#2&\else+ #2\fi}}}%
\renewcommand{\cready}[2]{#2}%


\let\oldemptyset\emptyset
\let\emptyset\varnothing

\lstset{
numbers=left,
frame=single,
language=C,
basicstyle=\fontfamily{fvm}\scriptsize,
showlines=true
%xleftmargin=.2\textwidth, xrightmargin=.2\textwidth,
}

\newcommand{\highlight}[1]{\colorbox{yellow}{\textbf{#1}}}
\newcommand{\fixme}[1]{\highlight{FIXME:} \emph{#1}}
\newcommand{\todo}[1]{\highlight{TODO:} \emph{#1}}
\newcommand{\Qinkun}[1]{\highlight{Qinkun's Response:} \emph{#1}}
\newcommand{\jl}[1]{\highlight{JL}{\textbf{#1}}}
\renewcommand{\jl}[1]{}

\newcommand{\cut}[1]{}
\newcommand{\replace}[2]{#2}

\newcommand{\tool}{TANA}
\renewcommand{\tool}{CleverHans}
\renewcommand{\tool}{Cygne}
\renewcommand{\tool}{Do-Re-Mi}
\renewcommand{\tool}{Ta-fa Te-fe}
\renewcommand{\tool}{Ti-ri-ti-ri}
\renewcommand{\tool}{Du-Ta-De-Ta}
\renewcommand{\tool}{\textsf{Abacus}}
\newcommand{\tana}{\tool}

\newtheorem{mydef}{Definition}
\newtheorem{theorem}{Theorem}
%\renewcommand{\baselinestretch}{0.98}\selectfont


% *** Do not adjust lengths that control margins, column widths, etc. ***
% *** Do not use packages that alter fonts (such as pslatex).         ***
% There should be no need to do such things with IEEEtran.cls V1.6 and later.
% (Unless specifically asked to do so by the journal or conference you plan
% to submit to, of course. )


% correct bad hyphenation here
\hyphenation{op-tical net-works semi-conduc-tor}


\begin{document}
%
% paper title
% can use linebreaks \\ within to get better formatting as desired
%\title{\tool{}: Precise, Scalable, and Fine-grained Side-channel Information Leakage Quantification for Production Software}
%\title{\tool{}: Precise and Scalable Fine-grained Side-channel Information Leakage Quantification for Production Software Using Program Analysis and Monte Carlo Sampling}
\title{\tool{}: Precise Side-Channel Analysis}

% author names and affiliations
% use a multiple column layout for up to three different
% affiliations
\author{\IEEEauthorblockN{Qinkun Bao\IEEEauthorrefmark{1},
Zihao Wang\IEEEauthorrefmark{1},
Xiaoting Li\IEEEauthorrefmark{1}, 
James R. Larus\IEEEauthorrefmark{2}, and
Dinghao Wu\IEEEauthorrefmark{1}}
\IEEEauthorblockA{\IEEEauthorrefmark{1}
The Pennsylvania State University}
\IEEEauthorblockA{\IEEEauthorrefmark{2}EPFL}
}

% conference papers do not typically use \thanks and this command
% is locked out in conference mode. If really needed, such as for
% the acknowledgment of grants, issue a \IEEEoverridecommandlockouts
% after \documentclass




% make the title area
\maketitle



\begin{abstract}

Side-channel attacks allow adversaries to infer sensitive information from non-functional characteristics. Prior side-channel
detection work is able to identify numerous potential vulnerabilities. 
However, in practice, many such vulnerabilities leak a negligible amount of
sensitive information, and thus developers are often reluctant to address
them. Existing tools do not provide information to evaluate a leak's severity, such as the number of
leaked bits.
  
To address this issue, we propose a new program analysis method to precisely quantify the 
leaked information in a single-trace attack through side-channels.
It can identify covert information flows in programs that expose 
confidential information and can reason about security flaws 
that would otherwise be difficult, if not impossible, for a developer to find.
We model an attacker's observation of each leakage site as a constraint.  
We use symbolic execution to generate these
constraints and then run Monte Carlo sampling to estimate the number of leaked
bits for each leakage site. By applying the Central Limit Theorem, 
we provide an error bound for these estimations.
  
We have implemented the technique in a tool called \tool{}, which
not only finds very fine-grained side-channel vulnerabilities but also estimates how many
bits are leaked. \tool{} outperforms existing dynamic side-channel
detection tools in performance and accuracy. 
We evaluate \tool{} on OpenSSL, mbedTLS, Libgcrypt, and Monocypher\@.  
Our results demonstrate that most reported vulnerabilities are difficult to
exploit in practice and should be de-prioritized by developers.
We also find several sensitive vulnerabilities that
are missed by the existing tools. We confirm those vulnerabilities with
manual checks and by contacting the developers.
\end{abstract}
% IEEEtran.cls defaults to using nonbold math in the Abstract.
% This preserves the distinction between vectors and scalars. However,
% if the conference you are submitting to favors bold math in the abstract,
% then you can use LaTeX's standard command \boldmath at the very start
% of the abstract to achieve this. Many IEEE journals/conferences frown on
% math in the abstract anyway.

% no keywords
\IEEEpeerreviewmaketitle
\pagenumbering{arabic}
\pagestyle{plain}
\section{Introduction}
%% side channels are important
Side channels are inevitable in modern computer systems as the sensitive
information may be leaked through many kinds of inadvertent behaviors, such as power,
electromagnetic radiation, and even
sound~\cite{agrawal2002side,kar20178,chari1999towards,217605,genkin2014rsa}.
Among them, software-based side channels, such as cache attacks, memory page
attacks, and controlled-channel attacks, are especially common and have been
studied for
years~\cite{7163052,217543,217589,lee2017inferring,191010,liu2015last}. These
vulnerabilities result from vulnerable software and shared hardware components.
By observing the outputs or hardware behaviors, attackers can infer the program
execution flow that manipulates secrets and guess the secrets such as encryption
keys~\cite{Osvik2006,Gullasch:2011:CGB:2006077.2006784,203878,10.1007/978-3-540-45238-6_6}.

%% to deal with side channels, we can protect or detect them and detection is better
Various countermeasures have been proposed to defend against software-based
side-channel attacks. Hardware-level solutions, such as reducing shared
resources, adopting oblivious RAM, and using transnational
memory~\cite{203878,217537,shih2017t,Zhang:2015:HDL:2775054.2694372}, need new
hardware features or changes in modern complex computer systems, which is
impractical and hard to adopt in reality. Therefore, a more promising and
universal direction is software countermeasures, detecting and eliminating
side-channel vulnerabilities from code base.

Regarding the root cause of software-based side channels, many of them originate
from the following two specific circumstances: data flow from secrets to load
addresses and data flow from secrets to branch conditions. We refer to them as
 secret-dependent memory-access and control-flow, respectively. A
central problem is how to identify these two code patterns automatically. Recent
works~\cite{203878,217537,Wichelmann:2018:MFF:3274694.3274741,Brotzman19Casym,236338,182946}
adopt static and dynamic analysis to detect side-channels. They are capable to find
potential leak sites in real-world software, but fail to report how severe each
potential leakage could be. Moreover, many of the reported vulnerabilities are typically
hard to exploit and leak very little information. For example,
DATA~\cite{217537} reports 2,246 potential leakage site for the RSA
implementation in OpenSSL\@. After some inspections, 1,510 are dismissed, but it
still leaves 460 data-access leakages and 278 control-flow leakages. For software
developers, it is hard to fix all these vulnerabilities, let alone the majority
of them are negligible. That is, some vulnerabilities can be exploited to recover the
full secret keys~\cite{184415}, but many other vulnerabilities prove to be less
severe in reality.

To assess the sensitive level of side-channel vulnerabilities, we need a proper
quantification metric. Static methods~\cite{182946,5207642}, usually with
abstract interpretation, can give a leakage upper bound, which is useful to
justify the implementation is secure when they report zero or little leakage.
However, they cannot indicate how serious the leakage is because of the
over-approximation method they apply. For example, CacheAudit~\cite{182946} reports that the upper
bound leakage of AES-128 exceeds the original key size! The dynamic methods take
another approach with a concrete input and run the program in a real
environment. Although they are very precise in terms of actual leakages, no
existing tool can precisely assess the severity of the vulnerabilities in production
software. 

To overcome these limitations, we propose a novel method to quantify information
leakage precisely. Unlike previous works that only consider the
``average'' information leakage, we study the problem based on real attack
scenarios. The average information assumes that the target program has
\emph{variable} or \emph{random} sensitive information as input when an attack is
launched. However, for real-world attacks, an adversary may run the target
problem again and over again with \emph{fixed} unknown sensitive information
as the input. Therefore, the previous threat model cannot model real attack
scenarios. In contrast, our method is more precise and fine-grained. We quantify
the amount of leaked information as the cardinality of the set of possible
inputs based on attackers' observations.


Before an attack, an adversary has a large but finite input space. Every time
when the adversary observes a leakage site, he can eliminate some potential
inputs and reduce the size of the input space. The smaller the input space is,
the more information is obtained. In an extreme case, if the size of the
input space reduces to one, the adversary can determine the input information
uniquely, which means all the secret information (e.g., the whole secret key) is
leaked. By counting the number of distinct inputs, we can quantify the
information leakage precisely.

We use constraints to model the relation between the original sensitive input
and the attacker's observations. We run the instruction level symbolic execution on the
whole execution trace to generate the constraints. Symbolic execution can
provide fine-grained information but is usually believed to be an expensive
operation in terms of performance. Therefore, existing dynamic symbolic
execution based works~\cite{203878,236338,Brotzman19Casym} either only analyze
small programs or apply some domain knowledge to simplify the execution. We
systematically analyze the bottleneck of the symbolic execution and optimize it
to be scalable to real-world cryptosystems.

We apply the above technique and build a tool called \tool{},
%\footnote{CleverHans is a horse that can ``count''. Our tool uses an advanced
%%method to count the number of leaked bits from side channels.} 
to discover potential information leakage sites as well as estimate how
many bits they can leak for each leakage site. We assume that adversaries can
exploit secret-dependent control-flow transfers and data-access patterns when
the program processes different sensitive data.
%We refer them as the potential information leakage sites. 
First, we collect the dynamic execution trace for each input of the target
libraries and then run symbolic execution on the traces. In this way, we model
each side-channel leakage as a logic formula. The sensitive input is divided into
several independent bytes, and each byte is regarded as a unique symbol. Those
formulas can precisely model side-channel vulnerabilities. Then we extend the
problem to multiple leakages and related leakages and introduce the Monte Carlo
sampling method to estimate the single and combined information leakage. 


%Based on the fixed attack target, we classify the software-based side-channel
%vulnerabilities into two categories: 1.\textit{secret-dependent control-flow
%transfers} and 2.\textit{secret-dependent data accesses} and model them with
%math formulas which constrain the value of sensitive information. We quantify
%the amount of leaked information as the number of possible solutions that are
%reduced after applying each constrains.


%Our method can identify and quantify address-based sensitive information
%leakage sites in real-world applications automatically. Adversaries can exploit
%different control-flow transfers and data-access patterns when the program
%processes different sensitive data. We refer them as the potential information
%leakage sites. Our tool can discover and estimate those potential information
%leakage sites as well as how many bits they can leak. We are also able to
%report precisely how many bits can be leaked in total if an attacker observes
%more than one site. We run symbolic execution on execution traces. We model
%each side-channel leakage as a math formula. The sensitive input is divided
%into several independent bytes and each byte is regarded as a unique symbol.
%Those formulas can precisely model every the side-channel vulnerability. In
%other words, if the application has a different sensitive input but still
%satisfies the formula, the code can still leak the same information.  
%Those information leakage sites may spread in the whole program and their
%leakages may not be dependent. Simply adding them up can only get a coarse
%upper bound estimate. In order to accurately calculate the total information
%leakage, we must know the dependent relationships among those multiple leakages
%sites. Therefore, we introduce a monte carlo sampling method to estimate the
%total information leakage.

We apply \tool{} on both symmetric and asymmetric ciphers from real-world crypto
libraries, including OpenSSL and mbed TLS\@. The experimental result confirms
that \tool{} can precisely identify previously known vulnerabilities, report
how much information is leaked and which byte in the original sensitive buffer
is leaked. Although some of the analyzed crypto libraries have a number of
side-channels, they actually leak very little information. Also, we perform the
analysis of widely deployed software countermeasures against side channels.
\tool\ also discovers new vulnerabilities. With the help of \tool{}, we confirm
that those vulnerabilities are serious.

In summary, we make the following contributions:

\begin{itemize}
      \item We propose a novel method that can quantify fine-grained leaked
            information from side-channel vulnerabilities to match real attack
            scenarios.  Our method is different from previous ones in that we
            model real attack scenarios more precisely, while the previous
            research only models the ``average'' or ``random'' case. 
            % compared to previous results and
            %%   We model each side-channel vulnerabilities as math formulas %
            %and mutiple side-channel vulnerabilities can be seen as the %
            %conjunction of those formulas, which precisely models the % program
            %semantics.

      \item We transfer the information quantification problem into a counting
            problem and use the Monte Carlo sampling method to estimate the
            information leakage. Some initial results indicate the sampling
            method suffers from the curse of dimensionality problem. Therefore, we
            design a guided sampling method and provide the
            corresponding error estimate.

      \item We implement the proposed method into a practical tool and apply it
            on several real-world software. \tool{} successfully identifies
            memory-related side-channel vulnerabilities and calculates the
            corresponding information leakage. 
            Our results are surprisingly different, much more useful in practice.
            The information leakage results
            provide detailed information that can help developers to fix the
            reported vulnerabilities.
\end{itemize}

\section{Background and Threat Model}
In this section, we first present an introduction to address-based side-channel
attacks and show many of them are caused by two specific side-channel
vulnerabilities: secret-dependent control-flow transfers and secret-dependent
memory accesses. Therefore, we will focus on identifying and quantifying those
leakages in the paper. After that, we discuss existing information leakage quantification 
metrics.

\subsection{Address-based Side-channels}
Side channels can leak sensitive information
unconsciously through different execution behaviors caused by shared 
hardware components (e.g., CPU cache, TLB, and
DRAM) in modern computer systems~\cite{ge2018survey,szefer2019survey}. Depending
on the layer causing side-channels, we can classify them into the following
types of side-channel attacks.

For example, cached-based
side-channels~\cite{yarom2017cachebleed,191010,184415,7163050,Osvik2006,liu2015last,yarom2014flush+}
rely on the time difference between cache misses and cache hits. We introduce two
common attack strategies, namely Prime+Probe~\cite{liu2015last} and
Flush+Reload~\cite{yarom2014flush+}. Prime+Probe targets a single cache set. An
attacker preloads the cache set with its own data and waits until the victim
executes the program. If the victim accesses the cache set and evicts part of
the data, the attacker will experience a slow measurement. If not, it will be
fast. By knowing which cache set the target program accesses, the attacker can
infer part of the sensitive information. While Flush+Reload targets a
single cache line, it requires the attacker and the victim share the same memory
address space. During the ``flush'' stage, the attacker flushes the ``monitored
memory'' from the cache and also waits for the victim to access the memory,
who will load the sensitive information to the cache line. In the next phase,
the attacker reloads the ``monitored memory''. By measuring the time difference
brought by cache hit and miss, the attacker can know whether the victim has
accessed the ``monitored memory'' and further infer the sensitive information.
Some other types of side-channels target different hardware
layers other than CPU cache. For example, the controlled-channel
attack~\cite{7163052}, where an attacker works in the kernel space, can infer
sensitive data in the shielding systems by observing the page fault sequences
after restricting some code and data pages.

\begin{figure}[]

    \noindent\begin{minipage}{0.45\linewidth}
        \noindent
        \begin{lstlisting}[numbers = none]
unsigned long long r;
int secret[32];
while(i>0){
    r = (r * r) % n;
    if(secret[--i] == 1){
        r = (r * x) % n;
    }
}
        \end{lstlisting}
\vspace*{-6pt}
        \caption{Secret-dependent control-flow transfers}
        \label{fig:secret:cf}
    \end{minipage}
    \hfill
    \begin{minipage}{0.45\linewidth}
        \begin{lstlisting}[numbers = none]
static char Fsb[256] = {...}
... 
uint32_t a = *RK++ ^ \ 
(FSb[(secret)) ^
(FSb[(secret >> 8)] << 8 ) ^
(FSb[(secret >>16)] << 16 ) ^
(FSb[(secret >>24)] << 24 );
...
        \end{lstlisting}
\vspace*{-6pt}
        \caption{Secret-dependent memory accesses}
        \label{fig:secret:da}
    \end{minipage}
\vspace*{-12pt}
\end{figure}

The key intuition is that each side-channel attacks above happen when a
program accesses different memory addresses if the program has different
sensitive inputs. As shown in Figure~\ref{fig:secret:cf} and Figure~\ref{fig:secret:da}, 
if a program shows different patterns in
control transfers or data accesses when the program processes different
sensitive inputs, the program could possibly have side channels vulnerabilities.
Different kinds of side-channels can be exploited to retrieve information in
various granularities. For example, many cache channels can observe cache
accesses at the level of a cache line. For most CPU, one cache line holds 64
bytes of data. Hence according to the cache associativity, the low 6 bits of the
address is irrelevant in causing those cached-based side-channels.

%\lstinputlisting[language=c, 
%                 numbers=left,
%                 caption={Sample code shows secret-dependent memory access and 
%                          secret-dependent control-flow transfer.},
%                 captionpos=b,
%                 label={code:background},
%                 frame=single,
%                 basicstyle=\fontsize{7}{9}\selectfont\ttfamily]
%                 {sample_code/background.c}

%For example, the above code~\ref{code:background} show a simple encryption
%function that has the two kinds of side-channels. At line 11, depending on the
%value of a key, the code will access the different entry in the predefined
%table. At the line 13, the code will do a series of computation and determine
%if the code in the if branch is executed or not. Such vulnerabilities are
%called the memory-based side-channles. We identify and quantify the leakage of
%the two kinds of vulnerabilities in the paper.

\subsection{Exisiting Information Leakage Quantification}\label{sec:background_leak}

Given an event $e$ that occurs with the probability $p(e)$, we receive
\begin{displaymath}
    I = - \log_2p(e)
\end{displaymath}
bits of information by knowing the event $e$ happens according to information theory~\cite{shannon1948mathematical}. 
Considering a char variable $a$
with one byte storage size in a C program, its value ranges from 0 to 255.
Assume $a$ has a uniform distribution. If we observe that
$a$ equals $1$, the probability of this observation is $\frac{1}{256}$. So 
we get $-\log(\frac{1}{256}) = 8$ bits information, which is exactly the size
of a char variable in the C program.

Existing works on information leakage quantification typically uses Shannon
entropy~\cite{Wichelmann:2018:MFF:3274694.3274741},
min-entropy~\cite{10.1007/978-3-642-00596-1_21}, and max-entropy~\cite{182946,
Doychev:2017:RAS:3062341.3062388}. In these frameworks, the input sensitive
information $K$ is considered as a random variable.

Let $k$ be one of the possible
value of $K$. The Shannon entropy $H(K)$ is defined as
\begin{displaymath}
    H(K) = - \sum_{k {\in} K}p(k)\log_2(k)
\end{displaymath}

Shannon entropy can be used to quantify the initial uncertainty about the
sensitive information. It measures the amount of information in a system.

Min-entropy describes the information leaks for a program with the most likely input. 
For example, min-entropy can be used to describe the
best chance of success in guessing one's password using the
most common password, which is defined as
\begin{displaymath}
    \mathit{min\text{-}entropy} = - \log_2(p_{\mathit{max}})
\end{displaymath}

Max-entropy is defined solely on the number of possible observations.
%It is equal to $-\log_2{n}$.
\begin{displaymath}
    \mathit{max\text{-}entropy} = -\log_2{n}
\end{displaymath}
As it is easy to compute, most recent works~\cite{182946,Doychev:2017:RAS:3062341.3062388} use max-entropy as the definition of
the amount of leaked information.

To illustrate how these definitions work, we consider the following code
fragment.

\begin{figure}[h!]
    \centering
    \begin{lstlisting}[xleftmargin=.03\textwidth,xrightmargin=.01\textwidth]
uint_8 key[2], t1, t2;
get_key(key);              // 0 <= key[0], key[1] < 256
t1 = key[0] + key[1];
t2 = key[0] - key[1];
if (t1 < 8) {
    A();                   // branch 1
}
if (t2 > 0) {              // branch 2
    B();
}
\end{lstlisting}
\vspace*{-6pt}
    \caption{Side-channel leakage}
    \label{fig:side-channel}
\end{figure}
In this paper we assume an attacker can observe whether branch 1 and branch 2 are
executed or not. Therefore, an attacker can have four different observations
depending on the value of $\mathit{key}$: $\emptyset$ for neither branch 1 nor
branch 2 is executed, $\{1\}$ for only branch 1 is executed, $\{2\}$ for only branch 2 is
executed, and $\{1, 2\}$ for both branch 1 and branch 2 are executed. Now the
question is how much information can be leaked from the above code if an
attacker knows which branch is executed?

\begin{table}[ht]
    \centering
    \caption{The distribution of observation}\label{shtable}
    \resizebox{\columnwidth}{!}{
    \begin{tabular}{l|cccc}
        \hline
        %Observation (o)     & $\emptyset$ & ${\{1\}}$ & ${\{2\}}$ & ${\{1, 2\}}$ \\ \hline
        %Number of Solutions &  32876 & 20 & 32634 & 16 \\ \hline
        %Possibility (p)     & 0.5016 & 0.0003 & 0.4980  & 0.0002   \\
        Observation ($o$)   & $\emptyset$ & ${\{1\}}$ & ${\{2\}}$ & ${\{1, 2\}}$ \\ \hline
        Number of Solutions & 32876       & 20        & 32634     & 16           \\ \hline
        Possibility (p)     & 0.5016      & 0.0003    & 0.4980    & 0.0002       \\
        \hline
    \end{tabular}
        }
\end{table}


Assuming $\mathit{key}$ is uniformly distributed, we can calculate the corresponding
possibility by counting the number of possible inputs. Table~\ref{shtable}
describes the probability of each observation. The three types of leakage metrics are
calculated as follows.

\vspace{3pt}
\textbf{Min Entropy.}
As $p_{\mathit{max}} = 0.5016$, with the definition, min-entropy equals to
\begin{displaymath}
    \mathit{min\text{-}entropy} = -\log_2{0.5016} = 0.995\ \mathrm{bits}
\end{displaymath}

\textbf{Max Entropy.}
Depending on the value of key, the code can run four different branches which
corresponding to four different observations. Therefore, with the max entropy
definition, the leakage equals to

\begin{displaymath}
    \mathit{max\text{-}entropy} = -\log_2{4} = 2.000\ \mathrm{bits}
\end{displaymath}

\vspace{3pt}
\textbf{Shannon Entropy.}
Based on Shannon entropy, the leakage equals to
{\footnotesize
\begin{align*}
    \mathit{Shannon\text{-}entropy} & = -(0.5016*\log_{2}0.5016      \\
                                    & \qquad+ 0.0003*\log_{2}0.0003  \\
                                    & \qquad+ 0.4980*\log_{2}0.4980  \\
                                    & \qquad+ 0.0002*\log_{2}0.0002) \\
                                    & = 1.006\ \mathrm{bits}
\end{align*}
}
In the next section, we will show that these measures work well only
theoretically in a static analysis setting where only assume the average 
leakage. Generally, they do not apply to dynamic analysis or practical
settings. We will present that the static or theoretical results could be
dramatically different from the real world, and we do need a better method to
quantify the information leakage from a practical point of view.

\subsection{Threat Model}
We consider an attacker shares the same hardware resource with the victim.
The attacker attempts to retrieve sensitive information via memory-based
side-channel attacks. The attacker has no direct access to the memory or cache
but can probe the memory or cache at each program point. In reality, the
attacker will face many possible obstacles, including the noisy observations,
limited observations on memory or cache. However, for this project, we assume
the attacker can have noise-free observations. The threat model captures most of
the cache-based and memory-based side-channel attacks. We only consider the
deterministic program for the project and assume an attacker has access to the
source code of the target program.

\section{\tool{} Leakage Definition}
\label{sec:trace-qif}
In this section, we discuss how \tool{} quantifies the amount of leaked
information. \tool{} adopts a dynamic-based approach to quantifying the leaked
information. We first present the limitation of existing quantification metrics.
After that, we introduce the abstract of our model and math notations for the
rest of the paper and propose our method.

\subsection{Problem Setting}
Existing static-based side-channel quantification
works~\cite{182946,Wichelmann:2018:MFF:3274694.3274741 } define information
leakage using max entropy or Shannon entropy.  If zero bit of
information leakage is reported, the program is secure. However, it is not
useful in practice if their tools report the program leaks some information.
Because their reported result is the ``average'' leakage, while in an attack
scenario, the leakage could be much severe.
%Moreover, those metrics does not apply to static method.


\begin{figure}[h!]
    \centering
    \begin{lstlisting}[xleftmargin=.03\textwidth,xrightmargin=.01\textwidth]
char key[9] = input();
if(strcmp(key, "password")){
    pass();     //branch 1
}else{
    fail();     //branch 2
}
\end{lstlisting}
\vspace*{-9pt}
    \caption{A dummy password checker}
    \label{fig:password-checker}
\end{figure}

We consider a dummy password checker shown in Figure~\ref{fig:password-checker}.
The password checker will take an 8-byte char array as the input and check if
the input is the correct password. If an attacker knows the code executes branch
$\{{1\}}$ by side-channel attacks, he can infer the password equals to
``password'', in which case the attacker can entirely retrieve the password.
Therefore, the total leaked information should be 64 bits, which equals to the
size of the original sensitive input if the code executes branch
$1$.

However, previous static-based approaches cannot precisely reflect the amount of
the leakage. According to the definition of Shannon entropy, the leakage will be
$\frac{1}{2^{64}}*\log_{2}\frac{1}{2^{64}} + \frac{2^{64}-1}{2^{64}}
*\log_{2}\frac{2^{64}-1}{2^{64}} \approx 0$ bits. Because the program has two
branches, tools based on max-entropy will report the code has $\log_2{2} = 1$
bit leakage.

Both approaches fail to tell how much information is leaked during the execution
precisely. The problem with existing methods is that they are static-based and
input values are neglected by their approaches. They assume an attacker runs
the program multiple times with many different or random sensitive inputs. As
shown in Figure~\ref{fig:gap}(a), previous models, both Shannon entropy and max
entropy, give an ``average'' estimate of the information leakage. However, it is
not the typical scenario for an adversary to launch a side-channel attack. When
a side-channel attack happens, the adversary wants to retrieve the sensitive
information, in which case the sensitive information is fixed (e.g., AES keys).
The adversary will run the attack over and over again with fixed input and 
guess the value bit by
bit, as in Figure~\ref{fig:gap}(b). We want to have a
theory for dynamic analysis that if the theory says an attack leaks $x$ bits of
secret information from a side-channel vulnerability, then $x$ should be useful
in estimating the sensitive level of the vulnerability. However, the above
methods all fail in real attack models. This is the first challenge we face
\textbf{(Challenge C1)}.

\begin{figure}
    \centering
    \includegraphics[width=.75\columnwidth]{./figures/RA.pdf}
\vspace*{-6pt}
    \caption{The gap between the real attack and previous models}\label{fig:gap}
\end{figure}


\subsection{Notations}
In the section, we give necessary definitions and notations for dealing with
programs and side-channels. We use capital letters (e.g., $A$) to represent a
set. $|A|$ represents the cardinality of the set $A$. We use corresponding lower case
letters to represent one element in the set (e.g., $a \in A$).

We assume a program ($\beta$) has $K$ as the sensitive input. $K$ should be a
finite set of keys. The program also takes known messages $M$ as the input. The
model applies to most cryptosystems. For example, the AES encryption,
$\beta$ is the encryption function. $K$ is AES key, and $M$ is messages to be
encrypted. During the execution, an adversary may have some observations ($O$)
from the program. Examples of those observations include timing, CPU usages, and
Electromagnetic signals (EM). In this paper, we consider secret-dependent
control-flows and secret-dependent memory accesses as observations.

With the above definitions, we define the following mapping between $\beta$,
$K$, $M$, and $O$:

\begin{displaymath}
    \beta(K, M) \rightarrow O
\end{displaymath}

We model a side-channel with the following way. An adversary does not have
access to $K$, but he knows $\beta$, $M$, and $O$. For one execution of a
deterministic program, once $k \in K$ and $m \in M$ are fixed, the observation
($o \in O$) should also be determined. As an attacker, he knows $\beta$, $o$,
and $m$. The attacker wants to infer the value of $k$. We use $K^o$ to denote
the set of possible $k$ values that produce the same observation:

\begin{displaymath}
    K^o = \{ k \in K \, |\, \beta(k, m) \rightarrow o\}
\end{displaymath}

Then the problem of quantifying the amount of leaked information can be
transferred into the following question.
\emph{How much uncertainty of $K$ can be reduced if an attacker knows $\beta$, $m$, and $o$?}

\subsection{Theoretical Analysis \textbf{(Solution to Challege C1)}}
Now we present our metric to quantify the amount of leaked information from
dynamic analysis.

In information theory, the mutual information (MI) is a measure of the mutual
dependence between the two variables. Here we use MI to describe the information
dependence between $K$ and $O$, which is defined as:

\begin{equation} \label{eq:1}
    I(K;O) = \sum_{k {\in} K}{\sum_{o {\in} O}{p(k, o)\log_2\frac{p(k, o)}{p(k)p(o)}}}
\end{equation}

where $P(k_i, o_i)$ is the joint discrete distribution of $K$ and $O$.
Alternatively, the mutual information can also be equivalently expressed as:

\begin{equation} \label{eq:2}
    I(K;O) = H(K) - H(K|O)
\end{equation}

$H(K|O)$ is the entropy of $K$ with the condition $O$. It quantifies the
uncertainty of $K$ given the value of $O$. In other word, the conditional
entropy $H(K|O)$ marks the uncertainty about $K$ after an adversary has gained
some observations ($O$).
\begin{equation}
    H(K|O) = - \sum_{o {\in} O} {p(o) \sum_{k {\in} K}{p(k|o)\log_2p(k|o)}}
\end{equation}

In this project, we hope to give a very precise definition of information
leakages. Suppose an attacker runs the target program with one
fixed input, we want to know how much information he can infer by observing the
memory access patterns ($o$). We come to the simple slogan
~\cite{10.1007/978-3-642-00596-1_21} %% where the information
%% leakage equals:
%% \textbf{Initial uncertainty - remaining uncertainty}
that
\begin{align*}
     & \mathit{Information\ leakage} =                                         \\
     & ~~~~~~ \mathit{Initial\ uncertainty} - \mathit{Remaining\ uncertainty}.
\end{align*}

Now we compare Eq.~(\ref{eq:2}) with the above slogan, we find $H(K)$
is the $\mathit{Initial\ uncertainty}$ and $H(K|O)$ is $\mathit{Remaining\
uncertainty}$. During a real attack, the observation ($o$) is known.  We
have $H(K|O) = H(K|o)$.

Therefore, we define the amount of leaked information as
\begin{displaymath}
    Leakage = H(K;o) = H(K) - H(K|o)
\end{displaymath}

For a program ($\beta$) without knowing any domain information, all the sensitive
input should appear equally. Therefore, for any $k \in K$, $p(k) =
\frac{1}{|K|}$. So we have
$$H(K) = \sum_{k {\in} K}\frac{1}{|K|}\log_2{|K|} = \log_2{|K|}$$
For any $k' \in K \setminus K^o$, $p(k'|o) = 0$. We get
\begin{align*}
    H(K;o) & = - \sum_{k {\in} K^o}{p(k|o)\log_2p(k|o)}                         \\
           & \qquad   - \sum_{k` {\in} (K \setminus K^o)}{p(k'|o)\log_2p(k'|o)} \\
           & = \sum_{k {\in} K^o}\frac{1}{|K^o|}\log_2{|K^o|}                   \\
           & = \log_2{|K^o|}
\end{align*}


\begin{mydef}
    \label{def}
    Given a program $\beta$ with the input set $K$,
    an adversary has the observation $o$ when the input $k{\in}K^o$.
    We denote it as
    $$\beta(K^o, m) \rightarrow	o$$

    The leakage $L_{\beta(k)\rightarrow o}$ based on the observation ($o$) is
    $$L_{\beta(k)\rightarrow o} = \log_2{|K|} - \log_2{|K^o|}$$
\end{mydef}

With the new definition, if an attacker observes that the code in
Figure~\ref{fig:password-checker} runs the branch 1, then the $K^{o^{1}} =
\{\mathrm{``password"}\}$. Therefore, the information leakage $L_{P(k)=o^{1}} =
\log_2{2^{64}} - \log_2{1} = 64$ bits, which means the key is totally leaked. If
the attacker observes the code hits branch 2, the leaked information is
$L_{P(k)=o^{2}} = \log_2{2^{64}} - \log_2{(2^{64}-1)} \approx 0$ bit.


We can also calculate the leaked information from the sample code in
Figure~\ref{fig:side-channel}. As the size of input sensitive information is
usually public. The problem of quantifying the leaked information has been
transferred into the problem of estimating the size of input key $|K^o|$ under
the condition $o \in O$. The result is shown in Table~\ref{shtable2}. We can see
that some branches or traces leak much more information than some others. In
contrast, an \emph{average} estimate based on random secret input information of
1 or 2 bits, as shown in \S\ref{sec:background_leak} and Table~\ref{shtable}, is
not very useful in practice as an attacker is able to get much more leaked
information in some attack scenarios.

\begin{table}[ht]
    \centering\small\footnotesize
    \caption{New leakage modeling results}
    \label{shtable2}
    \vspace*{-9pt}
%    \resizebox{\columnwidth}{!}{
    \begin{tabular}{l|cccc}
        \hline
        %Observation (o)     & $\emptyset$ & ${\{1\}}$ & ${\{2\}}$ & ${\{1, 2\}}$ \\ \hline
        %Number of Solutions &  32876 & 20 & 32634 & 16 \\ \hline
        %Possibility (p)     & 0.5016 & 0.0003 & 0.4980  & 0.0002   \\
        Observation ($o$)         & $\emptyset$ & ${\{1\}}$ & ${\{2\}}$ & ${\{1, 2\}}$ \\ \hline
        Number of Solutions       & 32876       & 20        & 32634     & 16           \\ \hline
        Leaked Information (bits) & 1.0         & 11.7      & 1.0       & 12.0         \\
        \hline
    \end{tabular}
%     }
\end{table}
\subsection{Our Conceptual Framework}
\label{side-channel:condition}
We now discuss how to model the observation (o), which is the direct information
that an adversary can get during the attack.

During the execution, a program ($\beta$) have many temporary values ($t_i \in
T$). Once $\beta$ (program), $k$ (secret), and $m$ (message, public) are
determined, $t_i$ is also fixed. Therefore, $ t_i = f_i(\beta, k, m)$, where $f_
i$ is a function that maps between $t_i$ and ($\beta$, $k$, $m$).

In the paper, we consider two code patterns that can be exploited by an attacker,
\emph{secret-dependent control transfers} and \emph{secret-dependent data
accesses}. In other words, an adversary has observations based on control-flows
and data accesses.

\subsubsection{Secret-dependent Control Transfers}
We think a control-flow is secret-dependent if different input sensitive keys
($K$) can lead to different branch conditions. Therefore,
We define a branch is secret-dependent if:
$$\exists k_{i1}, k_{i2} \in K, \,f_i(\beta, k_{i1}, m) \neq f_i(\beta, k_{i2}, m)$$

An adversary can observe which branch the code executes, if the branch condition
equals to $t_b$. We use the constraint $c_i : f_i(\beta, k, m) = t_b$ to model
the observation on secret-dependent control-transfers.

\subsubsection{Secret-dependent Data Accesses}
Similar to secret-dependent control transfers, a data access operation is
secret-dependent if different input sensitive keys ($K$) can lead to different
memory addresses. We use the model from CacheD~\cite{203878}. The low $L$ bits
of the address are irrelevant in side-channels.

We consider a data access is secret-dependent if:
$$\exists k_{i1}, k_{i2} \in K, \,f_i(\beta, k_{i1}, m) >> L \neq f_i(\beta, k_{i2}, m) >> L$$

If the branch condition equals to $t_b$, we can use the constraint $c_i :
f_i(\beta, k, m) >> L = t_b >> L$ to model the observation on secret-dependent
control-transfers.

With the above definitions, we can model an attacker's observation by math
formulas. For example, in Figure~\ref{fig:side-channel}, if an attacker observes
the code executes the branch 1, we have $c_5: k_1 + k_2 < 8$ to describe an
attacker's knowledge and $K^{o5} = \{k_1,\, k_2\,|\, (k_1 + k_2) < 8\}$. If an
attacker observes the code executes the branch 2, we have $c_8: k_1 - k_2 > 0$
and $K^{o8} = \{k_1,\, k_2\,|\, (k_1 - k_2) > 0\}$.

\section{Scalable to Real-world Crypto Systems}
\label{sec:scala}

In \S\ref{sec:trace-qif}, we propose an advanced information leakage definition for
realistic attack scenarios, model two types of address-based side-channel
leakages as math formulas, and quantify them by calculating the number of input
keys ($K^o$) that satisfy those math formulas. Intuitively, we can use
traditional symbolic execution to capture math formulas and model counting
to get the number of satisfying input keys ($K^o$). However, some preliminary
experiments show that the above approach suffers from unbearable costs, which
impede its usage to detect and quantify side-channel leakages in real-world
applications. In this section, we begin by discussing the bottlenecks of
applying the above approaches in real-world cryptosystems. After that, we
propose our methods.

In general, \tool{} faces the following performance and cost  challenges in
order to \emph{scale to production crypto system analysis}.
\begin{itemize}
      \item Symbolic execution (\textbf{Challenge C2})
      \item Constraint solving (\textbf{Challenge C3})
      \item Counting the number of items in $K^o$ (\textbf{Challenge C4})
\end{itemize}

\subsection{Trace-oriented Symbolic Execution}
While symbolic execution can capture fine-grained semantics of programs, it
is also notorious for its unbearable performance cost. Previous trace-oriented
symbolic execution based
works~\cite{203878,Chattopadhyay:2017:QIL:3127041.3127044} all have large
performance bottlenecks. As a result, those approaches either only apply to
small-size programs~\cite{Chattopadhyay:2017:QIL:3127041.3127044} or apply some
domain knowledge to simplify the analysis. Those tools interpret each
instruction and update memory cells and registers with formulas that
captured the semantics of the execution and search different input values that
can lead to different execution behaviors using constraint solver. We implement
the approach presented in \S\ref{sec:trace-qif} and model the side-channels as
formulas. While the tool can finish analyzing some simple cases like AES, it can
not handle complicated cases like RSA.

We observe that finding side-channels using symbolic execution is different from
traditional general symbolic execution and can be optimized to be as efficient
as other methods with approaches below.

%\subsubsection{Interpret Instructions Symbolically}
Existing binary analysis tools~\cite{shoshitaishvili2016state,
10.1007/978-3-642-22110-1_37} usually translate machine instructions into
intermediate languages (IR). The reason is that the number of machine instructions is
enormous, and the semantics of each instruction is complex. Intel Developer
Manual~\cite{intelsys} introduces more than 1000 different x86 instructions. It
is tedious and hard to implement the manual rule for each instruction. 
On the contrary,
IR typically has fewer instructions compared to the original machine ISA\@.
However, the IR layer, which predigest the implementation
and reduce the workload of those tools, also introduce significant
overhead~\cite{217563}.


First, transferring machine instructions into IR is time-consuming. For example,
REIL IR~\cite{dullien2009reil}, adopted in CacheS~\cite{236338}, has multiple
transform processes, from binary to VEX IR, BAP IR, and finally REIL IR\@. As IR
can also introduce additional conditional jump instructions, in order to
precisely identify secret-dependent control-flows, we need to rule out
conditional jump instructions introduced by IR, which is also time-consuming.
Second, IR increases the total number of instructions. For example, x86
instruction \textit{test eax, eax} transfers into 18 REIL IR instructions. If we
assume the time of symbolically executing one instruction is constant, the
design of adopting IR layers can introduce large overhead.

\vspace*{2pt}
\textbf{Our Solution to Challenge C2:}
We adopt the approach from QSYM~\cite{217563} and implement the symbolic execution
directly on the top of x86 instructions. Table~\ref{scala:ir} shows that
eliminating the IR layer can reduce the number of instructions executed during
the analysis.

\begin{table}%[ht]
      \centering%\small\footnotesize
      \caption{The number of x86,  % instructions and the number of 
             REIL IR, and VEX IR instructions on the traces of crypto programs.}
      \label{scala:ir}
      \resizebox{\columnwidth}{!}{%

            \begin{tabular}{cccc}
                  \hline
                                    & \begin{tabular}[c]{@{}c@{}}Number of\\ x86 Instructions\end{tabular} & \begin{tabular}[c]{@{}c@{}}Number of\\ VEX IR\end{tabular} & \begin{tabular}[c]{@{}c@{}}Number of\\ REIL IR\end{tabular} \\ \hline
                  AES OpenSSL 0.9.7 & $1,704$                   & $23,938$ (15x)            & $62,045$ (36x)            \\
                  DES OpenSSL 0.9.7 & $2,976$                   & $41,897$ (15x)            & $100,365$ (33x)           \\
                  RSA OpenSSL 0.9.7 & $1.6*10^7$                & $2.4*10^8$ (15x)          & $5.9*10^8$ (37x)          \\
                  RSA mbedTLS 2.5  & $2.2*10^7$                & $3.1*10^8$ (15x)          & $8.6*10^8$  (39x)         \\ \hline
            \end{tabular}
      }
\end{table}

\subsection{Constraint Solving}
As discussed in \S\ref{side-channel:condition}, the problem of identifying
side-channels can be reduced to the question below.

\begin{quote}
      \textit{Can we find two different input variables $k_1, k_2 \in K$ that
            satisfy the formula $f_a(k_1) \neq f_a(k_2)$?}
\end{quote}

Existing approach relies on satisfiability modulo theories (SMT) solvers (e.g,
Z3~\cite{DeMoura:2008:ZES:1792734.1792766}) to find satisfying $k_1$ and $k_2$.
We argue that while it is a universal approach to solving constraints with SMT
solvers, for constraints with the above formats, using custom heuristics and
testing is much more efficient in practice. Constraint solving is a decision
problem expressed in logic formulas. SMT solvers transfer the inputted SMT
formula into the boolean conjunctive normal form (CNF) and feed it into the
internal boolean satisfiability problem (SAT) solver. The translation process,
called ``bit blasting'', is time-consuming. Also, as the SAT problem is a
well-known NP-complete problem, it is also hard to deal when it comes to
practical uses with huge formulas. Despite the rapid development of SMT solvers
in recent years, constraint solving remains one of the obstacles to achieve the
scalability for real-world cryptosystems.

\vspace*{2pt}
\textbf{Our Solution to Challenge C3:}
Instead of feeding the formula $f_a(k_1) \neq f_a(k_2)$ into a SMT solver, we
just randomly pick up $k_1, k_2 \in K$ and test them if they can satisfy the
formula. Our solution is based on the following intuition. For most combination
of $(k_{1}, k_{2} )$, the formula $f_a(k_1) \neq f_a(k_2)$ holds. As long as
$f_a$ is not a constant function, such $k_1, k_2$ must exist. For example,
suppose each time we only have 5\% chance to find such $k_1, k_2$, then after we
test with different input combination with 100 times, we have $1 -
(1-0.05)^{100} = 99.6\%$ chance find such $k_1, k_2$. Such random algorithms
work well for our problem.

\subsection{Counting the Number}
\label{MCreasons}
The problem of quantifying the amount of leaked information can be reduced to
the problem of computing the number of items in $K^o$, according to
Definition~\ref{def} introduced in \S\ref{sec:trace-qif}. However, we find that while
there are various propositional model counters (e.g., \#SAT), they are not
sufficient scalable for production cryptosystem analysis.
%there is no open source modulo theories counter (\#SMT) available.

One straightforward method approximating the number of solutions is based on Monte Carlo
sampling. However, the number of satisfying values could be exponentially small.
Consider the formula $f_i\equiv{k_1} = 1\land{k_2} = 2\land{k_3} = 3\land{k_4} =
4$, where $k_1$, $k_2$, $k_3$, and $k_4$ each represents one byte in the
original sensitive input buffer, there is only one satisfying solution of total
$2^{32}$ possible values, which requires exponentially many samples to get a
tight bound. Monte Carlo method also suffers from the curse of dimensionality.
For example, the length of an RSA private key can be as long as 4096 bits. If we
take each byte (8 bits) in the original buffer as one symbol, the formula can
have as many as 512 symbols.

\vspace*{6pt}
\textbf{Our Solution to Challenge C4:}
We adopt multiple-step Monte Carlo sampling methods to count the number of
possible inputs that satisfy the logic formula groups. The key idea is to split
those constraints into several small formulas and sample them independently.
%We will introduce the method in the following subsection.

\subsection{Information Leakage Estimation}

\newcommand{\addr}[1]{{l}_{#1}}
\renewcommand{\addr}[1]{{\gamma}_{#1}}
\renewcommand{\addr}[1]{{\zeta}_{#1}}
\renewcommand{\addr}[1]{{\xi}_{#1}}

In this section, we present the algorithm to calculate the information leakage
based on Definition~\ref{def} (\S\ref{sec:trace-qif}), answering to
\textbf{Challenge C4}.

\subsubsection{Problem Statement}
For each leakage site, we model it with a math formula constraint with the
method presented in~\S\ref{side-channel:condition}. Suppose the address of the
leakage site is $\addr{i}$, we use $c_{\addr{i}}$ to denote the constraint. For
multiple leakage sites, we take the conjunction of those constraints to
represent those leakage sites.

According to the Definition~\ref{def}, to calculate the amount of leaked
information, the key is to calculate $\frac{|K|}{|K^o|}$. $K^o$ represents the
set that contains every input keys that satisfy the constraint. As the
cardinality of $K$ is known, the primary problem is to estimate the cardinality
of $K^o$. Suppose an attacker can observe $n$ leakage sites, and each leakage
site has the following constraints: $c_{\addr{1}}, c_{\addr{2}}, \ldots,
c_{\addr{n}}$ respectively. The total leakage has the constraint
$c_t({\addr{1}},{\addr{2}},\ldots,{\addr{n}}) = c_{\addr{1}} \land c_{\addr{2}}
\land \ldots \land c_{\addr{n}}$. The problem of estimating the total leaked
information can be reduced to the problem of counting the number of different
solutions that satisfies the constraint
$c_t({\addr{1}},{\addr{2}},\ldots,{\addr{n}})$. A native method for
approximating the result is to pick elements $k$ from $K$ and check if the
element also contained in $K^o$. Assume $q$ elements satisfy this condition. In
expectation, we can use $\frac{k}{q}$ to approximate the value of
$\frac{|K|}{|K^o|}$.

However, as discussed in \S\ref{MCreasons}, the above sampling method will
typically fail in practice due to the following two problems:

\begin{enumerate}
      \item The curse of dimensionality. $c_t({\addr{1}},\ldots,{\addr{n}})$ is
            the conjunction of many constraints. Therefore, the input variables
            of each constraints will also be the input variables of the
            $c_t({\addr{1}},\ldots,{\addr{n}})$. The sampling method will fail
            as $n$ increases. For example, if the program has $2$ byte input
            equals to 2, the whole search space is a $256^2$ cube. If we want
            the sampling distance between each point equals to $d$, we need
            $256^2d$ points. If the program has $10$ byte input, we need
            $256^{10}d$ points if we still we want the sampling distance equals
            to $d$.

      \item The number of satisfying assignments could be exponentially small.
            According to Chernoff bound, we need exponentially many samples to
            get a tight bound. On an extreme situation, if the constraint only
            has one unique satisfying solution, the simple Monte Carlo method
            cannot find the satisfying assignment even after sampling many
            points.
\end{enumerate}

However, despite the two problems, we also observe two characteristics of the
problem:
\begin{enumerate}
      \item $c_t({\addr{1}},{\addr{2}},\ldots,{\addr{n}})$ is the conjunction of
            several short constraints $c_{\addr{i}}$. The set containing the
            input variables of $c_{\addr{i}}$ is the subset of the input
            variables of $c_t({\addr{1}},{\addr{2}},\ldots,{\addr{n}})$. Some
            constraints have completely different input variables from other
            constraints.

            \item Each time when we sample $c_t({\addr{1}},{\addr{2}},\ldots,{\addr{n}})$
            with a point, the sampling result is \emph{Satisfied} or not \emph{Not Satisfied}.
            The result is randomly generated in a way that does not depend on the result in 
            previous experiments. Also, as the amount of leaked information is calculated
            by $\log$ function, we do not need to precisely count the number of solutions for
            a given constraint.
            %\item For each constraint $F(C_{{addr}_i})$, the satisfying assignments
            %are close to each other, which means if we find one satisfying assignment, we 
            %are more likely to find other satisfying assignments nearby than randomly
            %pick one point in the whole searching space.

\end{enumerate}

In regard to the above problems, we present our methods. First, we split
$c_t(\addr{1},\addr{2},\ldots,\addr{n})$ into several independent constraint
groups. After that, we run a multi-step sampling method for each constraint.

\subsubsection{Maximum Independent Partition}

For a constraint $c_{\addr{i}}$, we define function $\pi$, which maps the
constraint into a set of different input symbols. For example, $\pi(k1 + k2 >
128) = \{k1, k2\}$.

\begin{mydef}[]
      \label{independentC}
      Given two constraints $c_m$ and $c_n$, we call them independent iff
      $$\pi(c_m) \cap \pi(c_n) = \emptyset$$
\end{mydef}

Based on the Definition~\ref{independentC}, we can split the constraint
$c_t(\addr{1},\addr{2},\ldots,\addr{n})$ into several independent constraints.
There are many partitions. For our project, we are interested in the following
one.

\begin{mydef}\label{Goodpartition}
      For the constraint $c_t(\addr{1},\addr{2},\ldots,\addr{n})$,
      we call the constraint group
      $g_{1}, g_{2}, \ldots, g_{m}$
      the maximum independent partition of $c_t(\addr{1},\addr{2},\ldots,\addr{n})$ iff
      \begin{enumerate}
            \item $g_{1} \land g_{2} \land \ldots \land g_{m} = c_t(\addr{1},\addr{2},\ldots,\addr{n})$
            \item $\forall \quad i, j \in \{1, 2, 3, \ldots, m\} \quad \textrm{and} \quad
                        i \neq j, \quad \pi(g_{i}) \cap \pi(g_{j}) = \emptyset $
            \item For any other partitions  $h_{1}, h_{2}, \ldots, h_{m'}$ satisfy 1) and
                  2), $m \geq m'$
      \end{enumerate}

\end{mydef}

The reason we want a good partition of the constraints is that we want to reduce
the dimensions. Consider the example in the previous section,
$$c: ({k_1} = 1)\land({k_2} = 2)\land({k_3} > 4)\land({k_3} - {k_4} > 10)$$ A
good partition of $F$ would be
$$g_{1}: ({k_1} = 1)\quad g_{2}: ({k_2} = 2)\quad g_{3}: ({k_3} > 4) \land
({k_3} - {k_4} > 10)$$ So instead of sampling in the four dimension space, we
can sample each constraint in the less dimension space and combine them together
with Theorem~\ref{IndependentConstraint} .

\begin{theorem}
      \label{IndependentConstraint}
      Let $g_{1}, g_{2}, \ldots, g_{m}$ be a maximum independent partition of
      $c_t(\addr{1},\addr{2},\ldots,\addr{n})$.
      $K_c$ is the input set that satisfies constraint $c$. We can have the following
      equation in regard to the size of $K_c$
      $$|K_{c_t(\addr{1},\addr{2},\ldots,\addr{n})}| = |K_{g_{1}}|*|K_{g_{2}}|*\ldots*|K_{g_{n}}|$$
\end{theorem}

With Theorem~\ref{IndependentConstraint}, we can transfer the problem of
counting the number of solutions to a complicated constraint in high-dimension
space into counting solutions to several small constraints. The algorithm to
compute the Maximum Independent Partition of the
$c_t(\addr{1},\addr{2},\ldots,\addr{n})$ is shown in
Appendix~\ref{appendix:partition}.

%% We apply the following
%% algorithm~\ref{algo:max-inde} to get the Maximum Independent Partition of the
%% $c_t(\addr{1},\addr{2},\ldots,\addr{n})$.

%% \IncMargin{1em}
%% \begin{algorithm}[h]
%%       \DontPrintSemicolon
%%       \SetKwInOut{Input}{input}\SetKwInOut{Output}{output}
%%       \Input{$c_t(\addr{1},\addr{2},\ldots,\addr{n}) = c_{\addr{1}} \land c_{\addr{2}} \land \ldots \land c_{\addr{m}}$}
%%       \Output{The Maximum Independent Partition of $G = \{g_{1}, g_{2}  , \ldots,  g_{m} \}$ }
%%       \For{$i\leftarrow 1$ \KwTo $n$}
%%       {
%%             $S_{c_i}$ $\leftarrow$ $\pi(c_{\addr{i}})$ \;
%%             \For{$g_{i} \in G$}
%%             {
%%                   $S_{g_i}$ $\leftarrow$ $\pi(g_{i})$ \;
%%                   $S$ $\leftarrow$ $S_{C_i} \cap S_{G_i}$  \;
%%                   \If{$S \neq \emptyset$}
%%                   {
%%                         $g_{i} \leftarrow g_{i} \land g_{\addr{i}}$ \;
%%                         \textbf{break} \;
%%                   }
%%                   Insert $c_{\addr{i}}$ to $G$
%%             }
%%       }
%%       \caption{The Maximum Independent Partition}
%%       \label{algo:max-inde}
%% \end{algorithm}
%% \DecMargin{1em}

\subsubsection{Multiple Step Monte Carlo Sampling}

After we split those constraints into several small constraints, we count the
number of solutions for each constraint. Even though the dimension has been
significantly reduced after the previous step, this is still a \#P problem. For our
project, we apply the approximate counting instead of exact counting for two
reasons. First, we do not need to have a very precise result of the exact number
of total solutions since the information is defined with a logarithmic function.
We do not need to distinguish between constraints having $10^{10}$ or $10^{10} +
10$ solutions; they are very close to after taking logarithmic. Second, the
precise model counting approaches, like Davis-Putnam-Logemann-Loveland (DPLL)
search, have difficulty scaling up to large problem sizes.

We apply the ``counting by sampling" method. The basic idea is as follows. For
the constraint $g_{i}= c_{i_1} \land c_{i_2} \land ,\ldots, \land c_{i_j} \land
,\ldots, \land c_{i_m}$, if the solution satisfies $g_{i}$, it should also
satisfies any constraint from $c_{i_1}$ to $c_{i_m}$. In other words, $K_{c_gi}$
should be the subset of $K_{c_1}$, $K_{c_2}$, \ldots , $K_{c_m}$. We notice that
$c_i$ usually has less numbers of input compared to $g_{i}$. For example, if
$c_{i_j}$ has only one 8-bit input variable, we can find the exact solution set
$K_{c_{i_j}}$ of $c_{i_j}$ by trying every possible 256 solutions. After that,
we can only generate random input numbers for the rest input variables in
constraint $g_{i}$. With this simple yet effective trick, we can reduce the number of input
while still ensure the accuracy.
The detailed algorithm is shown in Appendix~\ref{appendix:montecarlo}.

%% the algorithm
%\IncMargin{1em}
%\begin{algorithm}
%\SetAlgoLined
%\DontPrintSemicolon

%\KwIn{{The constraint $G_{i}= C_{i_1} \land C_{i_2}
%\land \ldots \land C_{i_m}$}}    
%\KwOut{{The number of assignments that satisfy $G_{i}$ $|K_{G_{i}}|$}}

%\SetKwProg{RW}{RandomWalk}{}{}
%\SetKwProg{MM}{MetropolisMove}{}{} 
%$n$: the number of sampling times \;
%$P$: a probability generator \;
%$k$: the input assignment \;
%$n_{s}$: the number of satisfying assignments \;
%$\#k$: the satisfying number of k \fixme{this number is not used syntactically} \; 
%Initialization: \;
%$\#{k_0}$ $\leftarrow$ $\sum_{j=1}^{m}C_{i_j}(k_0)$ \;
%\For{$t\leftarrow 1$ \KwTo $n$} {
%      $p$ $\leftarrow$ $P$ \;
%      \If{$p \geq 0.5$}
%      {
%        $v$ $\leftarrow$ \RW{$v$} {}
%      }
%      \Else{
%            $v$ $\leftarrow$ \MM{$v$} {}
%      }
%      \If{$v$ satisfies $G_{i}$}
%      {$n_{s}$ $\leftarrow$ $n_{s} + 1$}
%}
%$|K_{G_{i}}|$ $\leftarrow$ $n_s|K| / n$
%\caption{Metropolis Sampling}
%\end{algorithm}
%\DecMargin{1em}

\subsubsection{Error Estimation}
\label{sssec:errest}
In this part, we analyze the accuracy of the result from the Monte Carlo
approximation. We use the central limit theorem (CLT) and uncertainty
propagation theorem to estimate errors of the number of leaked bits for each
site.

Let $n$ be the number of samples and $n_s$ be the number of samples that satisfy
the constraint $C$. Then we can get $\hat{p} = \frac{n_s}{n}$. If we repeat the
experiment multiple times, each time we can get a $\hat{p}$. As each
$\hat{p}$ is independent and identically distributed, according to the central limit
theorem, the mean value should follow normal distribution.
$$ \frac{\bar{p}-E(p)}{\sigma\sqrt{n}} \rightarrow N(0,1) $$ Here $E(p)$ is the
mean value of $p$, and $\sigma$ is the standard variance of $p$. If we use the
observed value $\hat{p}$ to the describe standard deviation. We can claim that
we have 95\% confidence that the error $\Delta p= \bar{p} - E(p)$ falls in the
interval:
$$ |\Delta p| \leq 1.96\sqrt{\frac{ \hat{p} (1- \hat{p} )}{n}}$$

Since we use $L = \log_{2}p$ to estimate the amount of leaked information, we
can have the following error propagation formula $\Delta L = \frac{\Delta
p}{p\ln2}$ by differentiation. For \tool, we want the error of estimated leaked
information ($\Delta L$) to be less than 1 bit. So we can get $\frac{\Delta
p}{p\ln2} \leq 1$. As long as $ n \geq \frac{1.96^2(1-p)}{p(\ln2)^2}$, we have
95\% confidence that the error of estimated leaked information is less than 1 bit.
During the simulation, if $n$ and $p$ satisfy the above inequation, the Monte Carlo
simulation will terminate.

%\section{Design}
\section{Design and Implementation}

%\subsection{Overview}
%\subsection{Workflow}
\subsection{Design}
\tool{} has three steps,
as shown in Figure~\ref{fig:workflow}. First, we run the target program with a
concrete input (sensitive information) under the dynamic binary instrumentation
(DBI) frameworks to collect execution traces. After that, we run the symbolic
execution to capture the fine-grained semantic information of each
secret-dependent control-flow transfers and data-accesses. Finally, we run Monte
Carlo (MC) simulations to estimate the amount of leaked information.

\begin{figure}[t]
    \centering
    \includegraphics[width=0.45\textwidth]{./figures/workflow.pdf}
    \caption{The workflow of \tool{}.}
    \label{fig:workflow}
    \vspace*{-15pt}
\end{figure}

\begin{enumerate}
    \item \emph{Execution trace generation.} The design goal of \tool{} is to
          estimate the information leakage as precisely as possible. 
          We run the target binary under dynamic binary instrumentations (DBI)
          to record execution traces and the runtime information.
          Once sensitive information is loaded into memory, we start to collect the 
          trace.
    \item \emph{Instruction level symbolic execution.} We model attackers'
          observations from side-channel vulnerabilities with logic formulas.
          Each formula captures the fine-grained information between input
          secrets and leakage sites. The engine only symbolically executes the
          instruction that might be affected by the input key.  
    \item \emph{Leakage estimation.} We transfer the information leakage 
          quantification problem into the counting problem. We propose a
          Monte Carlo method to estimate the number of satisfying solutions.
          With the help of the Central Limit Theorem (CLT), we also give an error 
          estimate with the probability, which provides us with the \emph{precision guarantee}.

\end{enumerate}


%% \subsection{Trace Logging}
%% The trace information can be logged via some emulators (e.g., QEMU) or 
%% dynamic binary instrumentation tools (DBI). 
%% We run a program with the concrete input under the DBI to record
%% execution traces.
%% The trace data has the following information:
%% \begin{itemize}
%%     \item Each instruction mnemonics and its memory address.
%%     \item The operands of each instruction and their concrete values during the 
%%           runtime.
%%     \item The value of EFLAGS register. 
%%     \item The memory address and the length of the sensitive information.
%%      Most crypto libraries stores sensitive information in arrays,
%%      variables or contiguous buffer.
%% \end{itemize}

%% \subsection{Instruction Level Symbolic Execution}
%% \label{InstructionSE}
%% The primary purpose of the step is to generate constraints of the input 
%% sensitive information from the execution trace. 
%% If we give the target program a new input which 
%% is different from the original input that was used 
%% to generate the execution trace but still satisfies those constraints,
%% as an attacker, he will have the same observations on
%% control-flow transfer and 
%% data-access patterns.

%% The tool runs symbolic execution on top of execution traces.
%% At the beginning of the symbolic execution, the tool creates new 
%% symbols for each byte in the raw buffer. For other data in the 
%% register or memory at the beginning, we use actual values from the 
%% runtime information collected during the runtime. 
%% During the symbolic execution for each instruction, 
%% the tool updates every variable in the memory and registers with a
%% math formula. The formula is made up of concrete values and the input 
%% key as the symbols accumulated through the symbolic execution.
%% For each formula, the tool will check weather it can be reduced
%% into a concrete values (e.g., $k_1+12-k_1 = 12$ ). 
%% If so, the tool will only use the concrete values in the 
%% following symbolic execution.

%% \subsubsection{Verification and Optimization}
%% We run the symbolic execution (SE) on top of x86 instructions.
%% In other words, we do not rely on any intermediate languages to simplify the implementation of symbolic execution. 
%% While the implementation itself has a lot of benefits (Better performance, accurate memory model), 
%% we need to implement the symbolic execution rules for each x86 instruction. 
%% However, due to the complexity of x86, it is inevitable to make mistakes. 
%% Therefore, we verify the correctness of the SE engine during the execution. 
%% The tool will collect the runtime information (Register values, 
%% memory values) and compare them with the formula generated from the symbolic execution. Whenever the tool finishes the symbolic execution of each instruction, the tool will compare the formula for each symbol and its actual value. If the two values do not match, we check the code
%% and fix the error. Also, if the formula does not contain any symbols,
%% the tool will use the concrete value instead of symbolic execution.

%% \subsubsection{Secret-dependent control-flows}
%% An adversary can infer sensitive information from secret dependent control-flows. 
%% There are two kinds of control-transfer instructions: the unconditional 
%% control-transfer instructions and the conditional transfer instructions.
%% The unconditional instructions, like CALL, JUMP, RET transfer control
%% from one code segment location to another. Since the transfer is independent of the input sensitive information, an attacker was not able to infer any sensitive information from the control-flow. 
%% So the unconditional control-transfer does not leak any information based on our threat model. During the symbolic execution, 
%% we update the register information and memory cells with new formulas accordingly.

%% The conditional control-flow transfer instructions, like conditional jumps,
%% depending on CPU states, may or may not transfer control flows.
%% For conditional jumps, the CPU will test if certain condition flag 
%% (e.g., CF = 0, ZF =1) is met and jump to certain branches, respectively.
%% The symbolic engine will compute the flag and represent the flag in a symbol 
%% formula. Because we are running on a symbolic execution on an execution trace, 
%% we know which branch is executed.
%% If a conditional jump uses the CPU status flag, we will generate the constraint 
%% accordingly.


%% %\begin{figure}[ht]
%% %      \centering
%% %      \includegraphics[width=\columnwidth]{./figures/secretCF.pdf}
%% %      \caption{The workflow of \tool{}. \fixme{fix the caption, fix the drawing.} \fixme{duplicate label.}}
%% %      \label{fig:Test-------------------}
%% %  \end{figure}

%% For examples,

%% \begin{lstlisting}
%% ...
%% 0x0000e781      add dword [local_14h], 1
%% 0x0000e785      cmp dword [local_14h], 4
%% 0x0000e789      jne 0xe7df
%% 0x0000e78b      mov dword [local_14h], 0
%% ...
%% \end{lstlisting}

%% At the beginning of the instruction segment, the value at the 
%% address of local\_14h can be written as $F(\vec{K})$. At the address $e785$, 
%% the value will be updated with $F(\vec{K})+1$. Then the code compares 
%% the value with 4 and use the result as a conditional jump. 
%% Based on the result, we can have the following formula:

%% $$F(\vec{K}) + 1 = 4$$

%% The formula, together with the memory address (0xe789) is store
%% as a \textit{formula tuple (address, formula)}. 
%% Each formula tuple represents one leakage site.

%% \subsubsection{Secret-dependent data access}
%% Like input-dependent control-flow transfers, an adversary can also infer 
%% sensitive information from the data access pattern as well. 
%% We try to find this kind of leakages by checking 
%% every memory operand of the instruction. We generate the memory addressing 
%% formulas. As discussed before, every symbols in the formula is the input key. 
%% If the formula does not contain any symbols, the memory access is independent 
%% from the input sensitive information and will not leak any sensitive information 
%% according to our threat model. Otherwise, we will generate the constraint for
%% the memory addressing. We model the memory address with a symbolic formula 
%% $F(\vec{K})$. 
%% Because we also have the concrete value of the memory address $Addr1$. 
%% Inspired by the work from~\cite{203878}, the formula can be written as:
%% $F(\vec{K}) >> L = Addr1 >> L$

%% $L$ represents the minimum memory address granularity that an attacker 
%% can observe. For example, Flush and Reload can distinguish between different
%% cache lines, which means the value of L is 6.

%% \subsubsection{Information Flow Check}
%% \tool{} is designed to help software developers find and understand the 
%% side-channel vulnerabilities. To ease the procedure of fixing the bug,
%% we also track the information flow for each byte of the input
%% buffer. 
%% The step can be seen as the multiple-tag taint analysis.
%% With the help of the information from symbolic execution, we can implement 
%% a relatively simple but relatively precise information flow track.
%% At the beginning of the analysis, \tool{} keep a track for each byte in 
%% the original buffer. When \tool{} symbolically executes each
%% instruction in the trace, it will check every value read from
%% registers or memory. If the value is concrete, it means the
%% instruction has nothing to do with the original buffer.
%% If the value is a formula, it means the original information passes through 
%% the instruction. Since each byte in the sensitive
%% buffer is represented as a symbol with a unique ID, \tool{} can
%% know which byte in the origin buffer goes through the
%% instruction.



\subsection{Implementation}
\tool{} consists of 16,729 lines of code in C++17 and Python. It has three
components: an Intel Pin tool that collects the execution trace, the
instruction-level symbolic execution engine, and the backend that estimates
the information leakage. 

Our current implementation supports most Intel 32-bit instructions that are essential to find memory-based side-channel
vulnerabilities, including bitwise operations, control transfer, data movement, and logic
instructions. The tool uses the real values to update the registers and memory for instructions that the implementation does not support. 
Therefore, the tool may miss some leakages but will not raise false positives. 

\section{Evaluation}
\label{res_overview}

\begin{table*}[h]
    \centering
    \caption{Evaluation results overview. We evaluate two different versions of mbed TLS, five different
        versions of OpenSSL and Monocypher 3.0\@. CF represents secret-dependent control-flow transfers and
        DF represents secret-dependent data-flow transfers. A summary of vulnerabilities 
        with the amount of leaked information can be found in Appendix \S\ref{sec:result-table}.
    }\label{table:over_result}
    \newlength{\x}
    \newlength{\y}
    \settowidth{\x}{~~}
    \settowidth{\y}{m}
    \addtolength{\x}{-1\y}
    \newcommand{\foo}{\mbox{\hspace*{\the\x}}}
    \resizebox{1.9\columnwidth}{!}{

    \begin{tabular}{llrrrrrrr}
        \hline
        \textbf{Algorithm} & \textbf{Version}  & \textbf{Leakage Sites} & \textbf{CF}         & \textbf{DF}
                           & \textbf{\# Instructions} & \textbf{Max Leakage}   & \textbf{Sym.\ Exe.} & \textbf{Monte Carlo}                                                    \\\hline
                           &                          &                        &                     &                      &             & bits & ms        & ms              \\\cline{7-9}
        AES                & Mbed TLS 2.5             & 68                     & 0                   & 68                   & 39,855      & 8.6  & 512 ~~    & 1,052 ~~        \\
        AES                & Mbed TLS 2.15            & 68                     & 0                   & 68                   & 39,855      & 9.1  & 520 ~~    & 1,057 ~~        \\
        AES                & OpenSSL 0.9.7            & 75                     & 0                   & 75                   & 1,704       & 10.6 & 231 ~~    & 9,199 ~~        \\
        AES                & OpenSSL 1.0.2f           & 88                     & 0                   & 88                   & 1,350       & 12.0 & 36 ~~     & 1,924 ~~        \\
        AES                & OpenSSL 1.0.2k           & 88                     & 0                   & 88                   & 1,350       & 12.5 & 35 ~~     & 1,961 ~~        \\
        AES                & OpenSSL 1.1.0f           & 88                     & 0                   & 88                   & 1,420       & 12.6 & 36 ~~     & 2,161 ~~        \\
        AES                & OpenSSL 1.1.1            & 88                     & 0                   & 88                   & 1,586       & 4.4  & 43 ~~     & 1,631 ~~        \\
        DES                & Mbed TLS 2.5             & 15                     & 0                   & 15                   & 4,596       & 1.1  & 58 ~~     & 162 ~~          \\
        DES                & Mbed TLS 2.15            & 15                     & 0                   & 15                   & 4,596       & 1.0  & 57 ~~     & 162 ~~          \\
        DES                & OpenSSL 0.9.7            & 6                      & 0                   & 6                    & 2,976       & 7.6  & 163 ~~    & 4,677       ~~  \\
        DES                & OpenSSL 1.0.2f           & 8                      & 0                   & 8                    & 2,593       & 9.8  & 166 ~~    & 6,509       ~~  \\
        DES                & OpenSSL 1.0.2k           & 8                      & 0                   & 8                    & 2,593       & 10.1 & 165 ~~    & 5,975        ~~ \\
        DES                & OpenSSL 1.1.0f           & 8                      & 0                   & 8                    & 4,260       & 8.8  & 182 ~~    & 5,292        ~~ \\
        DES                & OpenSSL 1.1.1            & 6                      & 0                   & 6                    & 8,272       & 7.5  & 229 ~~    & 5,152       ~~  \\
                           &                          &                        &                     &                      &             &      & seconds   & seconds               \\\cline{8-9}
        Chacha20           & Monocypher 3.0           & 0                      & 0                   & 0                    & 149,353     & 0    & 2 ~~    & 0           ~~  \\  
        Poly1305           & Monocypher 3.0           & 0                      & 0                   & 0                    & 1,213,937   & 0    & 15 ~~   & 0           ~~  \\
        Argon2i            & Monocypher 3.0           & 0                      & 0                   & 0                    & 4,595,142   & 0    & 37 ~~   & 0           ~~  \\
        Ed25519              & Monocypher 3.0           & 0                      & 0                   & 0                    & 5,713,619   & 0    & 271 ~~  & 0           ~~  \\
                           &                          &                        &                     &                      &             &      & minutes   & minutes         \\\cline{8-9}
        RSA                & Mbed TLS 2.5             & 6                      & 6                   & 0                    & 22,109,246  & 9.6  & 39 ~~     & 41  ~~          \\
        RSA                & Mbed TLS 2.15            & 12                     & 12                  & 0                    & 24,484,441  & 8.7  & 44 ~~     & 251  ~~         \\
        RSA                & OpenSSL 0.9.7            & 107                    & 105                 & 2                    & 17,002,523  & 17.2 & 23 ~~     & 428 ~~          \\
        RSA                & OpenSSL 1.0.2f           & 38                     & 27                  & 11                   & 14,468,307  & 16.2 & 29 ~~     & 436  ~~         \\
        RSA                & OpenSSL 1.0.2k           & 36                     & 27                  & 9                    & 15,285,210  & 14.2 & 40 ~~     & 714   ~~        \\
        RSA                & OpenSSL 1.1.0f           & 31                     & 22                  & 9                    & 16,390,750  & 17.2 & 34 ~~     & 490 ~~          \\
        RSA                & OpenSSL 1.1.1            & 27                     & 20                  & 7                    & 18,207,020  & 14.9 & 7 ~~      & 501 ~~          \\
        Total              &                          & 886                    & 219                 & 667                  & 139,733,961 &      & 214m \foo & 2,861m \foo     \\\hline
        %                    &                          &                       &                     &                      &             & bits & ms        & ms              \\\cline{7-9}
        % AES                & Mbed TLS 2.5             & 68                    & 0                   & 68                   & 39,855      & 8    & 570 ~~    & 850 ~~          \\
        % AES                & Mbed TLS 2.15            & 68                    & 0                   & 68                   & 39,855      & 8    & 550 ~~    & 829 ~~          \\
        % AES                & OpenSSL 0.9.7            & 75                    & 0                   & 75                   & 1,704       & 10   & 319 ~~    & 7,720 ~~        \\
        % AES                & OpenSSL 1.0.2f           & 88                    & 0                   & 88                   & 1,350       & 12   & 72 ~~     & 1,500 ~~        \\
        % AES                & OpenSSL 1.0.2k           & 88                    & 0                   & 88                   & 1,350       & 11   & 83 ~~     & 1,441 ~~        \\
        % AES                & OpenSSL 1.1.0f           & 88                    & 0                   & 88                   & 1,420       & 12   & 87 ~~     & 1,454 ~~        \\
        % AES                & OpenSSL 1.1.1            & 88                    & 0                   & 88                   & 1,586       & 8    & 91 ~~     & 1,250 ~~        \\
        % DES                & Mbed TLS 2.5             & 15                    & 0                   & 15                   & 4,596       & 1    & 114 ~~    & 144 ~~          \\
        % DES                & Mbed TLS 2.15            & 15                    & 0                   & 15                   & 4,596       & 1    & 106 ~~    & 137 ~~          \\
        % DES                & OpenSSL 0.9.7            & 6                     & 0                   & 6                    & 2,976       & 7    & 149 ~~    & 4,193       ~~  \\
        % DES                & OpenSSL 1.0.2f           & 8                     & 0                   & 8                    & 2,593       & 9    & 239 ~~    & 5,311       ~~  \\
        % DES                & OpenSSL 1.0.2k           & 8                     & 0                   & 8                    & 2,593       & 9    & 235 ~~    & 5,080        ~~ \\
        % DES                & OpenSSL 1.1.0f           & 8                     & 0                   & 8                    & 4,260       & 9    & 256 ~~    & 5,027        ~~ \\
        % DES                & OpenSSL 1.1.1            & 6                     & 0                   & 6                    & 8,272       & 7    & 235 ~~    & 4,584       ~~  \\
        %                    &                          &                       &                     &                      &             &      & minutes   & minutes         \\\cline{8-9}
        % RSA                & Mbed TLS 2.5             & 6                     & 6                   & 0                    & 22,109,246  & 9    & 38 ~~     & 20  ~~          \\
        % RSA                & Mbed TLS 2.15            & 12                    & 0                   & 12                   & 24,484,441  & 9    & 39 ~~     & 241  ~~         \\
        % RSA                & OpenSSL 0.9.7            & 105                   & 103                 & 2                    & 16,980,109  & 13   & 28 ~~     & 266 ~~          \\
        % RSA                & OpenSSL 1.0.2f           & 38                    & 27                  & 11                   & 14,468,307  & 10   & 28 ~~     & 160  ~~         \\
        % RSA                & OpenSSL 1.0.2k           & 36                    & 27                  & 9                    & 15,285,210  & 12   & 39 ~~     & 282   ~~        \\
        % RSA                & OpenSSL 1.1.0f           & 31                    & 22                  & 9                    & 16,390,750  & 13   & 32 ~~     & 262 ~~          \\
        % RSA                & OpenSSL 1.1.1            & 26                    & 20                  & 6                    & 18,207,020  & 12   & 7 ~~      & 455 ~~          \\\hline
        % Total              &                          & 883                   & 205                 & 678                  & 128,042,089 &      & 213m \foo & 1,688m \foo     \\\hline
    \end{tabular}
    }
\end{table*}

We evaluate \tool{} on real-world crypto libraries including OpenSSL, mbed TLS and Monocypher\@. 
OpenSSL is the most commonly used crypto libraries in today's software. mbed TLS\@
(previous known as PolarSSL) is designed to be easy to understand and fit on
embedded devices. We also evaluate Monocypher, a new cryptographic library that
resists to most side-channel attacks. 
Monocypher is designed to have no 
secrets dependent indices and no secret dependent branches. Therefore,
Monocypher should be secure under our threat model.

We build the source code into 32-bit x86 Linux executables with GCC 8.0
under Ubuntu 14.04. Although we use use symbol information to track back leakage
sites into the source code, our tool can work on stripped binaries as well. We
develop a Pin tool based on Intel Pin (version 3.7) to record the execution
trace. We run our experiments on a 2.90GHz Intel Xeon(R) E5-2690 CPU with 128GB
RAM memory. During our evaluation process, we are interested in the following
aspects:
\begin{enumerate}
    \item  \textbf{Identifying side-channels leakages.}
          The first step of \tool{} is to identify side-channel leakages. Is
          \tool{} effective to detect side-channels in real-world crypto
          systems? (\S\ref{sec:eval_overview} and \S\ref{eval:scala})
    \item  \textbf{Quantifying side-channel leakages.}
          Can \tool{} precisely report the number of leaked bits in crypto
          libraries? Are the numbers of leaked bits reported by \tool{} useful
          to justify the severity levels of each side-channel vulnerability?
          (\S\ref{eval:sym}, \S\ref{eval:asym}, \S\ref{eval:mono})
\end{enumerate}

\subsection{Evaluation Result Overview} \label{sec:eval_overview}
Table~\ref{table:over_result} shows the overview of evaluation results. \tool{} finds
883 leakages in total from real-world crypto system libraries. Among those 883
leak points, 205 of them are leaked due to secret-dependent control-flow
transfers and 678 of them are leaked due to secret-dependent memory accesses.

\tool{} also identifies that most side-channel
vulnerabilities leak very little information in practice, which confirms our
initial assumptions.  Without our tool, developers will not be able to
distinguish those ``vulnerabilities'' from severe ones and ignore others for sure easily.
However, we do find some vulnerabilities that \tool{} reports with more severe
leakages. Some of them have been confirmed by existing research that those
vulnerabilities can be exploited to realize real attacks.

All tested symmetric encryption implementations in OpenSSL and mbed TLS\@ have significant
leakages due to the lookup table to speed up the
computation. Every leakages found during the evaluation belong to the type of
secret-dependent memory accesses. \tool{} finds several leakage sites for both 
implementations of DES and AES in
OpenSSL and mbed TLS\@. \tool{} confirms that all those leakages come from table
lookups. mbed TLS 2.15 and 2.5 have the same implementation of DES and AES so
they have the same leakage report. One proper fix would be a scalar bit-sliced
implementation. However, we do not see the bit-sliced implementation of AES and
DES in various versions of OpenSSL and mbed TLS\@. However, we find the new
implementation of OpenSSL instead uses typical four 1K tables. It only uses one 1K
of the tables. This implementation is rather easy but does somehow decrease the total
amount of leaked information as the quantification result shown in the next
section.

We also evaluates our tool on the RSA implementation. With the optimization
introduced in \S\ref{sec:scala}, we do not apply any domain knowledge to
simplify the analysis. Therefore, our tools can not only identify all the leakage sites
reported by CacheD~\cite{203878}, but find new leakages in a shorter time. 
We also find the newer
versions of RSA in OpenSSL tend to have fewer leakages detected by \tool{}. We
will discuss the version changes and corresponding leakages in \S\ref{eval:asym}.

\tool{} can also estimate how
much information is leaked from each vulnerability. \tool{} achieves the goal by
estimating number of keys that satisfy the constraints. During the evaluation,
for each leakage site, \tool{} will stop once 1) it has 95\% confidence
possibility that the error of estimated leaked information is less than 1 bit,
which gives us confidence on the leakage quantification with the \emph{precision guarantee}, 
or 2) it cannot reach the termination condition after 10 minutes. In
the latter case, it means the number of satisfying keys is very small and the
leakage is quite severe. \emph{That is, timeout indicates severe leakage.}
\label{loc:timeout}
We manually check those leakage sites and find most of them are quite severe.
We will present the details in the subsequent sections.

\subsection{Comparison with the Existing Tools}
%\subsection{Scalability}
\label{eval:scala}

%% Fix tonight
%\subsubsection{Running Time}
%\begin{figure}
%    \centering
%    \includegraphics[width=.8\columnwidth]{./figures/result/running_time.pdf}
%    \caption{With the optimization introduced, \tool{} can be scalable to RSA}
%\end{figure}

%\subsubsection{Comparison with the Existing Tools}
\tool{} is designed to quantify side-channel leakages. But it can detect
side-channels leakages as well. In this section, we compare \tool{} with the
existing trace-based side-channel detection tools.

As shown in Table~\ref{eval:cacheD},
\tool{} not only discovers all the leakage sites reported by CacheD~\cite{203878}, but also
find many new ones. CacheD fails to detect many vulnerabilities for two
reasons. First, CacheD can only detect secret-dependent memory access
vulnerabilities. But \tool{} can detect secret-dependent control-flows as well.
Second, CacheD uses some domain
knowledge to simplify symbolic execution and has to trim the traces before
processing, which does not introduce false positives, but can neglect some
vulnerabilities. The table~\ref{eval:cacheD} shows that
\tool{} is three times faster than CacheD. As the time of symbolic execution
grow quadratically, \tool{} is much faster than CacheD when analyzing the same
number of instructions. For example, when we test~\tool{} on AES from OpenSSL
0.9.7, ~\tool{} is more than 100x faster than CacheD.

Since DATA~\cite{217537} compares several execution traces to identify
side-channel leakages, \tool{} also outperforms DATA in terms of performance.
For example, it takes 234 minutes for DATA to analysis the RSA of implementation
in OpenSSL 1.1.0f. \tool{} only spends 34 minutes according to Table~\ref{table:over_result}.
Also, DATA reports report 278 control-flow and 460 data leaks. Among those leakages,
they found two vulnerabilities. On the contrary, \tool{} can report how many bits is
actually leaked for each vulnerability, which eases the pain to identify real sensitive leaks.

\begin{table}[]
    \caption{Comparison with CacheD}
    \label{eval:cacheD}
    \resizebox{\columnwidth}{!}{%
        \begin{tabular}{c|c|c|c|ccc}
            \hline
            \multicolumn{1}{l|}{} & \multicolumn{2}{c|}{Number of Instructions} & \multicolumn{2}{c|}{Time (s)} & \multicolumn{2}{c}{Number of Leakages}                                                                                         \\ \cline{2-7}
            \multicolumn{1}{l|}{} & \multicolumn{1}{c|}{CacheD}                 & \multicolumn{1}{c|}{\tool}    & \multicolumn{1}{c|}{CacheD}            & \multicolumn{1}{c|}{\tool}  & \multicolumn{1}{c|}{CacheD} & \multicolumn{1}{c}{\tool} \\ \hline
            AES 0.9.7             & 791                                         & 1,704                         & 43.4                                   & \multicolumn{1}{c|}{0.30}   & \multicolumn{1}{c|}{48}     & 75                        \\
            AES 1.0.2f            & 2,410                                       & 1,350                         & 48.5                                   & \multicolumn{1}{c|}{0.08}   & \multicolumn{1}{c|}{32}     & 88                        \\
            RSA 0.9.7             & 674,797                                     & 16,980,109                    & 199.3                                  & \multicolumn{1}{c|}{1681}   & \multicolumn{1}{c|}{2}      & 105                       \\
            RSA 1.0.2f            & 473,392                                     & 14,468,307                    & 165.6                                  & \multicolumn{1}{c|}{1692}   & \multicolumn{1}{c|}{2}      & 38                        \\ \hline
            Total                 & 1,151,390                                   & 31,451,470                    & 456.8                                  & \multicolumn{1}{c|}{3373.4} & \multicolumn{1}{c|}{84}     & 317                       \\ \hline
            \multicolumn{7}{l}{\# of Instructions per second \qquad  CacheD: 2,519 \qquad \tool: 9,324}                                                                                                                                          \\ \hline
        \end{tabular}
    }
\end{table}

\subsection{Case Studies}

\subsubsection{Symmetric Ciphers: DES and AES}\label{eval:sym}
We test both DES and AES ciphers from mbed TLS and OpenSSL. Both cipher
implementations apply lookup tables, which can
speed up the calculation, but can also introduce additional side-channels
as well. During our evaluation, we find mbed TLS 2.5 and 2.15.1 have the same
implementation of AES\@ and DES\@. Therefore, our tool also provides the same 
leakage report for both
versions. A simple of the generated report can be found in the Appendix~\ref{sec:result-table}.

According to \tool{}, we find DES\@ implementations in both mbed TLS and OpenSSL have several
sensitive information leakages in the key schedule function.
However, leakages in OpenSSL are more severe. We do not see any mitigation
in the new version. We think it is not seen as worth the engineering
efforts given the life cycles of DES\@.

\tool{} shows that the AES\@ in OpenSSL 1.1.1 
has less amount of leakages compared to other versions by our tools. 
(e.g., the max leakage of AES in OpenSSL 1.1.1 is 4.4 bits, but other version
have leakages that can leak can around 10 bits.)
We find that OpenSSL 1.1.1 version 
instead uses the 1KB lookup tables with 32bit entries like older versions, it uses an 
tables with 8 bit entries. In other words, our tools shows that lookup tables with smaller 
entries will leak less amount of information. Our tool also suggests a smaller lookup
table can mitigate side-channels vulnerabilities. For example, Shown in Appendix~\ref{mbedtls_aes},
\tool{} identify seven leakages from function \textsf{mbedtls\_internal\_aes\_decrypt}.
However, \tool{} reports leakage 1, 2, 3 leak more information
compared to leakage 4, 5, 6, 7. 
We check the source code and find leakage 1, 2,
3 use secret to access the lookup table \textsf{RT0, RT1, RT2, RT3}, which is 8K
each. On the contrary, leakage 4, 5, 6, 7 each access a smaller lookup table
(2K).


\subsubsection{Asymmetric Ciphers: RSA}\label{eval:asym}
We also evaluate \tool{} on RSA.  We find developers are more interested in 
fixing side-channel vulnerabilities for RSA implementations. 
We test five versions of OpenSSL (0.9.7, 1.0.2f, 1.0.2k, 1.1.0f, 1.1.1). The
result, as shown in Figure~\ref{fig:rsa}, indicates that the newer
version of OpenSSL leaks less amount of information compared to the previous
versions. After version 0.9.7g, OpenSSL adopts a fixed-window \textsf{mod\_exp\_mont}
implementation for RSA\@. With the new design, the sequence of squares and
multiples and the memory access patterns are independent of the secret key.
\tool{}'s result confirms the new exponentiation implementation has 
effectively mitigated most of leakages because the other four versions have fewer
leakages compared with version 0.9.7. 
OpenSSL version 1.0.2f, 1.0.2k and 1.1.0f almost have the
same amount of leakage. We check the changelog and find only one change for
patching the vulnerability CVE-2016-0702. 
\tool{} find OpenSSL 1.1.1 version has significantly less amount of 
leaked information compared to other versions.
We check the changelog of OpenSSL 1.1.1 and find it claims
the new RSA implementation adopted ``numerous side-channel attack mitigation.'', 
which proves the effectiveness of our quantifying method.

\begin{figure*}
    \centering
    \vspace*{-9pt}
    \hspace*{-8pt}
    \subfloat[RSA OpenSSL 0.9.7]{
        \includegraphics[width=.19\linewidth]{./figures/result/RSA-openssl-0-9-7.pdf}
        \label{fig:rsa-1}
    }
    \subfloat[RSA OpenSSL 1.0.2f]{
        \includegraphics[width=.19\linewidth]{./figures/result/RSA-openssl-1-0-2f.pdf}
        \label{fig:rsa-2}
    }
    \subfloat[RSA OpenSSL 1.0.2k]{
        \includegraphics[width=.19\linewidth]{./figures/result/RSA-openssl-1-0-2k.pdf}
        \label{fig:rsa-3}
    }
    \subfloat[RSA OpenSSL 1.1.0f]{
        \includegraphics[width=.19\linewidth]{./figures/result/RSA-openssl-1-1-0f.pdf}
        \label{fig:rsa-4}
    }
    \subfloat[RSA OpenSSL 1.1.1]{
        \includegraphics[width=.19\linewidth]{./figures/result/RSA-openssl-1-1-1.pdf}
        \label{fig:rsa-5}
    }
    \caption{RSA implementations in different versions of OpenSSL\@. 
        We round the number of leaked information into the nearest integer. 
        The mark $*$ means timeout,
        which indicates more severe leakages (see \S\ref{loc:timeout}).}
    \label{fig:rsa}
    \vspace*{-12pt}
\end{figure*}

Our quantification result shows vulnerabilities
that leak more information identified by \tool{} 
are more likely to be fixed in the updated version.
As presented in Figure~\ref{fig:rsa}, 
OpenSSL 0.9.7 has several severe leaks from
function \textsf{bn\_sqr\_comba8}, which is a main 
component of the OpenSSL big number implementation.
The value of the function parameter \textbf{a} is derived from
the secret key. 
Shown in Figure~\ref{fig:old_sqr2}, it has a 
secret-dependent control flow at line 8.
As function \textsf{bn\_sqr\_comba8}
call the macro (\textsf{sqr\_add\_c2}) multiple times, 
and the code will leak some information each times.
\tool{} thinks the vulnerability is quite serious. 
The vulnerability has been patched in OpenSSL 1.1.1. Seen in 
Figure~\ref{fig:new_sqr2}, control-flows transfers are replaced
with bitwise operations. So there is no leaks in the function
\textsf{sqr\_add\_c2} in OpenSSL 1.1.1. We also mention
that even line 4 and 9 in Figure~\ref{fig:old_sqr2} all have if branches,
however, it is not a secret-dependent control-flow transfers because
most compilers will use \emph{add with carry} instruction to remove the branch.
Besides, branches can also be compiled to non-branch machine instructions 
like conditional moves. Therefore, simple code reviews are not accurate
enough to detect side-channels. 

On the other hand, for those vulnerabilities that leak less information. 
Developers are more reluctant to fix them. 
For example, OpenSSL 0.9.7 adopts a fixed windows version of 
function \textsf{BN\_mod\_exp\_mont\_constime} to replace original function
\textsf{BN\_mod\_exp\_mont}.
However, \tool{} detects a minor vulnerability in the function that can
leaks the last bit of the big number \textbf{m}. In the updated version,
developers make the fixed windows become the default option and rewrite part of the 
function. However, the vulnerability still exits in OpenSSL 1.1.1,
despite the vulnerability is blunt. Details of the 
vulnerability can been seen in Appendix\S~\ref{appendix:minor:vul}.

\begin{figure}
    \centering
    \begin{lstlisting}[xleftmargin=.02\textwidth,xrightmargin=.01\textwidth]
# define mul_add_c2(a,b,c0,c1,c2) \
    t=(BN_ULLONG)a*b; \
    tt=(t+t)&BN_MASK; \
    if (tt < t) c2++; \
    t1=(BN_ULONG)Lw(tt); \
    t2=(BN_ULONG)Hw(tt); \
    c0=(c0+t1)&BN_MASK2;  \
    if ((c0 < t1) && (((++t2)&BN_MASK2) == 0)) c2++; \
    c1=(c1+t2)&BN_MASK2; if ((c1) < t2) c2++;
\end{lstlisting}
    \vspace*{-6pt}
    \caption{Macro \textsf{sqr\_add\_c2} in OpenSSL 0.9.7}
    \label{fig:old_sqr2}
    \vspace*{-6pt}
\end{figure}



\begin{figure}
    \centering
    \begin{lstlisting}[xleftmargin=.02\textwidth,xrightmargin=.01\textwidth]
# define mul_add_c2(a,b,c0,c1,c2)      do { \
    BN_ULONG ta = (a), tb = (b);            \
    BN_ULONG lo, hi, tt;                    \
    BN_UMULT_LOHI(lo,hi,ta,tb);             \
    c0 += lo; tt = hi+((c0<lo)?1:0);        \
    c1 += tt; c2 += (c1<tt)?1:0;            \
    c0 += lo; hi += (c0<lo)?1:0;            \
    c1 += hi; c2 += (c1<hi)?1:0;            \
    } while(0)
\end{lstlisting}
    \vspace*{-6pt}
    \caption{Macro \textsf{sqr\_add\_c2} in OpenSSL 1.1.1}
    \label{fig:new_sqr2}
    \vspace*{-6pt}
\end{figure}

Even for the up-to-date version of OpenSSL, \tool{} still find several
side-channel leakages. Several of them are quite serious.
Details can be found in Appendix \S\ref{appendix:minor:vul} and Table \ref{tab:RSAOpenSSL1.1.1}.

\subsubsection{Monocypher}\label{eval:mono}
Monocypher is a small, easy to use cryptographic library with a
comparable performance as LibSodium~\cite{libsodium} and NaCl~\cite{bernstein2012security}. 
We choose four ciphers from the library that are 
designed to be more side-channel resistant.
Because those ciphers have no 
data flow from secrets to branch conditions and load addresses.
Therefore, Monocypher should be safe under our threat models.
However, we are still interested in those vulnerabilities caused
by compiler optimizations and implementations.
We analyze those ciphers with \tool{}, and it reports no leaks from the
implementation. 



\section{Discussions and Limitations}
While recent works have reported lots
of potential side-channel vulnerabilities, we find many of them are not patched by
developers.
However, side-channels are inevitable in software and it is hard to fix all of them. 
Addressing old vulnerabilities may also introduce new leakage
sites. We need a
tool that can automatically estimate the sensitivity level of each vulnerability.
So software engineers can focus on
``severe'' leakages. For example, our tool will report that 
the modular exponentiation using square and multiply algorithms can
leak more information than a key validation function.

\tool{} can be used by software developers to find sensitive vulnerabilities
and reason about countermeasures.
\tool{} estimates the amount of leaked information for each side-channel leakage
in one execution trace. \tool{} is useful for software
engineers to test programs and fix vulnerabilities.
The design, which is very
precise in terms of true leakages compared to other static source code
method~\cite{197207,BacelarAlmeida:2013:FVS:2483313.2483334}, can omit some
leakages on other traces. The amount of leaked information also depends on the secret key.
However, as the tool is intended for debugging and testing,
we think it is software engineers' responsibility to select the input key and trigger 
the path in which they are interested. It is not a problem for crypto software 
since virtually all keys follow the same or similar computational paths.

We use the amount of leaked information to represent the sensitivity level of 
each side-channel vulnerability. Although imperfect, \tool{} produces a reasonable 
measurement for each leak. For example, the simple modular exponentiation is 
notoriously famous for various side-channel attacks~\cite{kocher1996timing}. 
During the execution, the single leak points will be executed multiple times, 
and each time it leaks one different bit. Therefore, \tool{} reports that the 
vulnerability can leak the whole key. However, not every leak point inside a 
loop is severe. Because very often the leakages may leak the same bit in the 
original key, and those leaks are not independent. \tool{} can capture the most 
fine-grained information by modeling each leak during the execution as a math 
formula and by using the conjunction of those formulas to describe the total effect. 
There could be multiple leakage sites in the program. While some leakages are independent, 
some other leakage sites are dependent. Some leakage sites (e.g., square and multiply) 
can leak one particular bit of the original key, but some leakage sites leak one bit 
from several bytes in the original key. Those constraints can precisely model the 
relationship between the key buffers and leakage sites.

\tool{} reaches full precision if the number of estimated leaked bits 
equals to Definition~\ref{def}. According to the threat model, the 
length of the secret is known, so the only estimated value is the number
of possible secrets under the attacker's observation ($|K^o|$). 
During the symbolic execution, \tool{} may lose precision from the 
memory model it uses in theory. However, we do not find false positives 
caused by the imprecise memory model during the evaluation. 
During the symbolic execution, for each constraint it generates,
the tool will check the correctness of the constraint by giving 
the constraint real values (the value of sensitive keys). Sampling 
can also introduce imprecision but with a probabilistic guarantee. 
However, during the evaluation, we find that \tool{} cannot estimate 
the amount of leakage for some leakage sites in a reasonable time, 
which means the number of $K^o$ is very small. According to Definition~\ref{def}, 
it means the leakage is very severe.


\section{Related Work}

There is a vast amount of work on 
side channel 
detection~\cite{182946, 236338, Brotzman19Casym, 203878,217537,Wichelmann:2018:MFF:3274694.3274741,langley2010ctgrind}, 
mitigation~\cite{Page2005PartitionedCA,
Wang:2007:NCD:1250662.1250723,Zhang:2015:HDL:2775054.2694372,Li:2014:SLH:2541940.2541947,
236344,shih2017t,Coppens:2009:PMT:1607723.1608124,
brickell2006software,crane2015thwarting}, 
information
quantification~\cite{biondi2018scalable,10.1007/978-3-642-31424-7_40,McCamantE2008,5207642,Phan:2012:SQI:2382756.2382791,Chattopadhyay:2017:QIL:3127041.3127044,zhang2010sidebuster,zhou2018static,pasareanu2016multi}, 
and model counting~\cite{wei2005new, gomes2007sampling, gomes2006model, kroc2008leveraging, Chattopadhyay:2017:QIL:3127041.3127044}.
Here we only present only work closely related to ours.
Due to space limit, we do not discuss related work on side-channel attacks.

\subsection{Detection and Mitigation}

CacheAudit~\cite{182946} uses abstract domains to compute an
over approximation of cache-based side-channel information leakage upper bound.
However, it is difficult to judge the sensitive level of the side-channel leakage based on
the leakage provided by CacheAudit. CacheS~\cite{236338} improves on
CacheAudit with new abstract domains that only track
secret-related code. Like CacheAudit, CacheS cannot
indicate the sensitive level of side-channel vulnerabilities.
CaSym~\cite{Brotzman19Casym} introduces a static cache-aware symbolic
reasoning technique to cover multiple paths in a target program. Again, their
approaches cannot evaluate the sensitive level for each side-channel
vulnerability, and it only work on small code snippets.

The dynamic approach, usually consists of taint analysis and symbolic execution,
can perform a very precise analysis. CacheD~\cite{203878} takes a concrete execution trace and runs symbolic execution on the trace
to get the formula of each memory address. Therefore, CacheD is
quite precise in avoiding false positives. However, CacheD is not able to detect secret-dependent control-flows. We adopted a similar approach to model the secret-dependent data accesses, but \tool{} also finds secret-dependent control-flows and give a precise quantification of the leakage.
DATA~\cite{217537} detects address-based side-channel vulnerabilities by comparing different execution traces under various test inputs. After collecting execution traces, DATA aligns them and finds the differences. It uses statistical hypothesis testing to find true leakages. However, both imperfect trace alignment and statistical testing result that DATA can produce false positives.
MicroWalk~\cite{Wichelmann:2018:MFF:3274694.3274741} uses
mutual information (MI) between sensitive input and execution state to detect side-channels. 

Both hardware~\cite{Page2005PartitionedCA,
Wang:2007:NCD:1250662.1250723,Zhang:2015:HDL:2775054.2694372,Li:2014:SLH:2541940.2541947,
236344, 236334} and software~\cite{shih2017t,Coppens:2009:PMT:1607723.1608124,
brickell2006software,crane2015thwarting, 197207} side-channels mitigation techniques have
been proposed recently. Hardware countermeasures, including partitioning hardware resources~\cite{Page2005PartitionedCA}, randomizing cache
accesses~\cite{Wang:2007:NCD:1250662.1250723, 236344}, and designing new
architecture~\cite{tiwari2011crafting}, require changes to complex processors and are complex to adopt. On the contrary, software approaches are
usually easy to implement. Coppens et
al.~\cite{Coppens:2009:PMT:1607723.1608124} uses a compiler
to eliminate key-dependent control-flow transfers. Crane et
al.~\cite{crane2015thwarting} mitigated side-channels by randomizing software.
As for crypto libraries, the basic idea is to eliminate key-dependent
control-flow transfers and data accesses. Common approaches include
bit-slicing~\cite{konighofer2008fast,rebeiro2006bitslice} and unifying
control-flows~\cite{Coppens:2009:PMT:1607723.1608124}.

\subsection{Quantification}

Proposed by Denning~\cite{robling1982cryptography} and Gray~\cite{gray1992toward}, 
Quantitative Information Flow (QIF) aims at providing an estimation of the amount of leaked
information from the sensitive information given the public output. If zero bits
of the information are leaked, the program is called non-interference. McCamant
and Ernst~\cite{McCamantE2008} quantify the information leakage as the network
flow capacity. Backes et al.~\cite{5207642} propose an automated method for QIF
by computing an equivalence relation on the set of input keys. But the approach
cannot handle real-world programs with bitwise operations. 
Phan et al.~\cite{Phan:2012:SQI:2382756.2382791} propose symbolic QIF. The goal of their
work is to ensure a program is non-interference. They adopt an over
approximation method to estimate the total information leakage and their method
does not work for secret-dependent memory access side-channels.
\cready{}{Pasareanu et al.~\cite{pasareanu2016multi} combine symbolic analysis and Max-SMT solving to synthesize the concrete public input that can lead to the worst case leakage. They assume the target program has multiple different input secrets and calculate the average leakage for one-fixed public input.}
CHALICE~\cite{Chattopadhyay:2017:QIL:3127041.3127044} quantifies the leaked
information for a given cache behavior. 
It symbolically reasons about cache
behavior and estimates the amount of leaked information based on cache miss/hit.
Their approach only scale to small programs, which limits its usage in
real-world applications. On the contrary, \tool{} assesses the sensitive level
of side-channels with different granularities. It can also analyze side-channels
in real-world crypto libraries.

\subsection{Model Counting}
Model counting refers to the problem of computing the number of 
models for a propositional formula (\#SAT). There are two approaches to
solving the problem, exact model counting and approximate model 
counting. We focus on approximate model counting since it is our approach. Wei and Selman~\cite{wei2005new} introduce
ApproxCount, a local search based method using Markov Chain Monte 
Carlo (MCMC). ApproxCount has the better scalability than 
exact model counters. Other approximate model counter includes 
SampleCount~\cite{gomes2007sampling},
Mbound~\cite{gomes2006model}, and MiniCount~\cite{kroc2008leveraging}. 
Unlike ApproxCount,
these model counters can give lower or upper bounds with guarantees.
Despite the rapid development of model counters for SAT and some 
research~\cite{chistikov2017approximate,phan2015model} on Modulo Theories model counting (\#SMT),
they cannot be directly applied to 
side channel leakage quantification.
ApproxFlow~\cite{biondi2018scalable} uses ApproxMC~\cite{chakraborty2016algorithmic} for information flow quantification,
but it has only been tested with small programs.

\section{Conclusion}
This paper presents a novel  %definition and
method to
quantify memory-based side-channel leakage. We implement the method in
a prototype called \tool{} and show its effectiveness in finding
and quantifying side-channel leakage. With the new definition of
information leakage that models actual side-channel attackers, quantifying the number of
leaked bits helps understand the severity level
of side-channel vulnerabilities. The evaluation confirms that \tool{} is useful in estimating the amount of leaked information in
real-world applications.


\section{Data Availability}
\cready{}{\tool{} is publicly available at \url{https://github.com/s3team/Abacus}. The repository also contains benchmarks, metadata, and raw results of our experiments.}

\section{Acknowledgement}
We thank Ziqi Wang, Pei Wang, Zhaofeng Chen, and the anonymous reviewers for their valuable feedback. The work was supported in part by the National Science Foundation (NSF) under grant CNS-1652790, and the Office of Naval Research (ONR) under grants N00014-16-1-2912, N00014-16-1-2265, and N00014-17-1-2894. 

%Any opinions, conclusions in the paper do not reflect the views of funding agencies.

\bibliographystyle{IEEEtran}
\bibliography{refs}

%\newpage
%\clearpage
%\appendices
%\setcounter{section}{0}
%\section*{Supplementary Materials to Submission \#63}
%\setcounter{section}{1}
%\setcounter{page}{1}
%
\newcommand{\vvv}{\vspace*{-3pt}}

\section{\tool{}' main components}

\label{appendix:code}
\begin{table}[h!]
    \centering
    %    \resizebox{.8\columnwidth}{!}{
    \caption{\tool{}' main components and sizes}\label{tbl:implementation}
    \begin{tabular}{lr@{~}@{}l}
        \hline
        Component            & \multicolumn{2}{c}{Lines of Code (LOC)}             \\ \hline
        Trace Logging        & 501 lines                               & of C++    \\
        Symbolic Execution   & 14,963 lines                            & of C++    \\
        Data Flow            & 451 lines                               & of C++    \\
        Monte Carlo Sampling & 603 lines                               & of C++    \\
        Others               & 211 lines                               & of Python \\ \hline
        Total                & 16,729 lines                            &           \\\hline
    \end{tabular}
    %    }
\end{table}



\section{Algorithm to Compute the Maximum Independent Partition}
\label{appendix:partition}
~
{\small
\IncMargin{1em}
\begin{algorithm}[h]\small
    \DontPrintSemicolon
    \SetKwInOut{Input}{input}\SetKwInOut{Output}{output}
    \Input{$c_t(\addr{1},\addr{2},\ldots,\addr{n}) = c_{\addr{1}} \land c_{\addr{2}} \land \ldots \land c_{\addr{m}}$}
    \Output{The Maximum Independent Partition of $G = \{g_{1}, g_{2}  , \ldots,  g_{m} \}$ }
    \For{$i\leftarrow 1$ \KwTo $n$}
    {
        $S_{c_{\addr{i}}}$ $\leftarrow$ $\pi(c_{\addr{i}})$ \;
        \For{$g_{i} \in G$}
        {
            $S_{g_j}$ $\leftarrow$ $\pi(g_{j})$ \;
            $S$ $\leftarrow$ $S_{c_{\addr{i}}} \cap S_{g_j}$  \;
            \If{$S \neq \emptyset$}
            {
                $g_{j} \leftarrow g_{i} \land g_{\addr{i}}$ \;
                \textbf{break} \;
            }
            Insert $c_{\addr{i}}$ to $G$
        }
    }
    \caption{The Maximum Independent Partition}
    \label{algo:max-inde}
\end{algorithm}
\DecMargin{1em}
}
~
\newpage
\section{Algorithm to Compute the Number of Satisfying Assignments}
\label{appendix:montecarlo}
~
{\small
\IncMargin{1em}
\begin{algorithm}\small
    \SetAlgoLined
    \DontPrintSemicolon

    \KwIn{{The constraint $g_{i}= c_{i_1} \land c_{i_2}
                    \land \ldots \land c_{i_m}$}}
    \KwOut{{The number of assignments that satisfy $g_{i}$ $|K_{g_{i}}|$}}

    $n$: the number of sampling times \;
    $S_{c_i}$: the set contains input variables for $c_{i}$ \;
    $n_{s}$: the number of satisfying assignments \;
    $N_{c_t}$: the set contains all solution for $c_t$ \;
    $r$: times of reducing $g$\;
    $k$: the input variable \;
    $R$: a function that produces a random point from $S_{c_i}$\;
    %$\#k$: the satisfying number of k \fixme{this number is not used syntactically} \;
    %Initialization: \;
    $r$ $\leftarrow$ $1$,
    $n$ $\leftarrow$ $0$ \;
    \For{$t$ $\leftarrow$ $1$ \KwTo $m$} {
        $S_{c_t}$ $\leftarrow$ $\pi(c_t)$ \;
        \If{$|S_{c_t}| = 1$}
        {
            $N_{c_t}$ $\leftarrow$ Compute all solutions of $c_i$ \;
            $N_{c_t} = \{n_1, \ldots, n_m\},\ S_{c_t} = \{k\}  $ \;
            $g_{i} = $ $g_i(k=n_1) \land \ldots \land g_i(k=n_m)$ \;
            $r \leftarrow r+1$ \;

        }
    }
    \While{$n \leq \frac{6p}{1-p}$} {
        $S_{g_i}$ $\leftarrow$ $\pi(g_i)$ \;
        $v \leftarrow R(S_{g_i})$ 
        \If{$v$ satisfies $g_i$}
        {
           $n_s \leftarrow n_s + 1$
        }
        $n \leftarrow n +1,\ p = \frac{n_s}{n}$
    }

    $|K_{g_{i}}|$ $\leftarrow$ $n_s|K| / (n * r * range(k))$
    \caption{Multiple Step Monte Carlo Sampling}
\end{algorithm}
\DecMargin{1em}
}
~\vvv

\section{AES Lookup Tables Leakage}
\begin{figure}[h!]
    \centering
    \begin{lstlisting}[xleftmargin=.01\textwidth,xrightmargin=.01\textwidth]
int mbedtls_internal_aes_encrypt( mbedtls_aes_context *ctx,
const unsigned char input[16],
unsigned char output[16] )
{
uint32_t *RK, X0, X1, X2, X3, Y0, Y1, Y2, Y3;
...
for( i = ( ctx->nr >> 1 ) - 1; i > 0; i-- )
{
  AES_FROUND( Y0, Y1, Y2, Y3, X0, X1, X2, X3 ); // Leakage 1
  AES_FROUND( X0, X1, X2, X3, Y0, Y1, Y2, Y3 ); // Leakage 2
}
AES_FROUND( Y0, Y1, Y2, Y3, X0, X1, X2, X3 );   // Leakage 3
X0 = *RK++ ^ \                                  // Leakage 4
    ( (uint32_t) FSb[ ( Y0       ) & 0xFF ]       ) ^
    ( (uint32_t) FSb[ ( Y1 >>  8 ) & 0xFF ] <<  8 ) ^
    ( (uint32_t) FSb[ ( Y2 >> 16 ) & 0xFF ] << 16 ) ^
    ( (uint32_t) FSb[ ( Y3 >> 24 ) & 0xFF ] << 24 );
// X1, X2, X3 do the same computation as X0
...                                          // Leakage 5,6,7
PUT_UINT32_LE( X0, output,  0 );
...
return( 0 );
}
\end{lstlisting}
    \vspace*{-6pt}
    \caption{Function \textsf{mbedtls\_internal\_aes\_encrypt}}
    \label{mbedtls_aes}
    \vspace*{-9pt}
\end{figure}

\clearpage
\section{Minor side-channels vulnerability}
\label{appendix:minor:vul}
Here we present a side-channel vulnerability that leaks
less than one bit information by~\tool{}. The vulnerability exists
from OpenSSL 0.9.7 to OpenSSL 1.1.1. Shown in Figure~\ref{fig:old_sqr2} 
and Figure~\ref{fig:new_sqr2}, \textbf{m} is a big number that
derives from the private key.
At line 6, it can leak the last bit of \textbf{m} by observing the branch. 
As the leak is tiny,
we think developers do not have enough motivations
to fix the vulnerability.

\begin{figure}[h]
\centering
\begin{lstlisting}[xleftmargin=.02\textwidth,xrightmargin=.01\textwidth]
int BN_mod_exp_mont_consttime(BIGNUM *rr, 
const BIGNUM *a, const BIGNUM *p,
const BIGNUM *m, BN_CTX *ctx, BN_MONT_CTX *in_mont)
{
... 
if (!(m->d[0] & 1)) {
    ... 
    return 0;
}
bits = BN_num_bits(p);
if (bits == 0) 
...
}
\end{lstlisting}
    \vspace*{-6pt}
    \caption{\textsf{BN\_mod\_exp\_mont\_consttime} in OpenSSL 0.9.7}
    \label{fig:old_sqr2}
    \vspace*{-6pt}
\end{figure}



\begin{figure}[h]
    \centering
    \begin{lstlisting}[xleftmargin=.02\textwidth,xrightmargin=.01\textwidth]
int BN_mod_exp_mont(BIGNUM *rr, 
const BIGNUM *a, const BIGNUM *p,
const BIGNUM *m, BN_CTX *ctx, BN_MONT_CTX *in_mont)
{
...
if (!BN_is_odd(m)) {
    ...
    return 0;
}
bits = BN_num_bits(p);
if (bits == 0) 
...
}

\end{lstlisting}
    \vspace*{-6pt}
    \caption{\textsf{BN\_mod\_exp\_mont} in OpenSSL 1.1.1}
    \label{fig:new_sqr2}
    \vspace*{-6pt}
\end{figure}

\section{Unknown Leaks in OpenSSL 1.1.1}
Shown in Table~\ref{tab:RSAOpenSSL1.1.1}, \tool{} discovers a series
of side-channel vulnerabilities in the up-to-date version of OpenSSL library. 
However, many of them are negligible quantified by \tool{}.
Here we present a unknown vulnerability. Despite the following
vulnerability leaks through the same code patterns as the vulnerability shown
in \S\ref{appendix:minor:vul}, \tool{} shows the following code can leak
around 15 bits from the original key.
\label{appendix:unknown}
\begin{figure}[h]
\centering
\begin{lstlisting}[xleftmargin=.02\textwidth,xrightmargin=.01\textwidth]
while (!BN_is_bit_set(B, shift)) { /* note that 0 < B */
    shift++;
    if (BN_is_odd(X)) {
        if (!BN_uadd(X, X, n))
            goto err;
    }
    ...
    if (!BN_rshift1(X, X))
        goto err;
}

\end{lstlisting}
    \vspace*{-6pt}
    \caption{Unknown sensitive secret-dependent branch leaks from function 
             \textsf{int\_bn\_mod\_inverse} in OpenSSL 1.1.1. Same as the example
             in \S\ref{appendix:minor:vul}, the code can leak the last digit from
             big number \textbf{X}. However, the leak is more sensitive because of the 
             function \textsf{BN\_rshift1}. Each time function \textsf{BN\_rshift1}
             will shift \textbf{X} right by one and places the result in \textbf{X}. Therefore,
             an attacker can infer multiple bits of \textbf{X} by observing the branch at line 3.}
    \label{fig:unknown}
    \vspace*{-6pt}
\end{figure}

\section{Detailed Experimental Results}
\label{sec:result-table}

Here we present the detailed experimental results.
Due to space limitation, we select the representative implementations of
AES, DES, and RSA in
mbed TLS 2.5,
OpenSSL 1.1.0f,  and
OpenSSL 1.1.1.  
%For RSA, we also include OpenSSL 1.0.2f and OpenSSL 1.0.2k.
The results are representative to other versions.
All the results will be made available in electronic format online
when the paper is published. %at \fixme{http://tinyurl}.

In all the tables presented in this appendix, the mark ``$*$'' means timeout,
which indicates more severe leakages. See \S\ref{loc:timeout} for the details.
Also note that we round the calculated numbers of leaked bits to include one digit
after the decimal point, so $0.0$ really means very small amount of leakage, but not exactly zero. See \S\ref{sssec:errest} for the details of error estimate.

  %as an integer.
  %Hence the `0' leakage does not equals to no leakage but some leakage size
 % between 0 and 0.5 bit, see \S\ref{sssec:errest} for the details.


\input{figures/form/latex/DES-mbedTLS-2-5.tex}
%% \input{figures/form/latex/DES-mbedTLS-2-15-1.tex}
%% \input{figures/form/latex/DES-openssl-0-9-7.tex}
%% \input{figures/form/latex/DES-openssl-1-0-2f.tex}
%% \input{figures/form/latex/DES-openssl-1-0-2k.tex}
\input{figures/form/latex/DES-openssl-1-1-0f.tex}
\input{figures/form/latex/DES-openssl-1-1-1.tex}


\input{figures/form/latex/RSA-mbedTLS-2-5.tex}
%% \input{figures/form/latex/RSA-mbedTLS-2-15-1.tex}
%%\input{figures/form/latex/RSA-openssl-0-9-7.tex}
\input{figures/form/latex/RSA-openssl-1-0-2f.tex}
\input{figures/form/latex/RSA-openssl-1-0-2k.tex}
\input{figures/form/latex/RSA-openssl-1-1-0f.tex}
\begin{table}[h!]
\centering\tiny\scriptsize
\caption{Leakages in RSA implemented by OpenSSL 1.1.1}\label{tab:RSAOpenSSL1.1.1}
%\resizebox{\columnwidth}{!}{
\begin{tabular}{l@{~~}rlr@{~~}r}
\hline
\textbf{File} & \textbf{Line No.} & \textbf{Function} & \hspace*{-20em}\textbf{\# Leaked Bits} & \textbf{Type} \\\hline
%\input{figures/form/latex/RSA-openssl-1-1-1-data.tex}
rsa\_ossl.c& 399&rsa\_ossl\_private\_decrypt&0.0 &CF\\
bn\_lib.c& 555&BN\_ucmp&*&CF\\
bn\_gcd.c& 199&int\_bn\_mod\_inverse&1.0 &CF\\
bn\_gcd.c& 247&int\_bn\_mod\_inverse&14.9 &CF\\
bn\_gcd.c& 225&int\_bn\_mod\_inverse&12.3 &CF\\
ct\_b64.c& 168&\_\_udivdi3&0.1 &CF\\
bn\_div.c& 374&bn\_div\_fixed\_top&*&CF\\
bn\_lib.c& 955&bn\_correct\_top&2.6 &CF\\
ct\_b64.c& 168&\_\_memset\_sse2\_rep&0.0 &CF\\
ct\_b64.c& 168&\_\_memset\_sse2\_rep&0.0 &CF\\
ct\_b64.c& 168&\_\_memset\_sse2\_rep&0.0 &DA\\
ct\_b64.c& 168&\_\_memset\_sse2\_rep&0.0 &DA\\
bn\_exp.c& 317&BN\_mod\_exp\_mont&1.0 &CF\\
bn\_asm.c& 592&bn\_mul\_comba8&2.1 &CF\\
bn\_exp.c& 383&BN\_mod\_exp\_mont&0.9 &CF\\
bn\_lib.c& 453&BN\_bn2binpad&0.0 &DA\\
bn\_lib.c& 450&BN\_bn2binpad&0.0 &CF\\
rsa\_oaep.c& 180&RSA\_padding\_check\_PKCS1\_OAEP\_mgf1&0.0 &DA\\
rsa\_oaep.c& 180&RSA\_padding\_check\_PKCS1\_OAEP\_mgf1&0.0 &DA\\
rsa\_oaep.c& 176&RSA\_padding\_check\_PKCS1\_OAEP\_mgf1&0.0 &CF\\
string3.h& 90&SHA1\_Final&0.0 &CF\\
rsa\_oaep.c& 200&RSA\_padding\_check\_PKCS1\_OAEP\_mgf1&0.0 &CF\\
rsa\_oaep.c& 209&RSA\_padding\_check\_PKCS1\_OAEP\_mgf1&0.0 &CF\\
rsa\_oaep.c& 250&RSA\_padding\_check\_PKCS1\_OAEP\_mgf1&0.0 &CF\\
rsa\_oaep.c& 253&RSA\_padding\_check\_PKCS1\_OAEP\_mgf1&0.0 &CF\\
ct\_b64.c& 168&\_\_memset\_sse2\_rep&0.0 &DA\\
ct\_b64.c& 168&\_\_memset\_sse2\_rep&0.0 &DA\\
\hline
\end{tabular}
%}
\renewcommand{\baselinestretch}{1.0}\selectfont
\end{table}



\input{figures/form/latex/AES-mbedTLS-2-5.tex}
%% \input{figures/form/latex/AES-mbedTLS-2-15-1.tex}
%% \input{figures/form/latex/AES-openssl-0-9-7.tex}
%% \input{figures/form/latex/AES-openssl-1-0-2f.tex}
%% \input{figures/form/latex/AES-openssl-1-0-2k.tex}
%%\input{figures/form/latex/AES-openssl-1-1-0f.tex}
%%\input{figures/form/latex/AES-openssl-1-1-1.tex}


\end{document}


