\section{Discussions and Limitations}
While recent works have reported lots
of potential side-channel vulnerabilities, we find most of them are not patched by
developers because they do not think those vulnerabilities are severe enough.
However, side-channels are inevitable in software. In fact, many of them
are easily spotted but it is hard to fix all of them. They need a
tool that can automatically estimate the sensitive level of each vulnerabilities.
Second, addressing old vulnerabilities may also introduce new leakage
sites. Our evaluation result confirms that many of them are not 
severe and can be safely ignored. So software engineers can focus on
``severe'' leakages. For example, our tool will report that 
the modular exponentiation with square and multiply algorithms can
leak much information than a key length validation function.

\tool{} can be used by software developers to find sensitive vulnerabilities
and reason about countermeasures.
\tool{} estimates the amount of leaked information for each side-channels
in particular executions. \tool{} is useful for software
engineers to tests programs and fix vulnerabilities.
\tool{} works on native x86 execution traces. The design, which is very
precise in terms of true leakages compared to other static source code
method~\cite{197207,BacelarAlmeida:2013:FVS:2483313.2483334}, can omit
leakages on other traces. However, as the tool is intended for debugging and testing.
we think it is software engineers' responsibilty to select the input and trigger the 
path in which they are interested.

We use the amount of leaked information to represent the sensitive level of each
side-channel vulnerability. Although imperfect, \tool{} produces for a reasonable
measurement for each leaks. For example, the simple modular exponentiation is notoriously
famous for various side-channel attacks~\cite{kocher1996timing}. During the execution, the single 
leak points will be executed multiple times and each times it leak one different bit.
Therefore, \tool{} reports the vulnerability can leak the whole key. However, no every
leak points inside a loop are serious. Because very often the leakages may leaks the same
bit in the original key and each leaks are not independent. \tool{} model each times a leak
as math formula and use the conjunction of those formulas to describe the total effect.

%% At the early stage of the
%% project, we hope to find some sensitive leakages but are neglected by
%% communities for years. But somehow every sensitive leakages identified by
%% \tool{} are known to people before. We think the main reason is that we only
%% test famous crypto algorithms in well-known crypto libraries. Those code bases
%% have been studied for years. So it is unlikely to have unknown sensitive
%% leakages.
%% We leave the task of applying \tool{} on other libraries and
%% non-crypto libraries in the future.
%

%\tool{} works on the native x86 instructions, while some existing
%works~\cite{197207,BacelarAlmeida:2013:FVS:2483313.2483334} find side-channels
%vulnerabilities from source code level or intermediate languages (e.g., VEX,
%REIL). Apart from the scalability issues for IR implementations, \tool{} is
%designed to work on the native x86 instructions for the following
%considerations. First, compliers can introduce or mitigate side channels
%vulnerabilities. For example, the GCC compiler may or may not translate the $!$
%operator into conditional branches. If the branch is secret-dependent, the
%attacker could learn some sensitive information. However, source-based methods
%fail to know how those the machine code looks like, which leads to false
%positives or false negatives.  
%Second, we find many crypto libraries have lots of inlined assembly code. In
%general, it is hard to convert assembly code into source code or IR.


