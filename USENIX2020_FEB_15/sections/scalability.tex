\section{Scalable to Real-world Crypto Systems}
\label{sec:scala}

In \S\ref{sec:trace-qif}, we propose an advanced information leakage definition for
realistic attack scenarios to model two types of address-based side-channel
leakages as math formulas, and quantify them by calculating the number of input
keys ($K^o$) that satisfy those math formulas. Intuitively, we can use
traditional symbolic execution to capture math formulas and model counting
to get the number of satisfying input keys ($K^o$). However, some preliminary
experiments show that the above approach suffers from unbearable costs, which
impede its usage to detect and quantify side-channel leakages in real-world
applications. In this section, we begin by discussing the bottlenecks of
applying the above approaches in real-world cryptosystems. After that, we
propose our methods.

In general, \tool{} faces the following performance and cost  challenges in
order to \emph{scale to production crypto system analysis}.
\begin{itemize}
      \item Symbolic execution (\textbf{Challenge C2})
      \item Constraint solving (\textbf{Challenge C3})
      \item Counting the number of items in $K^o$ (\textbf{Challenge C4})
\end{itemize}

\subsection{Trace-oriented Symbolic Execution}
While symbolic execution can capture fine-grained semantics of programs, it
is also notorious for its unbearable performance cost. Previous trace-oriented
symbolic execution based
works~\cite{203878,Chattopadhyay:2017:QIL:3127041.3127044} all have large
performance bottlenecks. As a result, those approaches either only apply to
small-size programs~\cite{Chattopadhyay:2017:QIL:3127041.3127044} or apply some
domain knowledge to simplify the analysis. 
%Those tools interpret each
%instruction and update memory cells and registers with formulas that
%captured the semantics of the execution and search different input values that
%can lead to different execution behaviors using constraint solver. 
We implement
the approach presented in \S\ref{sec:trace-qif} and model the side-channels as
formulas. While the tool can finish analyzing some simple cases like AES, it can
not handle complicated cases like RSA.
We observe that finding side-channels using symbolic execution is different from
traditional general symbolic execution and can be optimized to be as efficient
as other methods with approaches below.

\subsubsection{Interprete Instructions Symbolically}
Existing binary analysis tools~\cite{shoshitaishvili2016state,
10.1007/978-3-642-22110-1_37} usually translate machine instructions into
intermediate languages (IR). The reason is that the number of machine instructions is
enormous, and the semantics of each instruction is complex. Intel Developer
Manual~\cite{intelsys} introduces more than 1000 different x86 instructions. It
is tedious and hard to implement the manual rule for each instruction. 
On the contrary,
IR typically has fewer instructions compared to the original machine ISA\@.
However, the IR layer, which predigest the implementation
and reduce the workload of those tools, also introduce significant
overhead~\cite{217563}.

First, transferring machine instructions into IR is time-consuming. For example,
REIL IR~\cite{dullien2009reil}, adopted in CacheS~\cite{236338}, has multiple
transform processes, from binary to VEX IR, BAP IR, and finally REIL IR\@. As IR
can also introduce additional conditional jump instructions, in order to
precisely identify secret-dependent control-flows, it also need to rule out
conditional jump instructions introduced by IR.
Second, IR increases the total number of instructions. For example, x86
instruction \textit{test eax, eax} transfers into 18 REIL IR instructions. If we
assume the time of symbolically executing one instruction is constant, the
design of adopting IR layers can introduce large overhead.

\vspace*{2pt}
\textbf{Our Solution to Challenge C2:}
We adopt the approach from QSYM~\cite{217563} and implement the symbolic execution
directly on the top of x86 instructions. Table~\ref{scala:ir} shows that
eliminating the IR layer can reduce the number of instructions executed during
the analysis.

\begin{table}%[ht]
      \centering\small\footnotesize
      \caption{The number of x86,  % instructions and the number of 
             REIL IR, and VEX IR instructions on the traces of crypto programs.}
      \label{scala:ir}
\vspace*{-9pt}
      \resizebox{\columnwidth}{!}{%

            \begin{tabular}{cccc}
                  \hline
                                    & \begin{tabular}[c]{@{}c@{}}Number of\\ x86 Instructions\end{tabular} & \begin{tabular}[c]{@{}c@{}}Number of\\ VEX IR\end{tabular} & \begin{tabular}[c]{@{}c@{}}Number of\\ REIL IR\end{tabular} \\ \hline
                  AES OpenSSL 0.9.7 & $1,704$                   & $23,938$ (15x)            & $62,045$ (36x)            \\
                  DES OpenSSL 0.9.7 & $2,976$                   & $41,897$ (15x)            & $100,365$ (33x)           \\
                  RSA OpenSSL 0.9.7 & $1.6*10^7$                & $2.4*10^8$ (15x)          & $5.9*10^8$ (37x)          \\
                  RSA mbedTLS 2.5  & $2.2*10^7$                & $3.1*10^8$ (15x)          & $8.6*10^8$  (39x)         \\ \hline
            \end{tabular}
      }
\end{table}

\subsubsection{Constraint Solving}
As discussed in \S\ref{side-channel:condition}, the problem of identifying
side-channels can be reduced to the question below.

\begin{quote}
      \textit{Can we find two different input variables $k_1, k_2 \in K$ that
            satisfy the formula $f_a(k_1) \neq f_a(k_2)$?}
\end{quote}

Existing approach relies on satisfiability modulo theories (SMT) solvers (e.g,
Z3~\cite{DeMoura:2008:ZES:1792734.1792766}) to find satisfying $k_1$ and $k_2$.
We argue that while it is a universal approach to solving constraints with SMT
solvers, for constraints with the above formats, using custom heuristics and
testing is much more efficient in practice. Constraint solving is a decision
problem expressed in logic formulas. SMT solvers transfer the inputted SMT
formula into the boolean conjunctive normal form (CNF) and feed it into the
internal boolean satisfiability problem (SAT) solver. The translation process,
called ``bit blasting'', is time-consuming. Also, as the SAT problem is a
well-known NP-complete problem, it is also hard to deal when it comes to
practical uses with huge formulas. Despite the rapid development of SMT solvers
in recent years, constraint solving remains one of the obstacles to achieve the
scalability for real-world cryptosystems.

\vspace*{2pt}
\textbf{Our Solution to Challenge C3:}
Instead of feeding the formula $f_a(k_1) \neq f_a(k_2)$ into a SMT solver, we
just randomly pick up $k_1, k_2 \in K$ and test them if they can satisfy the
formula. Our solution is based on the following intuition. For most combination
of $(k_{1}, k_{2} )$, the formula $f_a(k_1) \neq f_a(k_2)$ holds. As long as
$f_a$ is not a constant function, such $k_1, k_2$ must exist. For example,
suppose each time we only have 5\% chance to find such $k_1, k_2$, then after we
test with different input combination with 100 times, we have $1 -
(1-0.05)^{100} = 99.6\%$ chance find such $k_1, k_2$. Such random algorithms
work well for our problem.

\subsection{Counting the Number}
\label{MCreasons}
The problem of quantifying the amount of leaked information can be reduced to
the problem of computing the number of items in $K^o$, according to
Definition~\ref{def} introduced in \S\ref{sec:trace-qif}. However, we find that while
there are various propositional model counters (e.g., \#SAT), they are not
sufficient scalable for production cryptosystem analysis.
%there is no open source modulo theories counter (\#SMT) available.

One straightforward method approximating the number of solutions is based on Monte Carlo
sampling. However, the number of satisfying values could be exponentially small.
Consider the formula $f_i\equiv{k_1} = 1\land{k_2} = 2\land{k_3} = 3\land{k_4} =
4$, where $k_1$, $k_2$, $k_3$, and $k_4$ each represents one byte in the
original sensitive input buffer, there is only one satisfying solution of total
$2^{32}$ possible values, which requires exponentially many samples to get a
tight bound. Monte Carlo method also suffers from the curse of dimensionality.
For example, the length of an RSA private key can be as long as 4096 bits. If we
take each byte (8 bits) in the original buffer as one symbol, the formula can
have as many as 512 symbols.

\vspace*{6pt}
\textbf{Our Solution to Challenge C4:}
We adopt multiple-step Monte Carlo sampling methods to count the number of
possible inputs that satisfy the logic formula groups. The key idea is to split
those constraints into several small formulas and sample them independently.
%We will introduce the method in the following subsection.

\subsection{Information Leakage Estimation}

\newcommand{\addr}[1]{{l}_{#1}}
\renewcommand{\addr}[1]{{\gamma}_{#1}}
\renewcommand{\addr}[1]{{\zeta}_{#1}}
\renewcommand{\addr}[1]{{\xi}_{#1}}

In this section, we present the algorithm to calculate the information leakage
based on Definition~\ref{def} (\S\ref{sec:trace-qif}), answering to
\textbf{Challenge C4}.

\subsubsection{Problem Statement}
For each leakage site, we model it with a math formula constraint with the
method presented in~\S\ref{side-channel:condition}. Suppose the address of the
leakage site is $\addr{i}$, we use $c_{\addr{i}}$ to denote the constraint. For
multiple leakage sites, we take the conjunction of those constraints to
represent those leakage sites.

According to the Definition~\ref{def}, to calculate the amount of leaked
information, the key is to calculate 
%%$\frac{|K|}{|K^o|}$. 
%%$K^o$ represents the
%%set that contains every input keys that satisfy the constraint. As the
%%cardinality of $K$ is known, the primary problem is to estimate 
the cardinality
of $K^o$. Suppose an attacker can observe $n$ leakage sites, and each leakage
site has the following constraints: $c_{\addr{1}}, c_{\addr{2}}, \ldots,
c_{\addr{n}}$ respectively. The total leakage has the constraint
$c_t({\addr{1}},{\addr{2}},\ldots,{\addr{n}}) = c_{\addr{1}} \land c_{\addr{2}}
\land \ldots \land c_{\addr{n}}$. 
%%The problem of estimating the total leaked
%%information can be reduced to the problem of counting the number of different
%%solutions that satisfies the constraint
%%$c_t({\addr{1}},{\addr{2}},\ldots,{\addr{n}})$. 
A native method for
approximating the result is to pick elements $k$ from $K$ and check if the
element also contained in $K^o$. Assume $q$ elements satisfy this condition. In
expectation, we can use $\frac{k}{q}$ to approximate the value of
$\frac{|K|}{|K^o|}$.

However, as discussed in \S\ref{MCreasons}, the above sampling method will
typically fail in practice due to the following two problems:

\begin{enumerate}
      \item The curse of dimensionality. $c_t({\addr{1}},\ldots,{\addr{n}})$ is
            the conjunction of many constraints. Therefore, the input variables
            of each constraints will also be the input variables of the
            $c_t({\addr{1}},\ldots,{\addr{n}})$. The sampling method will fail
            as $n$ increases. For example, if the program has $2$ byte input
            equals to 2, the whole search space is a $256^2$ cube. If we want
            the sampling distance between each point equals to $d$, we need
            $256^2d$ points. If the program has $10$ byte input, we need
            $256^{10}d$ points if we still we want the sampling distance equals
            to $d$.

      \item The number of satisfying assignments could be exponentially small.
            According to Chernoff bound, we need exponentially many samples to
            get a tight bound. On an extreme situation, if the constraint only
            has one unique satisfying solution, the simple Monte Carlo method
            cannot find the satisfying assignment even after sampling many
            points.
\end{enumerate}

However, despite the two problems, we also observe two characteristics of the
problem:
\begin{enumerate}
      \item $c_t({\addr{1}},{\addr{2}},\ldots,{\addr{n}})$ is the conjunction of
            several short constraints $c_{\addr{i}}$. The set containing the
            input variables of $c_{\addr{i}}$ is the subset of the input
            variables of $c_t({\addr{1}},{\addr{2}},\ldots,{\addr{n}})$. Some
            constraints have completely different input variables from other
            constraints.

            \item Each time when we sample $c_t({\addr{1}},{\addr{2}},\ldots,{\addr{n}})$
            with a point, the sampling result is \emph{Satisfied} or not \emph{Not Satisfied}.
            The result is randomly generated in a way that does not depend on the result in 
            previous experiments. Also, as the amount of leaked information is calculated
            by $\log$ function, we do not need to precisely count the number of solutions for
            a given constraint.
            %\item For each constraint $F(C_{{addr}_i})$, the satisfying assignments
            %are close to each other, which means if we find one satisfying assignment, we 
            %are more likely to find other satisfying assignments nearby than randomly
            %pick one point in the whole searching space.

\end{enumerate}

In regard to the above problems, we present our methods. First, we split
$c_t(\addr{1},\addr{2},\ldots,\addr{n})$ into several independent constraint
groups. After that, we run a multi-step sampling method for each constraint.

\subsubsection{Maximum Independent Partition}

For a constraint $c_{\addr{i}}$, we define function $\pi$, which maps the
constraint into a set of different input symbols. For example, $\pi(k1 + k2 >
128) = \{k1, k2\}$.

\begin{mydef}[]
      \label{independentC}
      Given two constraints $c_m$ and $c_n$, we call them independent iff
      $$\pi(c_m) \cap \pi(c_n) = \emptyset$$
\end{mydef}

Based on the Definition~\ref{independentC}, we can split the constraint
$c_t(\addr{1},\addr{2},\ldots,\addr{n})$ into several independent constraints.
There are many partitions. For our project, we are interested in the following
one.

\begin{mydef}\label{Goodpartition}
      For the constraint $c_t(\addr{1},\addr{2},\ldots,\addr{n})$,
      we call the constraint group
      $g_{1}, g_{2}, \ldots, g_{m}$
      the maximum independent partition of $c_t(\addr{1},\addr{2},\ldots,\addr{n})$ iff
      \begin{enumerate}
            \item $g_{1} \land g_{2} \land \ldots \land g_{m} = c_t(\addr{1},\addr{2},\ldots,\addr{n})$
            \item $\forall i, j \in \{1, 2, \ldots, m\} \quad \textrm{and} \quad
                        i \neq j, \quad\pi(g_{i}) \cap \pi(g_{j}) = \emptyset $
            \item For any other partitions  $h_{1}, h_{2}, \ldots, h_{m'}$ satisfy 1) and
                  2), $m \geq m'$
      \end{enumerate}

\end{mydef}

The reason we want a good partition of constraints is that we want to reduce
the dimensions. Consider the example in the previous section,
$$c: ({k_1} = 1)\land({k_2} = 2)\land({k_3} > 4)\land({k_3} - {k_4} > 10)$$ A
good partition of $F$ would be
$$g_{1}: ({k_1} = 1)\quad g_{2}: ({k_2} = 2)\quad g_{3}: ({k_3} > 4) \land
({k_3} - {k_4} > 10)$$ So instead of sampling in the four dimension space, we
can sample each constraint in the less dimension space and combine them together
with Theorem~\ref{IndependentConstraint} .

\begin{theorem}
      \label{IndependentConstraint}
      Let $g_{1}, g_{2}, \ldots, g_{m}$ be a maximum independent partition of
      $c_t(\addr{1},\addr{2},\ldots,\addr{n})$.
      $K_c$ is the input set that satisfies constraint $c$. We can have the following
      equation in regard to the size of $K_c$
      $$|K_{c_t(\addr{1},\addr{2},\ldots,\addr{n})}| = |K_{g_{1}}|*|K_{g_{2}}|*\ldots*|K_{g_{n}}|$$
\end{theorem}

With Theorem~\ref{IndependentConstraint}, we can transfer the problem of
counting the number of solutions to a complicated constraint in high-dimension
space into counting solutions to several small constraints. The algorithm to
compute the Maximum Independent Partition of the
$c_t(\addr{1},\addr{2},\ldots,\addr{n})$ is shown in
Appendix~\ref{appendix:partition}.

%% We apply the following
%% algorithm~\ref{algo:max-inde} to get the Maximum Independent Partition of the
%% $c_t(\addr{1},\addr{2},\ldots,\addr{n})$.

%% \IncMargin{1em}
%% \begin{algorithm}[h]
%%       \DontPrintSemicolon
%%       \SetKwInOut{Input}{input}\SetKwInOut{Output}{output}
%%       \Input{$c_t(\addr{1},\addr{2},\ldots,\addr{n}) = c_{\addr{1}} \land c_{\addr{2}} \land \ldots \land c_{\addr{m}}$}
%%       \Output{The Maximum Independent Partition of $G = \{g_{1}, g_{2}  , \ldots,  g_{m} \}$ }
%%       \For{$i\leftarrow 1$ \KwTo $n$}
%%       {
%%             $S_{c_i}$ $\leftarrow$ $\pi(c_{\addr{i}})$ \;
%%             \For{$g_{i} \in G$}
%%             {
%%                   $S_{g_i}$ $\leftarrow$ $\pi(g_{i})$ \;
%%                   $S$ $\leftarrow$ $S_{C_i} \cap S_{G_i}$  \;
%%                   \If{$S \neq \emptyset$}
%%                   {
%%                         $g_{i} \leftarrow g_{i} \land g_{\addr{i}}$ \;
%%                         \textbf{break} \;
%%                   }
%%                   Insert $c_{\addr{i}}$ to $G$
%%             }
%%       }
%%       \caption{The Maximum Independent Partition}
%%       \label{algo:max-inde}
%% \end{algorithm}
%% \DecMargin{1em}

\subsubsection{Multiple Step Monte Carlo Sampling}

After we split those constraints into several small constraints, we count the
number of solutions for each constraint. Even though the dimension has been
significantly reduced after the previous step, this is still a \#P problem. For our
project, we apply the approximate counting instead of exact counting for two
reasons. First, we do not need to have a very precise result of the exact number
of total solutions since the information is defined with a logarithmic function.
We do not need to distinguish between constraints having $10^{10}$ or $10^{10} +
10$ solutions; they are very close to after taking logarithmic. Second, the
precise model counting approaches, like Davis-Putnam-Logemann-Loveland (DPLL)
search, have difficulty scaling up to large problem sizes.

We apply the ``counting by sampling" method. The basic idea is as follows. For
the constraint $g_{i}= c_{i_1} \land c_{i_2} \land ,\ldots, \land c_{i_j} \land
,\ldots, \land c_{i_m}$, if the solution satisfies $g_{i}$, it should also
satisfy any constraint from $c_{i_1}$ to $c_{i_m}$. In other words, $K_{c_gi}$
should be the subset of $K_{c_1}$, $K_{c_2}$, \ldots , $K_{c_m}$. We notice that
$c_i$ usually has less numbers of input compared to $g_{i}$. For example, if
$c_{i_j}$ has only one 8-bit input variable, we can find the exact solution set
$K_{c_{i_j}}$ of $c_{i_j}$ by trying every possible 256 solutions. After that,
we can only generate random input numbers for the rest input variables in
constraint $g_{i}$. With this simple yet effective trick, we can reduce the number of input
while still ensure the accuracy.
The detailed algorithm is shown in Appendix~\ref{appendix:montecarlo}.

%% the algorithm
%\IncMargin{1em}
%\begin{algorithm}
%\SetAlgoLined
%\DontPrintSemicolon

%\KwIn{{The constraint $G_{i}= C_{i_1} \land C_{i_2}
%\land \ldots \land C_{i_m}$}}    
%\KwOut{{The number of assignments that satisfy $G_{i}$ $|K_{G_{i}}|$}}

%\SetKwProg{RW}{RandomWalk}{}{}
%\SetKwProg{MM}{MetropolisMove}{}{} 
%$n$: the number of sampling times \;
%$P$: a probability generator \;
%$k$: the input assignment \;
%$n_{s}$: the number of satisfying assignments \;
%$\#k$: the satisfying number of k \fixme{this number is not used syntactically} \; 
%Initialization: \;
%$\#{k_0}$ $\leftarrow$ $\sum_{j=1}^{m}C_{i_j}(k_0)$ \;
%\For{$t\leftarrow 1$ \KwTo $n$} {
%      $p$ $\leftarrow$ $P$ \;
%      \If{$p \geq 0.5$}
%      {
%        $v$ $\leftarrow$ \RW{$v$} {}
%      }
%      \Else{
%            $v$ $\leftarrow$ \MM{$v$} {}
%      }
%      \If{$v$ satisfies $G_{i}$}
%      {$n_{s}$ $\leftarrow$ $n_{s} + 1$}
%}
%$|K_{G_{i}}|$ $\leftarrow$ $n_s|K| / n$
%\caption{Metropolis Sampling}
%\end{algorithm}
%\DecMargin{1em}

\subsubsection{Error Estimation}
\label{sssec:errest}
In this part, we will show that even we use approximating methods, our result
can have probabilistic guarantee that the error of the estimated amount of leaked 
information is less than 1 bit under the central limit theorem (CLT) and uncertainty
propagation theorem.

Let $n$ be the number of samples and $n_s$ be the number of samples that satisfy
the constraint $C$. Then we can get $\hat{p} = \frac{n_s}{n}$. If we repeat the
experiment multiple times, each time we can get a $\hat{p}$. As each
$\hat{p}$ is independent and identically distributed, according to the central limit
theorem, the mean value should follow normal distribution
$ \frac{\bar{p}-E(p)}{\sigma\sqrt{n}} \rightarrow N(0,1) $. Here $E(p)$ is the
mean value of $p$, and $\sigma$ is the standard variance of $p$. If we use the
observed value $\hat{p}$ to the calculate the standard deviation, we can claim that
we have 95\%\footnote{The value falls within two sigmas of the mean.} 
confidence that the error $\Delta p= \bar{p} - E(p)$ falls in the interval:
$$ |\Delta p| \leq 1.96\sqrt{\frac{ \hat{p} (1- \hat{p} )}{n}}$$

Since we use $L = \log_{2}p$ to estimate the amount of leaked information, we
can have the following error propagation formula $\Delta L = \frac{\Delta
p}{p\ln2}$ by taking the derivative. For \tool, we want the error of estimated leaked
information ($\Delta L$) to be less than 1 bit. So we can get $\frac{\Delta
p}{p\ln2} \leq 1$. As long as $ n \geq \frac{1.96^2(1-p)}{p(\ln2)^2}$, we have
95\% confidence that the error of estimated leaked information is less than 1 bit.
During the simulation, if $n$ and $p$ satisfy the above inequation, Monte Carlo
simulation will terminate.
