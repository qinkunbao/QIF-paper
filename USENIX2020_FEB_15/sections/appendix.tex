
\newcommand{\vvv}{\vspace*{-3pt}}

\section{\tool{}' main components}

\label{appendix:code}
\begin{table}[h!]
    \centering
    %    \resizebox{.8\columnwidth}{!}{
    \caption{\tool{}' main components and sizes}\label{tbl:implementation}
    \begin{tabular}{lr@{~}@{}l}
        \hline
        Component            & \multicolumn{2}{c}{Lines of Code (LOC)}             \\ \hline
        Trace Logging        & 501 lines                               & of C++    \\
        Symbolic Execution   & 14,963 lines                            & of C++    \\
        Data Flow            & 451 lines                               & of C++    \\
        Monte Carlo Sampling & 603 lines                               & of C++    \\
        Others               & 211 lines                               & of Python \\ \hline
        Total                & 16,729 lines                            &           \\\hline
    \end{tabular}
    %    }
\end{table}



\section{Algorithm to Compute the Maximum Independent Partition}
\label{appendix:partition}
~
{\small
\IncMargin{1em}
\begin{algorithm}[h]\small
    \DontPrintSemicolon
    \SetKwInOut{Input}{input}\SetKwInOut{Output}{output}
    \Input{$c_t(\addr{1},\addr{2},\ldots,\addr{n}) = c_{\addr{1}} \land c_{\addr{2}} \land \ldots \land c_{\addr{m}}$}
    \Output{The Maximum Independent Partition of $G = \{g_{1}, g_{2}  , \ldots,  g_{m} \}$ }
    \For{$i\leftarrow 1$ \KwTo $n$}
    {
        $S_{c_{\addr{i}}}$ $\leftarrow$ $\pi(c_{\addr{i}})$ \;
        \For{$g_{i} \in G$}
        {
            $S_{g_j}$ $\leftarrow$ $\pi(g_{j})$ \;
            $S$ $\leftarrow$ $S_{c_{\addr{i}}} \cap S_{g_j}$  \;
            \If{$S \neq \emptyset$}
            {
                $g_{j} \leftarrow g_{i} \land g_{\addr{i}}$ \;
                \textbf{break} \;
            }
            Insert $c_{\addr{i}}$ to $G$
        }
    }
    \caption{The Maximum Independent Partition}
    \label{algo:max-inde}
\end{algorithm}
\DecMargin{1em}
}
~
\newpage
\section{Algorithm to Compute the Number of Satisfying Assignments}
\label{appendix:montecarlo}
~
{\small
\IncMargin{1em}
\begin{algorithm}\small
    \SetAlgoLined
    \DontPrintSemicolon

    \KwIn{{The constraint $g_{i}= c_{i_1} \land c_{i_2}
                    \land \ldots \land c_{i_m}$}}
    \KwOut{{The number of assignments that satisfy $g_{i}$ $|K_{g_{i}}|$}}

    $n$: the number of sampling times \;
    $S_{c_i}$: the set contains input variables for $c_{i}$ \;
    $n_{s}$: the number of satisfying assignments \;
    $N_{c_t}$: the set contains all solution for $c_t$ \;
    $r$: times of reducing $g$\;
    $k$: the input variable \;
    $R$: a function that produces a random point from $S_{c_i}$\;
    %$\#k$: the satisfying number of k \fixme{this number is not used syntactically} \;
    %Initialization: \;
    $r$ $\leftarrow$ $1$,
    $n$ $\leftarrow$ $0$ \;
    \For{$t$ $\leftarrow$ $1$ \KwTo $m$} {
        $S_{c_t}$ $\leftarrow$ $\pi(c_t)$ \;
        \If{$|S_{c_t}| = 1$}
        {
            $N_{c_t}$ $\leftarrow$ Compute all solutions of $c_i$ \;
            $N_{c_t} = \{n_1, \ldots, n_m\},\ S_{c_t} = \{k\}  $ \;
            $g_{i} = $ $g_i(k=n_1) \land \ldots \land g_i(k=n_m)$ \;
            $r \leftarrow r+1$ \;

        }
    }
    \While{$n \leq \frac{6p}{1-p}$} {
        $S_{g_i}$ $\leftarrow$ $\pi(g_i)$ \;
        $v \leftarrow R(S_{g_i})$ 
        \If{$v$ satisfies $g_i$}
        {
           $n_s \leftarrow n_s + 1$
        }
        $n \leftarrow n +1,\ p = \frac{n_s}{n}$
    }

    $|K_{g_{i}}|$ $\leftarrow$ $n_s|K| / (n * r * range(k))$
    \caption{Multiple Step Monte Carlo Sampling}
\end{algorithm}
\DecMargin{1em}
}
~\vvv

\section{AES Lookup Tables Leakage}
\begin{figure}[h!]
    \centering
    \begin{lstlisting}[xleftmargin=.01\textwidth,xrightmargin=.01\textwidth]
int mbedtls_internal_aes_encrypt( mbedtls_aes_context *ctx,
const unsigned char input[16],
unsigned char output[16] )
{
uint32_t *RK, X0, X1, X2, X3, Y0, Y1, Y2, Y3;
...
for( i = ( ctx->nr >> 1 ) - 1; i > 0; i-- )
{
  AES_FROUND( Y0, Y1, Y2, Y3, X0, X1, X2, X3 ); // Leakage 1
  AES_FROUND( X0, X1, X2, X3, Y0, Y1, Y2, Y3 ); // Leakage 2
}
AES_FROUND( Y0, Y1, Y2, Y3, X0, X1, X2, X3 );   // Leakage 3
X0 = *RK++ ^ \                                  // Leakage 4
    ( (uint32_t) FSb[ ( Y0       ) & 0xFF ]       ) ^
    ( (uint32_t) FSb[ ( Y1 >>  8 ) & 0xFF ] <<  8 ) ^
    ( (uint32_t) FSb[ ( Y2 >> 16 ) & 0xFF ] << 16 ) ^
    ( (uint32_t) FSb[ ( Y3 >> 24 ) & 0xFF ] << 24 );
// X1, X2, X3 do the same computation as X0
...                                          // Leakage 5,6,7
PUT_UINT32_LE( X0, output,  0 );
...
return( 0 );
}
\end{lstlisting}
    \vspace*{-6pt}
    \caption{Function \textsf{mbedtls\_internal\_aes\_encrypt}}
    \label{mbedtls_aes}
    \vspace*{-9pt}
\end{figure}

\clearpage
\section{Minor side-channels vulnerability}
\label{appendix:minor:vul}
Here we present a side-channel vulnerability that leaks
less than one bit information by~\tool{}. The vulnerability exists
from OpenSSL 0.9.7 to OpenSSL 1.1.1. Shown in Figure~\ref{fig:old_sqr2} 
and Figure~\ref{fig:new_sqr2}, \textbf{m} is a big number that
derives from the private key.
At line 6, it can leak the last bit of \textbf{m} by observing the branch. 
As the leak is tiny,
we think developers do not have enough motivations
to fix the vulnerability.

\begin{figure}[h]
\centering
\begin{lstlisting}[xleftmargin=.02\textwidth,xrightmargin=.01\textwidth]
int BN_mod_exp_mont_consttime(BIGNUM *rr, 
const BIGNUM *a, const BIGNUM *p,
const BIGNUM *m, BN_CTX *ctx, BN_MONT_CTX *in_mont)
{
... 
if (!(m->d[0] & 1)) {
    ... 
    return 0;
}
bits = BN_num_bits(p);
if (bits == 0) 
...
}
\end{lstlisting}
    \vspace*{-6pt}
    \caption{\textsf{BN\_mod\_exp\_mont\_consttime} in OpenSSL 0.9.7}
    \label{fig:old_sqr2}
    \vspace*{-6pt}
\end{figure}



\begin{figure}[h]
    \centering
    \begin{lstlisting}[xleftmargin=.02\textwidth,xrightmargin=.01\textwidth]
int BN_mod_exp_mont(BIGNUM *rr, 
const BIGNUM *a, const BIGNUM *p,
const BIGNUM *m, BN_CTX *ctx, BN_MONT_CTX *in_mont)
{
...
if (!BN_is_odd(m)) {
    ...
    return 0;
}
bits = BN_num_bits(p);
if (bits == 0) 
...
}

\end{lstlisting}
    \vspace*{-6pt}
    \caption{\textsf{BN\_mod\_exp\_mont} in OpenSSL 1.1.1}
    \label{fig:new_sqr2}
    \vspace*{-6pt}
\end{figure}

\section{Unknown Leaks in OpenSSL 1.1.1}
Shown in Table~\ref{tab:RSAOpenSSL1.1.1}, \tool{} discovers a series
of side-channel vulnerabilities in the up-to-date version of OpenSSL library. 
However, many of them are negligible quantified by \tool{}.
Here we present a unknown vulnerability. Despite the following
vulnerability leaks through the same code patterns as the vulnerability shown
in \S\ref{appendix:minor:vul}, \tool{} shows the following code can leak
around 15 bits from the original key.
\label{appendix:unknown}
\begin{figure}[h]
\centering
\begin{lstlisting}[xleftmargin=.02\textwidth,xrightmargin=.01\textwidth]
while (!BN_is_bit_set(B, shift)) { /* note that 0 < B */
    shift++;
    if (BN_is_odd(X)) {
        if (!BN_uadd(X, X, n))
            goto err;
    }
    ...
    if (!BN_rshift1(X, X))
        goto err;
}

\end{lstlisting}
    \vspace*{-6pt}
    \caption{Unknown sensitive secret-dependent branch leaks from function 
             \textsf{int\_bn\_mod\_inverse} in OpenSSL 1.1.1. Same as the example
             in \S\ref{appendix:minor:vul}, the code can leak the last digit from
             big number \textbf{X}. However, the leak is more sensitive because of the 
             function \textsf{BN\_rshift1}. Each time function \textsf{BN\_rshift1}
             will shift \textbf{X} right by one and places the result in \textbf{X}. Therefore,
             an attacker can infer multiple bits of \textbf{X} by observing the branch at line 3.}
    \label{fig:unknown}
    \vspace*{-6pt}
\end{figure}

\section{Detailed Experimental Results}
\label{sec:result-table}

Here we present the detailed experimental results.
Due to space limitation, we select the representative implementations of
AES, DES, and RSA in
mbed TLS 2.5,
OpenSSL 1.1.0f,  and
OpenSSL 1.1.1.  
%For RSA, we also include OpenSSL 1.0.2f and OpenSSL 1.0.2k.
The results are representative to other versions.
All the results will be made available in electronic format online
when the paper is published. %at \fixme{http://tinyurl}.

In all the tables presented in this appendix, the mark ``$*$'' means timeout,
which indicates more severe leakages. See \S\ref{loc:timeout} for the details.
Also note that we round the calculated numbers of leaked bits to include one digit
after the decimal point, so $0.0$ really means very small amount of leakage, but not exactly zero. See \S\ref{sssec:errest} for the details of error estimate.

  %as an integer.
  %Hence the `0' leakage does not equals to no leakage but some leakage size
 % between 0 and 0.5 bit, see \S\ref{sssec:errest} for the details.


\input{figures/form/latex/DES-mbedTLS-2-5.tex}
%% \input{figures/form/latex/DES-mbedTLS-2-15-1.tex}
%% \input{figures/form/latex/DES-openssl-0-9-7.tex}
%% \input{figures/form/latex/DES-openssl-1-0-2f.tex}
%% \input{figures/form/latex/DES-openssl-1-0-2k.tex}
\input{figures/form/latex/DES-openssl-1-1-0f.tex}
\input{figures/form/latex/DES-openssl-1-1-1.tex}


\input{figures/form/latex/RSA-mbedTLS-2-5.tex}
%% \input{figures/form/latex/RSA-mbedTLS-2-15-1.tex}
%%\input{figures/form/latex/RSA-openssl-0-9-7.tex}
\input{figures/form/latex/RSA-openssl-1-0-2f.tex}
\input{figures/form/latex/RSA-openssl-1-0-2k.tex}
\input{figures/form/latex/RSA-openssl-1-1-0f.tex}
\begin{table}[h!]
\centering\tiny\scriptsize
\caption{Leakages in RSA implemented by OpenSSL 1.1.1}\label{tab:RSAOpenSSL1.1.1}
%\resizebox{\columnwidth}{!}{
\begin{tabular}{l@{~~}rlr@{~~}r}
\hline
\textbf{File} & \textbf{Line No.} & \textbf{Function} & \hspace*{-20em}\textbf{\# Leaked Bits} & \textbf{Type} \\\hline
%\input{figures/form/latex/RSA-openssl-1-1-1-data.tex}
rsa\_ossl.c& 399&rsa\_ossl\_private\_decrypt&0.0 &CF\\
bn\_lib.c& 555&BN\_ucmp&*&CF\\
bn\_gcd.c& 199&int\_bn\_mod\_inverse&1.0 &CF\\
bn\_gcd.c& 247&int\_bn\_mod\_inverse&14.9 &CF\\
bn\_gcd.c& 225&int\_bn\_mod\_inverse&12.3 &CF\\
ct\_b64.c& 168&\_\_udivdi3&0.1 &CF\\
bn\_div.c& 374&bn\_div\_fixed\_top&*&CF\\
bn\_lib.c& 955&bn\_correct\_top&2.6 &CF\\
ct\_b64.c& 168&\_\_memset\_sse2\_rep&0.0 &CF\\
ct\_b64.c& 168&\_\_memset\_sse2\_rep&0.0 &CF\\
ct\_b64.c& 168&\_\_memset\_sse2\_rep&0.0 &DA\\
ct\_b64.c& 168&\_\_memset\_sse2\_rep&0.0 &DA\\
bn\_exp.c& 317&BN\_mod\_exp\_mont&1.0 &CF\\
bn\_asm.c& 592&bn\_mul\_comba8&2.1 &CF\\
bn\_exp.c& 383&BN\_mod\_exp\_mont&0.9 &CF\\
bn\_lib.c& 453&BN\_bn2binpad&0.0 &DA\\
bn\_lib.c& 450&BN\_bn2binpad&0.0 &CF\\
rsa\_oaep.c& 180&RSA\_padding\_check\_PKCS1\_OAEP\_mgf1&0.0 &DA\\
rsa\_oaep.c& 180&RSA\_padding\_check\_PKCS1\_OAEP\_mgf1&0.0 &DA\\
rsa\_oaep.c& 176&RSA\_padding\_check\_PKCS1\_OAEP\_mgf1&0.0 &CF\\
string3.h& 90&SHA1\_Final&0.0 &CF\\
rsa\_oaep.c& 200&RSA\_padding\_check\_PKCS1\_OAEP\_mgf1&0.0 &CF\\
rsa\_oaep.c& 209&RSA\_padding\_check\_PKCS1\_OAEP\_mgf1&0.0 &CF\\
rsa\_oaep.c& 250&RSA\_padding\_check\_PKCS1\_OAEP\_mgf1&0.0 &CF\\
rsa\_oaep.c& 253&RSA\_padding\_check\_PKCS1\_OAEP\_mgf1&0.0 &CF\\
ct\_b64.c& 168&\_\_memset\_sse2\_rep&0.0 &DA\\
ct\_b64.c& 168&\_\_memset\_sse2\_rep&0.0 &DA\\
\hline
\end{tabular}
%}
\renewcommand{\baselinestretch}{1.0}\selectfont
\end{table}



\input{figures/form/latex/AES-mbedTLS-2-5.tex}
%% \input{figures/form/latex/AES-mbedTLS-2-15-1.tex}
%% \input{figures/form/latex/AES-openssl-0-9-7.tex}
%% \input{figures/form/latex/AES-openssl-1-0-2f.tex}
%% \input{figures/form/latex/AES-openssl-1-0-2k.tex}
%%\input{figures/form/latex/AES-openssl-1-1-0f.tex}
%%\input{figures/form/latex/AES-openssl-1-1-1.tex}
