
\newcommand{\vvv}{\vspace*{-3pt}}

\section{\tool{}' main components}

\label{appendix:code}
\begin{table}[h!]
    \centering
    %    \resizebox{.8\columnwidth}{!}{
    \caption{\tool{}' main components and sizes}\label{tbl:implementation}
    \begin{tabular}{lr@{~}@{}l}
        \hline
        Component            & \multicolumn{2}{c}{Lines of Code (LOC)}             \\ \hline
        Trace Logging        & 501 lines                               & of C++    \\
        Symbolic Execution   & 14,963 lines                            & of C++    \\
        Data Flow            & 451 lines                               & of C++    \\
        Monte Carlo Sampling & 603 lines                               & of C++    \\
        Others               & 211 lines                               & of Python \\ \hline
        Total                & 16,729 lines                            &           \\\hline
    \end{tabular}
    %    }
\end{table}



\section{Algorithm to Compute the Maximum Independent Partition}
\label{appendix:partition}
~
{\small
\IncMargin{1em}
\begin{algorithm}[h]\small
    \DontPrintSemicolon
    \SetKwInOut{Input}{input}\SetKwInOut{Output}{output}
    \Input{$c_t(\addr{1},\addr{2},\ldots,\addr{n}) = c_{\addr{1}} \land c_{\addr{2}} \land \ldots \land c_{\addr{m}}$}
    \Output{The Maximum Independent Partition of $G = \{g_{1}, g_{2}  , \ldots,  g_{m} \}$ }
    \For{$i\leftarrow 1$ \KwTo $n$}
    {
        $S_{c_{\addr{i}}}$ $\leftarrow$ $\pi(c_{\addr{i}})$ \;
        \For{$g_{i} \in G$}
        {
            $S_{g_j}$ $\leftarrow$ $\pi(g_{j})$ \;
            $S$ $\leftarrow$ $S_{c_{\addr{i}}} \cap S_{g_j}$  \;
            \If{$S \neq \emptyset$}
            {
                $g_{j} \leftarrow g_{i} \land g_{\addr{i}}$ \;
                \textbf{break} \;
            }
            Insert $c_{\addr{i}}$ to $G$
        }
    }
    \caption{The Maximum Independent Partition}
    \label{algo:max-inde}
\end{algorithm}
\DecMargin{1em}
}
~
\newpage
\section{Algorithm to Compute the Number of Satisfying Assignments}
\label{appendix:montecarlo}
~
{\small
\IncMargin{1em}
\begin{algorithm}\small
    \SetAlgoLined
    \DontPrintSemicolon

    \KwIn{{The constraint $g_{i}= c_{i_1} \land c_{i_2}
                    \land \ldots \land c_{i_m}$}}
    \KwOut{{The number of assignments that satisfy $g_{i}$ $|K_{g_{i}}|$}}

    $n$: the number of sampling times \;
    $S_{c_i}$: the set contains input variables for $c_{i}$ \;
    $n_{s}$: the number of satisfying assignments \;
    $N_{c_t}$: the set contains all solution for $c_t$ \;
    $r$: times of reducing $g$\;
    $k$: the input variable \;
    $R$: a function that produces a random point from $S_{c_i}$\;
    %$\#k$: the satisfying number of k \fixme{this number is not used syntactically} \;
    %Initialization: \;
    $r$ $\leftarrow$ $1$,
    $n$ $\leftarrow$ $0$ \;
    \For{$t$ $\leftarrow$ $1$ \KwTo $m$} {
        $S_{c_t}$ $\leftarrow$ $\pi(c_t)$ \;
        \If{$|S_{c_t}| = 1$}
        {
            $N_{c_t}$ $\leftarrow$ Compute all solutions of $c_i$ \;
            $N_{c_t} = \{n_1, \ldots, n_m\},\ S_{c_t} = \{k\}  $ \;
            $g_{i} = $ $g_i(k=n_1) \land \ldots \land g_i(k=n_m)$ \;
            $r \leftarrow r+1$ \;

        }
    }
    \While{$n \leq \frac{6p}{1-p}$} {
        $S_{g_i}$ $\leftarrow$ $\pi(g_i)$ \;
        $v \leftarrow R(S_{g_i})$ 
        \If{$v$ satisfies $g_i$}
        {
           $n_s \leftarrow n_s + 1$
        }
        $n \leftarrow n +1,\ p = \frac{n_s}{n}$
    }

    $|K_{g_{i}}|$ $\leftarrow$ $n_s|K| / (n * r * range(k))$
    \caption{Multiple Step Monte Carlo Sampling}
\end{algorithm}
\DecMargin{1em}
}
~\vvv

\section{AES Lookup Tables Leakage}
\begin{figure}[h!]
    \centering
    \begin{lstlisting}[xleftmargin=.01\textwidth,xrightmargin=.01\textwidth]
int mbedtls_internal_aes_encrypt( mbedtls_aes_context *ctx,
const unsigned char input[16],
unsigned char output[16] )
{
uint32_t *RK, X0, X1, X2, X3, Y0, Y1, Y2, Y3;
...
for( i = ( ctx->nr >> 1 ) - 1; i > 0; i-- )
{
  AES_FROUND( Y0, Y1, Y2, Y3, X0, X1, X2, X3 ); // Leakage 1
  AES_FROUND( X0, X1, X2, X3, Y0, Y1, Y2, Y3 ); // Leakage 2
}
AES_FROUND( Y0, Y1, Y2, Y3, X0, X1, X2, X3 );   // Leakage 3
X0 = *RK++ ^ \                                  // Leakage 4
    ( (uint32_t) FSb[ ( Y0       ) & 0xFF ]       ) ^
    ( (uint32_t) FSb[ ( Y1 >>  8 ) & 0xFF ] <<  8 ) ^
    ( (uint32_t) FSb[ ( Y2 >> 16 ) & 0xFF ] << 16 ) ^
    ( (uint32_t) FSb[ ( Y3 >> 24 ) & 0xFF ] << 24 );
// X1, X2, X3 do the same computation as X0
...                                          // Leakage 5,6,7
PUT_UINT32_LE( X0, output,  0 );
...
return( 0 );
}
\end{lstlisting}
    \vspace*{-6pt}
    \caption{Function \textsf{mbedtls\_internal\_aes\_encrypt}}
    \label{mbedtls_aes}
    \vspace*{-9pt}
\end{figure}

\clearpage
\section{Minor side-channels vulnerability}
\label{appendix:minor:vul}
Here we present a side-channel vulnerability that leaks
less than one bit information by~\tool{}. The vulnerability exists
from OpenSSL 0.9.7 to OpenSSL 1.1.1. Shown in Figure~\ref{fig:old_sqr2} 
and Figure~\ref{fig:new_sqr2}, \textsf{m} is a big number that
derives from the private key.
At line 6, it can leak the last bit of \textsf{m} by observing the branch. 
As the leak is tiny,
we think developers do not have enough motivations
to fix the vulnerability.

\begin{figure}[h]
\centering
\begin{lstlisting}[xleftmargin=.02\textwidth,xrightmargin=.01\textwidth]
int BN_mod_exp_mont_consttime(BIGNUM *rr, 
const BIGNUM *a, const BIGNUM *p,
const BIGNUM *m, BN_CTX *ctx, BN_MONT_CTX *in_mont)
{
... 
if (!(m->d[0] & 1)) {
    ... 
    return 0;
}
bits = BN_num_bits(p);
if (bits == 0) 
...
}
\end{lstlisting}
    \vspace*{-6pt}
    \caption{\textsf{BN\_mod\_exp\_mont\_consttime} in OpenSSL 0.9.7}
    \label{fig:old_sqr2}
    \vspace*{-6pt}
\end{figure}



\begin{figure}[h]
    \centering
    \begin{lstlisting}[xleftmargin=.02\textwidth,xrightmargin=.01\textwidth]
int BN_mod_exp_mont(BIGNUM *rr, 
const BIGNUM *a, const BIGNUM *p,
const BIGNUM *m, BN_CTX *ctx, BN_MONT_CTX *in_mont)
{
...
if (!BN_is_odd(m)) {
    ...
    return 0;
}
bits = BN_num_bits(p);
if (bits == 0) 
...
}

\end{lstlisting}
    \vspace*{-6pt}
    \caption{\textsf{BN\_mod\_exp\_mont} in OpenSSL 1.1.1}
    \label{fig:new_sqr2}
    \vspace*{-6pt}
\end{figure}

\section{Unknown Leaks in OpenSSL 1.1.1}
Shown in Table~\ref{tab:RSAOpenSSL1.1.1}, \tool{} discovers a series
of side-channel vulnerabilities in the up-to-date OpenSSL library. 
However, many of them are negligible quantified by our \tool{}.
Here we present a unknown vulnerability. Despite the following
vulnerability leaks through from the same code patterns as vulnerabilities
in \S\ref{appendix:minor:vul}, \tool{} shows the following code can leak
around 15 bits from the original key.
\label{appendix:unknown}
\begin{figure}[h]
\centering
\begin{lstlisting}[xleftmargin=.02\textwidth,xrightmargin=.01\textwidth]
while (!BN_is_bit_set(B, shift)) { /* note that 0 < B */
    shift++;
    if (BN_is_odd(X)) {
        if (!BN_uadd(X, X, n))
            goto err;
    }
    if (!BN_rshift1(X, X))
        goto err;
}

\end{lstlisting}
    \vspace*{-6pt}
    \caption{Unknown sensitive secret-dependent branch leaks from function \textsf{int\_bn\_mod\_inverse} in OpenSSL 1.1.1}
    \label{fig:unknown}
    \vspace*{-6pt}
\end{figure}

\section{Detailed Experimental Results}
\label{sec:result-table}

Here we present the detailed experimental results.
Due to space limitation, we select the representative implementations of
AES, DES, and RSA in
mbed TLS 2.5,
OpenSSL 1.1.0f,  and
OpenSSL 1.1.1.  
%For RSA, we also include OpenSSL 1.0.2f and OpenSSL 1.0.2k.
The results are representative to other versions.
All the results will be made available in electronic format online
when the paper is published. %at \fixme{http://tinyurl}.

In all the tables presented in this appendix, the mark ``$*$'' means timeout,
which indicates more severe leakages. See \S\ref{loc:timeout} for the details.
Also note that we round the calculated numbers of leaked bits to include one digit
after the decimal point, so $0.0$ really means very small amount of leakage, but not exactly zero. See \S\ref{sssec:errest} for the details of error estimate.

  %as an integer.
  %Hence the `0' leakage does not equals to no leakage but some leakage size
 % between 0 and 0.5 bit, see \S\ref{sssec:errest} for the details.


\begin{table}[!ht]
\centering\tiny\scriptsize
\caption{Leakages in DES implemented by mbed TLS 2.5}\label{tab:DESmbed TLS2.5}
%\resizebox{\columnwidth}{!}{
\begin{tabular}{lrlrr}
\hline
\textbf{File} & \textbf{Line No.} & \textbf{Function} & \textbf{\# Leaked Bits} & \textbf{Type} \\\hline
des.c& 441&mbedtls\_des\_setkey&0.9 &DA\\
des.c& 438&mbedtls\_des\_setkey&1.0 &DA\\
des.c& 438&mbedtls\_des\_setkey&1.0 &DA\\
des.c& 439&mbedtls\_des\_setkey&1.1 &DA\\
des.c& 439&mbedtls\_des\_setkey&1.0 &DA\\
des.c& 440&mbedtls\_des\_setkey&1.0 &DA\\
des.c& 446&mbedtls\_des\_setkey&0.9 &DA\\
des.c& 446&mbedtls\_des\_setkey&1.0 &DA\\
des.c& 444&mbedtls\_des\_setkey&1.0 &DA\\
des.c& 444&mbedtls\_des\_setkey&1.0 &DA\\
des.c& 443&mbedtls\_des\_setkey&1.0 &DA\\
des.c& 443&mbedtls\_des\_setkey&1.0 &DA\\
des.c& 444&mbedtls\_des\_setkey&1.0 &DA\\
des.c& 445&mbedtls\_des\_setkey&1.1 &DA\\
des.c& 448&mbedtls\_des\_setkey&0.9 &DA\\
\hline
\end{tabular}
%}
\renewcommand{\baselinestretch}{1.0}\selectfont
\end{table}

%% \begin{table}[!ht]
\centering\tiny\scriptsize
\caption{Leakages in DES implemented by mbed TLS 2.15.1}\label{tab:DESmbed TLS2.15.1}
%\resizebox{\columnwidth}{!}{
\begin{tabular}{lrlrr}
\hline
\textbf{File} & \textbf{Line No.} & \textbf{Function} & \textbf{\# Leaked Bits} & \textbf{Type} \\\hline
des.c& 437&mbedtls\_des\_setkey&1.0 &DA\\
des.c& 434&mbedtls\_des\_setkey&1.0 &DA\\
des.c& 434&mbedtls\_des\_setkey&1.0 &DA\\
des.c& 435&mbedtls\_des\_setkey&1.0 &DA\\
des.c& 435&mbedtls\_des\_setkey&1.0 &DA\\
des.c& 436&mbedtls\_des\_setkey&1.0 &DA\\
des.c& 442&mbedtls\_des\_setkey&1.0 &DA\\
des.c& 442&mbedtls\_des\_setkey&1.0 &DA\\
des.c& 440&mbedtls\_des\_setkey&0.9 &DA\\
des.c& 440&mbedtls\_des\_setkey&1.0 &DA\\
des.c& 439&mbedtls\_des\_setkey&1.0 &DA\\
des.c& 439&mbedtls\_des\_setkey&0.9 &DA\\
des.c& 440&mbedtls\_des\_setkey&1.0 &DA\\
des.c& 441&mbedtls\_des\_setkey&1.0 &DA\\
des.c& 444&mbedtls\_des\_setkey&0.9 &DA\\
\hline
\end{tabular}
%}
\renewcommand{\baselinestretch}{1.0}\selectfont
\end{table}

%% \begin{table}[!ht]
\centering\tiny\scriptsize
\caption{Leakages in DES implemented by OpenSSL 0.9.7}\label{tab:DESOpenSSL0.9.7}
%\resizebox{\columnwidth}{!}{
\begin{tabular}{lrlrr}
\hline
\textbf{File} & \textbf{Line No.} & \textbf{Function} & \textbf{\# Leaked Bits} & \textbf{Type} \\\hline
set\_key.c& 380&DES\_set\_key\_unchecked&5.7 &DA\\
set\_key.c& 380&DES\_set\_key\_unchecked&6.6 &DA\\
set\_key.c& 380&DES\_set\_key\_unchecked&7.6 &DA\\
set\_key.c& 380&DES\_set\_key\_unchecked&6.5 &DA\\
set\_key.c& 385&DES\_set\_key\_unchecked&3.2 &DA\\
set\_key.c& 385&DES\_set\_key\_unchecked&1.9 &DA\\
\hline
\end{tabular}
%}
\renewcommand{\baselinestretch}{1.0}\selectfont
\end{table}

%% \begin{table}[!ht]
\centering\tiny\scriptsize
\caption{Leakages in DES implemented by OpenSSL 1.0.2f}\label{tab:DESOpenSSL1.0.2f}
%\resizebox{\columnwidth}{!}{
\begin{tabular}{lrlrr}
\hline
\textbf{File} & \textbf{Line No.} & \textbf{Function} & \textbf{\# Leaked Bits} & \textbf{Type} \\\hline
set\_key.c& 409&DES\_set\_key\_unchecked&9.8 &DA\\
set\_key.c& 417&DES\_set\_key\_unchecked&9.2 &DA\\
set\_key.c& 418&DES\_set\_key\_unchecked&8.0 &DA\\
set\_key.c& 423&DES\_set\_key\_unchecked&5.4 &DA\\
set\_key.c& 420&DES\_set\_key\_unchecked&2.0 &DA\\
set\_key.c& 422&DES\_set\_key\_unchecked&3.4 &DA\\
set\_key.c& 422&DES\_set\_key\_unchecked&2.9 &DA\\
set\_key.c& 425&DES\_set\_key\_unchecked&0.2 &DA\\
\hline
\end{tabular}
%}
\renewcommand{\baselinestretch}{1.0}\selectfont
\end{table}

%% \begin{table}[h!]
\centering\tiny\scriptsize
\caption{Leakages in DES implemented by OpenSSL 1.0.2k}\label{tab:DESOpenSSL1.0.2k}
%\resizebox{\columnwidth}{!}{
\begin{tabular}{lrlrr}
\hline
\textbf{File} & \textbf{Line No.} & \textbf{Function} & \textbf{\# Leaked Bits} & \textbf{Type} \\\hline
set\_key.c& 409&DES\_set\_key\_unchecked&8.2 &DA\\
set\_key.c& 417&DES\_set\_key\_unchecked&10.1 &DA\\
set\_key.c& 418&DES\_set\_key\_unchecked&8.1 &DA\\
set\_key.c& 423&DES\_set\_key\_unchecked&5.7 &DA\\
set\_key.c& 420&DES\_set\_key\_unchecked&3.5 &DA\\
set\_key.c& 422&DES\_set\_key\_unchecked&4.1 &DA\\
set\_key.c& 422&DES\_set\_key\_unchecked&3.6 &DA\\
set\_key.c& 425&DES\_set\_key\_unchecked&0.7 &DA\\
\hline
\end{tabular}
%}
\renewcommand{\baselinestretch}{1.0}\selectfont
\end{table}

\begin{table}[h!]
\centering\tiny\scriptsize
\caption{Leakages in DES implemented by OpenSSL 1.1.0f}\label{tab:DESOpenSSL1.1.0f}
%\resizebox{\columnwidth}{!}{
\begin{tabular}{lrlrr}
\hline
\textbf{File} & \textbf{Line No.} & \textbf{Function} & \textbf{\# Leaked Bits} & \textbf{Type} \\\hline
set\_key.c& 351&DES\_set\_key\_unchecked&7.1 &DA\\
set\_key.c& 353&DES\_set\_key\_unchecked&8.8 &DA\\
set\_key.c& 361&DES\_set\_key\_unchecked&8.0 &DA\\
set\_key.c& 362&DES\_set\_key\_unchecked&5.7 &DA\\
set\_key.c& 362&DES\_set\_key\_unchecked&2.0 &DA\\
set\_key.c& 364&DES\_set\_key\_unchecked&3.5 &DA\\
set\_key.c& 364&DES\_set\_key\_unchecked&4.9 &DA\\
set\_key.c& 365&DES\_set\_key\_unchecked&0.4 &DA\\
\hline
\end{tabular}
%}
\renewcommand{\baselinestretch}{1.0}\selectfont
\end{table}

\begin{table}[!ht]
\centering\tiny\scriptsize
\caption{Leakages in DES implemented by OpenSSL 1.1.1}\label{tab:DESOpenSSL1.1.1}
%\resizebox{\columnwidth}{!}{
\begin{tabular}{lrlrr}
\hline
\textbf{File} & \textbf{Line No.} & \textbf{Function} & \textbf{\# Leaked Bits} & \textbf{Type} \\\hline
set\_key.c& 350&DES\_set\_key\_unchecked&5.8 &DA\\
set\_key.c& 350&DES\_set\_key\_unchecked&6.6 &DA\\
set\_key.c& 350&DES\_set\_key\_unchecked&7.5 &DA\\
set\_key.c& 350&DES\_set\_key\_unchecked&6.4 &DA\\
set\_key.c& 355&DES\_set\_key\_unchecked&1.9 &DA\\
set\_key.c& 355&DES\_set\_key\_unchecked&3.1 &DA\\
\hline
\end{tabular}
%}
\renewcommand{\baselinestretch}{1.0}\selectfont
\end{table}



\begin{table}[!ht]
\centering\tiny\scriptsize
\caption{Leakages in RSA implemented by mbed TLS 2.5}\label{tab:RSAmbed TLS2.5}
%\resizebox{\columnwidth}{!}{
\begin{tabular}{lrlrr}
\hline
\textbf{File} & \textbf{Line No.} & \textbf{Function} & \textbf{\# Leaked Bits} & \textbf{Type} \\\hline
bignum.c& 1617&mbedtls\_mpi\_exp\_mod&0.9 &CF\\
bignum.c& 861&mbedtls\_mpi\_cmp\_mpi&8.5 &CF\\
bignum.c& 862&mbedtls\_mpi\_cmp\_mpi&7.7 &CF\\
bignum.c& 1167&mpi\_mul\_hlp&*&CF\\
bignum.c& 828&mbedtls\_mpi\_cmp\_abs&9.6 &CF\\
bignum.c& 829&mbedtls\_mpi\_cmp\_abs&9.5 &CF\\
\hline
\end{tabular}
%}
\renewcommand{\baselinestretch}{1.0}\selectfont
\end{table}

%% \begin{table}[h!]
\centering\tiny\scriptsize
\caption{Leakages in RSA implemented by mbed TLS 2.15.1}\label{tab:RSAmbed TLS2.15.1}
%\resizebox{\columnwidth}{!}{
\begin{tabular}{lrlrr}
\hline
\textbf{File} & \textbf{Line No.} & \textbf{Function} & \textbf{\# Leaked Bits} & \textbf{Type} \\\hline
bignum.c& 855&mbedtls\_mpi\_cmp\_mpi&*&CF\\
rsa.c& 184&rsa\_check\_context.isra.0&1.0 &CF\\
bignum.c& 825&mbedtls\_mpi\_cmp\_abs&*&CF\\
bignum.c& 197&mbedtls\_mpi\_copy&*&CF\\
bignum.c& 1629&mbedtls\_mpi\_exp\_mod&0.9 &CF\\
bignum.c& 829&mbedtls\_mpi\_cmp\_abs&*&CF\\
bignum.c& 859&mbedtls\_mpi\_cmp\_mpi&*&CF\\
bignum.c& 873&mbedtls\_mpi\_cmp\_mpi&2.8 &CF\\
bignum.c& 874&mbedtls\_mpi\_cmp\_mpi&2.7 &CF\\
bignum.c& 840&mbedtls\_mpi\_cmp\_abs&8.2 &CF\\
bignum.c& 841&mbedtls\_mpi\_cmp\_abs&8.7 &CF\\
bignum.c& 1201&mbedtls\_mpi\_mul\_mpi&*&CF\\
\hline
\end{tabular}
%}
\renewcommand{\baselinestretch}{1.0}\selectfont
\end{table}

%%\begin{table}[h!]
\centering\tiny\scriptsize
\caption{Leakages in RSA implemented by OpenSSL 0.9.7}\label{tab:RSAOpenSSL0.9.7}
%\resizebox{\columnwidth}{!}{
\begin{tabular}{lrlrr}
\hline
\textbf{File} & \textbf{Line No.} & \textbf{Function} & \textbf{\# Leaked Bits} & \textbf{Type} \\\hline
bn\_lib.c& 228&BN\_num\_bits&0.0 &CF\\
bn\_lib.c& 229&BN\_num\_bits&4.1 &DA\\
bn\_shift.c& 152&BN\_lshift&1.0 &CF\\
bn\_lib.c& 670&BN\_ucmp&*&CF\\
a\_gentm.c& 246&\_\_udivdi3&*&CF\\
bn\_div.c& 303&BN\_div&*&CF\\
bn\_add.c& 222&BN\_usub&4.2 &CF\\
bn\_add.c& 256&BN\_usub&*&CF\\
bn\_add.c& 256&BN\_usub&*&CF\\
bn\_gcd.c& 247&BN\_mod\_inverse&1.0 &CF\\
bn\_gcd.c& 268&BN\_mod\_inverse&6.8 &CF\\
bn\_gcd.c& 287&BN\_mod\_inverse&4.4 &CF\\
bn\_gcd.c& 291&BN\_mod\_inverse&15.2 &CF\\
bn\_gcd.c& 272&BN\_mod\_inverse&12.3 &CF\\
bn\_lib.c& 670&BN\_ucmp&*&CF\\
bn\_div.c& 303&BN\_div&*&CF\\
bn\_div.c& 307&BN\_div&*&CF\\
bn\_exp.c& 590&BN\_mod\_exp\_mont\_consttime&1.0 &CF\\
bn\_mont.c& 218&BN\_from\_montgomery&*&CF\\
bn\_asm.c& 691&bn\_sqr\_comba8&*&CF\\
bn\_asm.c& 696&bn\_sqr\_comba8&12.8 &CF\\
bn\_asm.c& 696&bn\_sqr\_comba8&10.5 &CF\\
bn\_asm.c& 704&bn\_sqr\_comba8&*&CF\\
bn\_asm.c& 709&bn\_sqr\_comba8&*&CF\\
bn\_asm.c& 709&bn\_sqr\_comba8&12.8 &CF\\
bn\_asm.c& 710&bn\_sqr\_comba8&14.0 &CF\\
bn\_asm.c& 714&bn\_sqr\_comba8&*&CF\\
bn\_asm.c& 714&bn\_sqr\_comba8&7.4 &CF\\
bn\_asm.c& 715&bn\_sqr\_comba8&*&CF\\
bn\_asm.c& 718&bn\_sqr\_comba8&15.6 &CF\\
bn\_asm.c& 718&bn\_sqr\_comba8&10.9 &CF\\
bn\_asm.c& 720&bn\_sqr\_comba8&*&CF\\
bn\_asm.c& 721&bn\_sqr\_comba8&*&CF\\
bn\_asm.c& 727&bn\_sqr\_comba8&*&CF\\
bn\_asm.c& 728&bn\_sqr\_comba8&*&CF\\
bn\_asm.c& 728&bn\_sqr\_comba8&*&CF\\
bn\_asm.c& 730&bn\_sqr\_comba8&12.1 &CF\\
bn\_asm.c& 730&bn\_sqr\_comba8&9.2 &CF\\
bn\_asm.c& 731&bn\_sqr\_comba8&*&CF\\
bn\_asm.c& 732&bn\_sqr\_comba8&*&CF\\
bn\_asm.c& 733&bn\_sqr\_comba8&*&CF\\
bn\_asm.c& 737&bn\_sqr\_comba8&*&CF\\
bn\_asm.c& 738&bn\_sqr\_comba8&*&CF\\
bn\_asm.c& 738&bn\_sqr\_comba8&*&CF\\
bn\_asm.c& 740&bn\_sqr\_comba8&*&CF\\
bn\_asm.c& 740&bn\_sqr\_comba8&*&CF\\
bn\_asm.c& 746&bn\_sqr\_comba8&*&CF\\
bn\_asm.c& 748&bn\_sqr\_comba8&*&CF\\
bn\_asm.c& 748&bn\_sqr\_comba8&*&CF\\
bn\_asm.c& 692&bn\_sqr\_comba8&*&CF\\
bn\_asm.c& 698&bn\_sqr\_comba8&16.2 &CF\\
bn\_asm.c& 698&bn\_sqr\_comba8&12.1 &CF\\
bn\_asm.c& 705&bn\_sqr\_comba8&*&CF\\
bn\_asm.c& 707&bn\_sqr\_comba8&*&CF\\
bn\_asm.c& 707&bn\_sqr\_comba8&*&CF\\
bn\_asm.c& 716&bn\_sqr\_comba8&*&CF\\
bn\_asm.c& 719&bn\_sqr\_comba8&*&CF\\
bn\_asm.c& 720&bn\_sqr\_comba8&*&CF\\
bn\_asm.c& 721&bn\_sqr\_comba8&*&CF\\
bn\_asm.c& 722&bn\_sqr\_comba8&*&CF\\
bn\_asm.c& 726&bn\_sqr\_comba8&*&CF\\
bn\_asm.c& 727&bn\_sqr\_comba8&*&CF\\
bn\_asm.c& 714&bn\_sqr\_comba8&*&CF\\
bn\_asm.c& 715&bn\_sqr\_comba8&*&CF\\
bn\_asm.c& 716&bn\_sqr\_comba8&*&CF\\
bn\_asm.c& 721&bn\_sqr\_comba8&12.6 &CF\\
bn\_asm.c& 722&bn\_sqr\_comba8&*&CF\\
bn\_asm.c& 726&bn\_sqr\_comba8&*&CF\\
bn\_asm.c& 732&bn\_sqr\_comba8&*&CF\\
bn\_asm.c& 733&bn\_sqr\_comba8&*&CF\\
bn\_asm.c& 737&bn\_sqr\_comba8&*&CF\\
bn\_asm.c& 741&bn\_sqr\_comba8&*&CF\\
bn\_asm.c& 742&bn\_sqr\_comba8&*&CF\\
bn\_asm.c& 742&bn\_sqr\_comba8&*&CF\\
bn\_asm.c& 746&bn\_sqr\_comba8&*&CF\\
bn\_asm.c& 749&bn\_sqr\_comba8&*&CF\\
bn\_asm.c& 700&bn\_sqr\_comba8&17.2 &CF\\
bn\_asm.c& 700&bn\_sqr\_comba8&14.6 &CF\\
bn\_asm.c& 704&bn\_sqr\_comba8&*&CF\\
bn\_asm.c& 726&bn\_sqr\_comba8&9.0 &CF\\
bn\_asm.c& 737&bn\_sqr\_comba8&*&CF\\
bn\_asm.c& 720&bn\_sqr\_comba8&*&CF\\
bn\_asm.c& 742&bn\_sqr\_comba8&*&CF\\
bn\_asm.c& 732&bn\_sqr\_comba8&*&CF\\
bn\_asm.c& 716&bn\_sqr\_comba8&*&CF\\
bn\_asm.c& 727&bn\_sqr\_comba8&*&CF\\
bn\_asm.c& 691&bn\_sqr\_comba8&11.2 &CF\\
bn\_asm.c& 696&bn\_sqr\_comba8&*&CF\\
bn\_asm.c& 699&bn\_sqr\_comba8&*&CF\\
bn\_asm.c& 700&bn\_sqr\_comba8&*&CF\\
bn\_asm.c& 704&bn\_sqr\_comba8&*&CF\\
bn\_asm.c& 705&bn\_sqr\_comba8&*&CF\\
bn\_asm.c& 708&bn\_sqr\_comba8&*&CF\\
bn\_asm.c& 709&bn\_sqr\_comba8&*&CF\\
bn\_asm.c& 710&bn\_sqr\_comba8&*&CF\\
bn\_asm.c& 722&bn\_sqr\_comba8&10.2 &CF\\
bn\_asm.c& 705&bn\_sqr\_comba8&15.6 &CF\\
bn\_asm.c& 738&bn\_sqr\_comba8&5.4 &CF\\
bn\_asm.c& 746&bn\_sqr\_comba8&8.2 &CF\\
bn\_asm.c& 728&bn\_sqr\_comba8&10.3 &CF\\
bn\_asm.c& 715&bn\_sqr\_comba8&13.8 &CF\\
bn\_asm.c& 733&bn\_sqr\_comba8&12.2 &CF\\
bn\_asm.c& 710&bn\_sqr\_comba8&9.7 &CF\\
bn\_div.c& 303&BN\_div&1.3 &CF\\
bn\_div.c& 303&BN\_div&1.2 &CF\\
bn\_lib.c& 228&BN\_bn2bin&3.1 &CF\\
bn\_lib.c& 230&BN\_bn2bin&5.0 &DA\\
\hline
\end{tabular}
%}
\renewcommand{\baselinestretch}{1.0}\selectfont
\end{table}

\begin{table}[h!]
\centering\tiny\scriptsize
\caption{Leakages in RSA implemented by OpenSSL 1.0.2f}\label{tab:RSAOpenSSL1.0.2f}
%\resizebox{\columnwidth}{!}{
\begin{tabular}{lrlrr}
\hline
\textbf{File} & \textbf{Line No.} & \textbf{Function} & \textbf{\# Leaked Bits} & \textbf{Type} \\\hline
bn\_lib.c& 199&BN\_num\_bits&*&CF\\
bn\_lib.c& 200&BN\_num\_bits&15.6 &CF\\
bn\_lib.c& 201&BN\_num\_bits&16.2 &DA\\
bn\_lib.c& 673&BN\_ucmp&*&CF\\
bio\_asn1.c& 482&\_\_udivdi3&8.5 &CF\\
bn\_div.c& 381&BN\_div&*&CF\\
bn\_div.c& 456&BN\_div&*&CF\\
bn\_gcd.c& 279&BN\_mod\_inverse&1.0 &CF\\
bn\_gcd.c& 302&BN\_mod\_inverse&5.0 &CF\\
bn\_gcd.c& 324&BN\_mod\_inverse&6.6 &CF\\
bn\_add.c& 255&BN\_usub&*&CF\\
bn\_gcd.c& 305&BN\_mod\_inverse&12.8 &CF\\
bn\_gcd.c& 327&BN\_mod\_inverse&15.6 &CF\\
bn\_lib.c& 203&BN\_num\_bits&15.2 &DA\\
bn\_lib.c& 208&BN\_num\_bits&12.7 &CF\\
bn\_lib.c& 209&BN\_num\_bits&*&DA\\
bn\_lib.c& 212&BN\_num\_bits&*&DA\\
bn\_gcd.c& 515&BN\_mod\_inverse&*&CF\\
bn\_div.c& 381&BN\_div&2.7 &CF\\
bn\_div.c& 439&BN\_div&14.2 &CF\\
bn\_div.c& 385&BN\_div&11.4 &CF\\
bn\_div.c& 381&BN\_div&1.1 &CF\\
bn\_div.c& 381&BN\_div&1.0 &CF\\
bn\_div.c& 469&BN\_div&0.9 &CF\\
bn\_exp.c& 676&BN\_mod\_exp\_mont\_consttime&1.0 &CF\\
bn\_exp.c& 796&BN\_mod\_exp\_mont\_consttime&1.0 &CF\\
bn\_mont.c& 262&BN\_from\_montgomery\_word&*&DA\\
bn\_mont.c& 263&BN\_from\_montgomery\_word&*&DA\\
bn\_mont.c& 264&BN\_from\_montgomery\_word&*&DA\\
bn\_mont.c& 266&BN\_from\_montgomery\_word&*&DA\\
bn\_mont.c& 276&BN\_from\_montgomery\_word&*&DA\\
bn\_mont.c& 276&BN\_from\_montgomery\_word&0.2 &DA\\
bn\_mont.c& 276&BN\_from\_montgomery\_word&0.2 &DA\\
bn\_mont.c& 275&BN\_from\_montgomery\_word&0.1 &CF\\
bn\_mont.c& 282&BN\_from\_montgomery\_word&*&CF\\
bn\_asm.c& 787&bn\_sqr\_comba8&*&CF\\
bn\_asm.c& 646&bn\_mul\_comba8&*&CF\\
bn\_mont.c& 201&BN\_from\_montgomery\_word&0.0 &CF\\
\hline
\end{tabular}
%}
\renewcommand{\baselinestretch}{1.0}\selectfont
\end{table}

\begin{table}[h!]
\centering\tiny\scriptsize
\caption{Leakages in RSA implemented by OpenSSL 1.0.2k}\label{tab:RSAOpenSSL1.0.2k}
%\resizebox{\columnwidth}{!}{
\begin{tabular}{lrlrr}
\hline
\textbf{File} & \textbf{Line No.} & \textbf{Function} & \textbf{\# Leaked Bits} & \textbf{Type} \\\hline
bn\_lib.c& 199&BN\_num\_bits&*&CF\\
bn\_lib.c& 200&BN\_num\_bits&9.8 &CF\\
bn\_lib.c& 201&BN\_num\_bits&12.1 &DA\\
bn\_shift.c& 168&BN\_lshift&5.3 &CF\\
bn\_lib.c& 673&BN\_ucmp&*&CF\\
bio\_asn1.c& 484&\_\_udivdi3&6.4 &CF\\
bn\_div.c& 381&BN\_div&*&CF\\
bn\_div.c& 456&BN\_div&*&CF\\
bn\_gcd.c& 279&BN\_mod\_inverse&1.0 &CF\\
bn\_gcd.c& 302&BN\_mod\_inverse&8.4 &CF\\
bn\_gcd.c& 324&BN\_mod\_inverse&8.6 &CF\\
bn\_add.c& 255&BN\_usub&*&CF\\
bn\_gcd.c& 327&BN\_mod\_inverse&14.2 &CF\\
bn\_gcd.c& 305&BN\_mod\_inverse&13.8 &CF\\
bn\_lib.c& 203&BN\_num\_bits&14.2 &DA\\
bn\_lib.c& 208&BN\_num\_bits&13.5 &CF\\
bn\_lib.c& 209&BN\_num\_bits&*&DA\\
bn\_lib.c& 212&BN\_num\_bits&*&DA\\
bn\_gcd.c& 515&BN\_mod\_inverse&*&CF\\
bn\_div.c& 381&BN\_div&8.2 &CF\\
bn\_div.c& 439&BN\_div&*&CF\\
bn\_div.c& 385&BN\_div&3.9 &CF\\
bn\_div.c& 381&BN\_div&1.0 &CF\\
bn\_div.c& 381&BN\_div&0.8 &CF\\
bn\_div.c& 469&BN\_div&1.6 &CF\\
bn\_exp.c& 716&BN\_mod\_exp\_mont\_consttime&1.0 &CF\\
bn\_exp.c& 836&BN\_mod\_exp\_mont\_consttime&1.1 &CF\\
bn\_mont.c& 262&BN\_from\_montgomery\_word&*&DA\\
bn\_mont.c& 263&BN\_from\_montgomery\_word&*&DA\\
bn\_mont.c& 264&BN\_from\_montgomery\_word&*&DA\\
bn\_mont.c& 266&BN\_from\_montgomery\_word&*&DA\\
bn\_mont.c& 276&BN\_from\_montgomery\_word&*&DA\\
bn\_mont.c& 282&BN\_from\_montgomery\_word&*&CF\\
bn\_mont.c& 201&BN\_from\_montgomery\_word&0.0 &CF\\
bn\_asm.c& 646&bn\_mul\_comba8&*&CF\\
bn\_asm.c& 787&bn\_sqr\_comba8&*&CF\\
\hline
\end{tabular}
%}
\renewcommand{\baselinestretch}{1.0}\selectfont
\end{table}

\begin{table}[h!]
\centering\tiny\scriptsize
\caption{Leakages in RSA implemented by OpenSSL 1.1.0f}\label{tab:RSAOpenSSL1.1.0f}
%\resizebox{\columnwidth}{!}{
\begin{tabular}{l@{~~}rlr@{~~}r}
\hline
\textbf{File} & \textbf{Line No.} & \textbf{Function} & \textbf{\# Leaked Bits} & \textbf{Type} \\\hline
bn\_lib.c& 143&BN\_num\_bits\_word&*&CF\\
bn\_lib.c& 144&BN\_num\_bits\_word&*&CF\\
bn\_lib.c& 145&BN\_num\_bits\_word&17.2 &DA\\
bn\_lib.c& 1029&bn\_correct\_top&*&CF\\
bn\_lib.c& 639&BN\_ucmp&*&CF\\
ct\_b64.c& 164&\_\_udivdi3&5.9 &CF\\
bn\_div.c& 330&BN\_div&*&CF\\
bn\_gcd.c& 192&int\_bn\_mod\_inverse&1.0 &CF\\
bn\_gcd.c& 215&int\_bn\_mod\_inverse&7.9 &CF\\
bn\_gcd.c& 237&int\_bn\_mod\_inverse&8.2 &CF\\
bn\_gcd.c& 218&int\_bn\_mod\_inverse&14.9 &CF\\
bn\_gcd.c& 240&int\_bn\_mod\_inverse&9.2 &CF\\
bn\_lib.c& 147&BN\_num\_bits\_word&*&DA\\
bn\_lib.c& 152&BN\_num\_bits\_word&12.6 &CF\\
bn\_lib.c& 153&BN\_num\_bits\_word&*&DA\\
bn\_lib.c& 156&BN\_num\_bits\_word&*&DA\\
bn\_div.c& 384&BN\_div&17.2 &CF\\
bn\_div.c& 330&BN\_div&11.9 &CF\\
bn\_div.c& 334&BN\_div&3.8 &CF\\
bn\_exp.c& 622&BN\_mod\_exp\_mont\_consttime&1.0 &CF\\
bn\_exp.c& 741&BN\_mod\_exp\_mont\_consttime&1.0 &CF\\
bn\_mont.c& 138&BN\_from\_montgomery\_word&*&DA\\
bn\_mont.c& 139&BN\_from\_montgomery\_word&*&DA\\
bn\_mont.c& 140&BN\_from\_montgomery\_word&*&DA\\
bn\_mont.c& 142&BN\_from\_montgomery\_word&*&DA\\
bn\_mont.c& 152&BN\_from\_montgomery\_word&*&DA\\
bn\_asm.c& 733&bn\_sqr\_comba8&*&CF\\
bn\_asm.c& 592&bn\_mul\_comba8&*&CF\\
bn\_mont.c& 98&BN\_from\_montgomery\_word&0.0 &CF\\
bn\_div.c& 330&BN\_div&0.3 &CF\\
bn\_div.c& 330&BN\_div&0.3 &CF\\
\hline
\end{tabular}
%}
\renewcommand{\baselinestretch}{1.0}\selectfont
\end{table}

\begin{table}[h!]
\centering\tiny\scriptsize
\caption{Leakages in RSA implemented by OpenSSL 1.1.1}\label{tab:RSAOpenSSL1.1.1}
%\resizebox{\columnwidth}{!}{
\begin{tabular}{l@{~~}rlr@{~~}r}
\hline
\textbf{File} & \textbf{Line No.} & \textbf{Function} & \textbf{\# Leaked Bits} & \textbf{Type} \\\hline
%\input{figures/form/latex/RSA-openssl-1-1-1-data.tex}
rsa\_ossl.c& 649&rsa\_ossl\_mod\_exp&1.0 &CF\\
bn\_asm.c& 592&bn\_mul\_comba8&0.3 &CF\\
bn\_exp.c& 613&BN\_mod\_exp\_mont\_consttime&1.0 &CF\\
bn\_exp.c& 745&BN\_mod\_exp\_mont\_consttime&1.0 &CF\\
\hline
\end{tabular}
%}
\renewcommand{\baselinestretch}{1.0}\selectfont
\end{table}



\begin{table}[!ht]
\centering\tiny\scriptsize
\caption{Leakages in AES implemented by mbed TLS 2.5}\label{tab:AESmbed TLS2.5}
\renewcommand{\baselinestretch}{0.96}\selectfont
%\resizebox{\columnwidth}{!}{
\begin{tabular}{lrlrr}
\hline
\textbf{File} & \textbf{Line No.} & \textbf{Function} & \textbf{\# Leaked Bits} & \textbf{Type} \\\hline
aes.c& 536&mbedtls\_aes\_setkey\_enc&7.9 &DA\\
aes.c& 536&mbedtls\_aes\_setkey\_enc&7.6 &DA\\
aes.c& 536&mbedtls\_aes\_setkey\_enc&7.3 &DA\\
aes.c& 536&mbedtls\_aes\_setkey\_enc&7.5 &DA\\
aes.c& 729&mbedtls\_internal\_aes\_encrypt&3.9 &DA\\
aes.c& 729&mbedtls\_internal\_aes\_encrypt&8.2 &DA\\
aes.c& 729&mbedtls\_internal\_aes\_encrypt&4.2 &DA\\
aes.c& 729&mbedtls\_internal\_aes\_encrypt&8.0 &DA\\
aes.c& 729&mbedtls\_internal\_aes\_encrypt&4.3 &DA\\
aes.c& 729&mbedtls\_internal\_aes\_encrypt&4.1 &DA\\
aes.c& 729&mbedtls\_internal\_aes\_encrypt&8.6 &DA\\
aes.c& 729&mbedtls\_internal\_aes\_encrypt&8.1 &DA\\
aes.c& 729&mbedtls\_internal\_aes\_encrypt&7.6 &DA\\
aes.c& 729&mbedtls\_internal\_aes\_encrypt&3.7 &DA\\
aes.c& 729&mbedtls\_internal\_aes\_encrypt&8.4 &DA\\
aes.c& 729&mbedtls\_internal\_aes\_encrypt&7.4 &DA\\
aes.c& 729&mbedtls\_internal\_aes\_encrypt&8.0 &DA\\
aes.c& 729&mbedtls\_internal\_aes\_encrypt&4.2 &DA\\
aes.c& 729&mbedtls\_internal\_aes\_encrypt&3.9 &DA\\
aes.c& 729&mbedtls\_internal\_aes\_encrypt&4.1 &DA\\
aes.c& 730&mbedtls\_internal\_aes\_encrypt&3.9 &DA\\
aes.c& 730&mbedtls\_internal\_aes\_encrypt&7.6 &DA\\
aes.c& 730&mbedtls\_internal\_aes\_encrypt&4.0 &DA\\
aes.c& 730&mbedtls\_internal\_aes\_encrypt&7.6 &DA\\
aes.c& 730&mbedtls\_internal\_aes\_encrypt&4.1 &DA\\
aes.c& 730&mbedtls\_internal\_aes\_encrypt&3.7 &DA\\
aes.c& 730&mbedtls\_internal\_aes\_encrypt&7.6 &DA\\
aes.c& 730&mbedtls\_internal\_aes\_encrypt&7.9 &DA\\
aes.c& 730&mbedtls\_internal\_aes\_encrypt&8.1 &DA\\
aes.c& 730&mbedtls\_internal\_aes\_encrypt&8.2 &DA\\
aes.c& 730&mbedtls\_internal\_aes\_encrypt&7.2 &DA\\
aes.c& 730&mbedtls\_internal\_aes\_encrypt&3.9 &DA\\
aes.c& 730&mbedtls\_internal\_aes\_encrypt&7.6 &DA\\
aes.c& 730&mbedtls\_internal\_aes\_encrypt&3.8 &DA\\
aes.c& 730&mbedtls\_internal\_aes\_encrypt&4.1 &DA\\
aes.c& 730&mbedtls\_internal\_aes\_encrypt&4.1 &DA\\
aes.c& 733&mbedtls\_internal\_aes\_encrypt&4.0 &DA\\
aes.c& 733&mbedtls\_internal\_aes\_encrypt&4.0 &DA\\
aes.c& 733&mbedtls\_internal\_aes\_encrypt&4.3 &DA\\
aes.c& 733&mbedtls\_internal\_aes\_encrypt&4.0 &DA\\
aes.c& 733&mbedtls\_internal\_aes\_encrypt&3.9 &DA\\
aes.c& 733&mbedtls\_internal\_aes\_encrypt&4.0 &DA\\
aes.c& 733&mbedtls\_internal\_aes\_encrypt&4.2 &DA\\
aes.c& 733&mbedtls\_internal\_aes\_encrypt&4.0 &DA\\
aes.c& 733&mbedtls\_internal\_aes\_encrypt&3.9 &DA\\
aes.c& 733&mbedtls\_internal\_aes\_encrypt&4.1 &DA\\
aes.c& 733&mbedtls\_internal\_aes\_encrypt&4.2 &DA\\
aes.c& 733&mbedtls\_internal\_aes\_encrypt&4.1 &DA\\
aes.c& 733&mbedtls\_internal\_aes\_encrypt&4.0 &DA\\
aes.c& 733&mbedtls\_internal\_aes\_encrypt&3.6 &DA\\
aes.c& 733&mbedtls\_internal\_aes\_encrypt&4.0 &DA\\
aes.c& 733&mbedtls\_internal\_aes\_encrypt&4.0 &DA\\
aes.c& 735&mbedtls\_internal\_aes\_encrypt&1.9 &DA\\
aes.c& 735&mbedtls\_internal\_aes\_encrypt&2.0 &DA\\
aes.c& 735&mbedtls\_internal\_aes\_encrypt&2.0 &DA\\
aes.c& 735&mbedtls\_internal\_aes\_encrypt&2.1 &DA\\
aes.c& 741&mbedtls\_internal\_aes\_encrypt&2.1 &DA\\
aes.c& 741&mbedtls\_internal\_aes\_encrypt&2.0 &DA\\
aes.c& 747&mbedtls\_internal\_aes\_encrypt&1.9 &DA\\
aes.c& 741&mbedtls\_internal\_aes\_encrypt&2.0 &DA\\
aes.c& 753&mbedtls\_internal\_aes\_encrypt&2.1 &DA\\
aes.c& 741&mbedtls\_internal\_aes\_encrypt&1.8 &DA\\
aes.c& 747&mbedtls\_internal\_aes\_encrypt&2.2 &DA\\
aes.c& 747&mbedtls\_internal\_aes\_encrypt&2.0 &DA\\
aes.c& 753&mbedtls\_internal\_aes\_encrypt&2.0 &DA\\
aes.c& 747&mbedtls\_internal\_aes\_encrypt&1.9 &DA\\
aes.c& 753&mbedtls\_internal\_aes\_encrypt&1.9 &DA\\
aes.c& 753&mbedtls\_internal\_aes\_encrypt&1.9 &DA\\
\hline
\end{tabular}
%}
\renewcommand{\baselinestretch}{1.0}\selectfont
\end{table}

%% \begin{table}[!ht]
\centering\tiny\scriptsize
\caption{Leakages in AES implemented by mbed TLS 2.15.1}\label{tab:AESmbed TLS2.15.1}
%\resizebox{\columnwidth}{!}{
\begin{tabular}{lrlrr}
\hline
\textbf{File} & \textbf{Line No.} & \textbf{Function} & \textbf{\# Leaked Bits} & \textbf{Type} \\\hline
aes.c& 595&mbedtls\_aes\_setkey\_enc&9.1 &DA\\
aes.c& 595&mbedtls\_aes\_setkey\_enc&7.6 &DA\\
aes.c& 595&mbedtls\_aes\_setkey\_enc&7.7 &DA\\
aes.c& 595&mbedtls\_aes\_setkey\_enc&8.0 &DA\\
aes.c& 860&mbedtls\_internal\_aes\_encrypt&3.7 &DA\\
aes.c& 860&mbedtls\_internal\_aes\_encrypt&8.2 &DA\\
aes.c& 860&mbedtls\_internal\_aes\_encrypt&4.0 &DA\\
aes.c& 860&mbedtls\_internal\_aes\_encrypt&7.6 &DA\\
aes.c& 860&mbedtls\_internal\_aes\_encrypt&3.9 &DA\\
aes.c& 860&mbedtls\_internal\_aes\_encrypt&3.9 &DA\\
aes.c& 860&mbedtls\_internal\_aes\_encrypt&8.5 &DA\\
aes.c& 860&mbedtls\_internal\_aes\_encrypt&8.0 &DA\\
aes.c& 860&mbedtls\_internal\_aes\_encrypt&8.2 &DA\\
aes.c& 860&mbedtls\_internal\_aes\_encrypt&4.3 &DA\\
aes.c& 860&mbedtls\_internal\_aes\_encrypt&7.9 &DA\\
aes.c& 860&mbedtls\_internal\_aes\_encrypt&7.6 &DA\\
aes.c& 860&mbedtls\_internal\_aes\_encrypt&7.6 &DA\\
aes.c& 860&mbedtls\_internal\_aes\_encrypt&4.1 &DA\\
aes.c& 860&mbedtls\_internal\_aes\_encrypt&4.1 &DA\\
aes.c& 860&mbedtls\_internal\_aes\_encrypt&4.0 &DA\\
aes.c& 861&mbedtls\_internal\_aes\_encrypt&3.8 &DA\\
aes.c& 861&mbedtls\_internal\_aes\_encrypt&8.2 &DA\\
aes.c& 861&mbedtls\_internal\_aes\_encrypt&4.1 &DA\\
aes.c& 861&mbedtls\_internal\_aes\_encrypt&7.9 &DA\\
aes.c& 861&mbedtls\_internal\_aes\_encrypt&3.8 &DA\\
aes.c& 861&mbedtls\_internal\_aes\_encrypt&3.9 &DA\\
aes.c& 861&mbedtls\_internal\_aes\_encrypt&7.3 &DA\\
aes.c& 861&mbedtls\_internal\_aes\_encrypt&7.5 &DA\\
aes.c& 861&mbedtls\_internal\_aes\_encrypt&7.3 &DA\\
aes.c& 861&mbedtls\_internal\_aes\_encrypt&8.1 &DA\\
aes.c& 861&mbedtls\_internal\_aes\_encrypt&8.2 &DA\\
aes.c& 861&mbedtls\_internal\_aes\_encrypt&3.9 &DA\\
aes.c& 861&mbedtls\_internal\_aes\_encrypt&7.7 &DA\\
aes.c& 861&mbedtls\_internal\_aes\_encrypt&4.4 &DA\\
aes.c& 861&mbedtls\_internal\_aes\_encrypt&3.7 &DA\\
aes.c& 861&mbedtls\_internal\_aes\_encrypt&3.8 &DA\\
aes.c& 864&mbedtls\_internal\_aes\_encrypt&3.9 &DA\\
aes.c& 864&mbedtls\_internal\_aes\_encrypt&4.2 &DA\\
aes.c& 864&mbedtls\_internal\_aes\_encrypt&3.9 &DA\\
aes.c& 864&mbedtls\_internal\_aes\_encrypt&4.2 &DA\\
aes.c& 864&mbedtls\_internal\_aes\_encrypt&4.0 &DA\\
aes.c& 864&mbedtls\_internal\_aes\_encrypt&3.9 &DA\\
aes.c& 864&mbedtls\_internal\_aes\_encrypt&4.0 &DA\\
aes.c& 864&mbedtls\_internal\_aes\_encrypt&3.8 &DA\\
aes.c& 864&mbedtls\_internal\_aes\_encrypt&4.2 &DA\\
aes.c& 864&mbedtls\_internal\_aes\_encrypt&3.6 &DA\\
aes.c& 864&mbedtls\_internal\_aes\_encrypt&3.9 &DA\\
aes.c& 864&mbedtls\_internal\_aes\_encrypt&4.4 &DA\\
aes.c& 864&mbedtls\_internal\_aes\_encrypt&3.6 &DA\\
aes.c& 864&mbedtls\_internal\_aes\_encrypt&4.2 &DA\\
aes.c& 864&mbedtls\_internal\_aes\_encrypt&3.8 &DA\\
aes.c& 864&mbedtls\_internal\_aes\_encrypt&4.3 &DA\\
aes.c& 866&mbedtls\_internal\_aes\_encrypt&2.0 &DA\\
aes.c& 866&mbedtls\_internal\_aes\_encrypt&2.0 &DA\\
aes.c& 866&mbedtls\_internal\_aes\_encrypt&1.9 &DA\\
aes.c& 866&mbedtls\_internal\_aes\_encrypt&2.0 &DA\\
aes.c& 872&mbedtls\_internal\_aes\_encrypt&2.0 &DA\\
aes.c& 872&mbedtls\_internal\_aes\_encrypt&2.1 &DA\\
aes.c& 878&mbedtls\_internal\_aes\_encrypt&2.0 &DA\\
aes.c& 872&mbedtls\_internal\_aes\_encrypt&1.9 &DA\\
aes.c& 884&mbedtls\_internal\_aes\_encrypt&2.1 &DA\\
aes.c& 872&mbedtls\_internal\_aes\_encrypt&1.9 &DA\\
aes.c& 878&mbedtls\_internal\_aes\_encrypt&2.0 &DA\\
aes.c& 878&mbedtls\_internal\_aes\_encrypt&2.1 &DA\\
aes.c& 884&mbedtls\_internal\_aes\_encrypt&2.0 &DA\\
aes.c& 878&mbedtls\_internal\_aes\_encrypt&2.0 &DA\\
aes.c& 884&mbedtls\_internal\_aes\_encrypt&2.1 &DA\\
aes.c& 884&mbedtls\_internal\_aes\_encrypt&1.9 &DA\\
\hline
\end{tabular}
%}
\renewcommand{\baselinestretch}{1.0}\selectfont
\end{table}

%% \begin{table}[h!]
\centering\tiny\scriptsize
\caption{Leakages in AES implemented by OpenSSL 0.9.7}\label{tab:AESOpenSSL0.9.7}
%\resizebox{\columnwidth}{!}{
\begin{tabular}{lrlrr}
\hline
\textbf{File} & \textbf{Line No.} & \textbf{Function} & \textbf{\# Leaked Bits} & \textbf{Type} \\\hline
aes\_core.c& 662&AES\_set\_encrypt\_key&5.3 &DA\\
aes\_core.c& 662&AES\_set\_encrypt\_key&3.9 &DA\\
aes\_core.c& 662&AES\_set\_encrypt\_key&4.4 &DA\\
aes\_core.c& 662&AES\_set\_encrypt\_key&4.3 &DA\\
aes\_core.c& 662&AES\_set\_encrypt\_key&4.1 &DA\\
aes\_core.c& 662&AES\_set\_encrypt\_key&4.4 &DA\\
aes\_core.c& 662&AES\_set\_encrypt\_key&4.1 &DA\\
aes\_core.c& 662&AES\_set\_encrypt\_key&4.0 &DA\\
aes\_core.c& 662&AES\_set\_encrypt\_key&3.7 &DA\\
aes\_core.c& 662&AES\_set\_encrypt\_key&4.0 &DA\\
aes\_core.c& 662&AES\_set\_encrypt\_key&3.7 &DA\\
aes\_core.c& 662&AES\_set\_encrypt\_key&4.2 &DA\\
aes\_core.c& 662&AES\_set\_encrypt\_key&3.8 &DA\\
aes\_core.c& 662&AES\_set\_encrypt\_key&4.2 &DA\\
aes\_core.c& 662&AES\_set\_encrypt\_key&3.8 &DA\\
aes\_core.c& 662&AES\_set\_encrypt\_key&4.1 &DA\\
aes\_core.c& 662&AES\_set\_encrypt\_key&4.1 &DA\\
aes\_core.c& 662&AES\_set\_encrypt\_key&4.0 &DA\\
aes\_core.c& 662&AES\_set\_encrypt\_key&4.2 &DA\\
aes\_core.c& 662&AES\_set\_encrypt\_key&3.9 &DA\\
aes\_core.c& 662&AES\_set\_encrypt\_key&4.2 &DA\\
aes\_core.c& 662&AES\_set\_encrypt\_key&3.9 &DA\\
aes\_core.c& 662&AES\_set\_encrypt\_key&4.0 &DA\\
aes\_core.c& 662&AES\_set\_encrypt\_key&3.8 &DA\\
aes\_core.c& 662&AES\_set\_encrypt\_key&4.2 &DA\\
aes\_core.c& 662&AES\_set\_encrypt\_key&4.1 &DA\\
aes\_core.c& 662&AES\_set\_encrypt\_key&3.8 &DA\\
aes\_core.c& 662&AES\_set\_encrypt\_key&4.1 &DA\\
aes\_core.c& 662&AES\_set\_encrypt\_key&4.1 &DA\\
aes\_core.c& 662&AES\_set\_encrypt\_key&4.0 &DA\\
aes\_core.c& 662&AES\_set\_encrypt\_key&4.1 &DA\\
aes\_core.c& 662&AES\_set\_encrypt\_key&4.2 &DA\\
aes\_core.c& 880&AES\_encrypt&5.4 &DA\\
aes\_core.c& 880&AES\_encrypt&9.1 &DA\\
aes\_core.c& 880&AES\_encrypt&5.1 &DA\\
aes\_core.c& 880&AES\_encrypt&8.3 &DA\\
aes\_core.c& 886&AES\_encrypt&4.8 &DA\\
aes\_core.c& 886&AES\_encrypt&4.8 &DA\\
aes\_core.c& 886&AES\_encrypt&9.2 &DA\\
aes\_core.c& 886&AES\_encrypt&9.6 &DA\\
aes\_core.c& 892&AES\_encrypt&10.6 &DA\\
aes\_core.c& 892&AES\_encrypt&4.5 &DA\\
aes\_core.c& 892&AES\_encrypt&9.3 &DA\\
aes\_core.c& 892&AES\_encrypt&4.9 &DA\\
aes\_core.c& 898&AES\_encrypt&10.6 &DA\\
aes\_core.c& 898&AES\_encrypt&9.8 &DA\\
aes\_core.c& 898&AES\_encrypt&5.2 &DA\\
aes\_core.c& 898&AES\_encrypt&5.1 &DA\\
aes\_core.c& 910&AES\_encrypt&4.9 &DA\\
aes\_core.c& 910&AES\_encrypt&7.5 &DA\\
aes\_core.c& 910&AES\_encrypt&9.1 &DA\\
aes\_core.c& 910&AES\_encrypt&9.1 &DA\\
aes\_core.c& 916&AES\_encrypt&8.7 &DA\\
aes\_core.c& 916&AES\_encrypt&5.5 &DA\\
aes\_core.c& 916&AES\_encrypt&8.6 &DA\\
aes\_core.c& 916&AES\_encrypt&9.0 &DA\\
aes\_core.c& 922&AES\_encrypt&7.6 &DA\\
aes\_core.c& 922&AES\_encrypt&9.7 &DA\\
aes\_core.c& 922&AES\_encrypt&9.0 &DA\\
aes\_core.c& 922&AES\_encrypt&5.0 &DA\\
aes\_core.c& 928&AES\_encrypt&8.4 &DA\\
aes\_core.c& 928&AES\_encrypt&9.0 &DA\\
aes\_core.c& 928&AES\_encrypt&5.3 &DA\\
aes\_core.c& 928&AES\_encrypt&7.5 &DA\\
aes\_core.c& 940&AES\_encrypt&3.8 &DA\\
aes\_core.c& 940&AES\_encrypt&3.7 &DA\\
aes\_core.c& 940&AES\_encrypt&4.2 &DA\\
aes\_core.c& 947&AES\_encrypt&4.0 &DA\\
aes\_core.c& 947&AES\_encrypt&5.0 &DA\\
aes\_core.c& 947&AES\_encrypt&4.0 &DA\\
aes\_core.c& 954&AES\_encrypt&4.0 &DA\\
aes\_core.c& 954&AES\_encrypt&4.0 &DA\\
aes\_core.c& 954&AES\_encrypt&3.9 &DA\\
aes\_core.c& 961&AES\_encrypt&4.1 &DA\\
aes\_core.c& 961&AES\_encrypt&4.2 &DA\\
\hline
\end{tabular}
%}
\renewcommand{\baselinestretch}{1.0}\selectfont
\end{table}

%% \begin{table}[!ht]
\centering\tiny\scriptsize
\caption{Leakages in AES implemented by OpenSSL 1.0.2f}\label{tab:AESOpenSSL1.0.2f}
%\resizebox{\columnwidth}{!}{
\begin{tabular}{lrlrr}
\hline
\textbf{File} & \textbf{Line No.} & \textbf{Function} & \textbf{\# Leaked Bits} & \textbf{Type} \\\hline
aes\_core.c& 654&private\_AES\_set\_encrypt\_key&4.4 &DA\\
aes\_core.c& 654&private\_AES\_set\_encrypt\_key&3.9 &DA\\
aes\_core.c& 672&private\_AES\_set\_encrypt\_key&3.9 &DA\\
aes\_core.c& 673&private\_AES\_set\_encrypt\_key&4.1 &DA\\
aes\_core.c& 694&private\_AES\_set\_encrypt\_key&4.0 &DA\\
aes\_core.c& 694&private\_AES\_set\_encrypt\_key&3.9 &DA\\
aes\_core.c& 695&private\_AES\_set\_encrypt\_key&4.1 &DA\\
aes\_core.c& 695&private\_AES\_set\_encrypt\_key&3.9 &DA\\
aes\_core.c& 643&private\_AES\_set\_encrypt\_key&4.0 &DA\\
aes\_core.c& 644&private\_AES\_set\_encrypt\_key&4.0 &DA\\
aes\_core.c& 661&private\_AES\_set\_encrypt\_key&3.9 &DA\\
aes\_core.c& 661&private\_AES\_set\_encrypt\_key&4.0 &DA\\
aes\_core.c& 661&private\_AES\_set\_encrypt\_key&4.0 &DA\\
aes\_core.c& 657&private\_AES\_set\_encrypt\_key&4.4 &DA\\
aes\_core.c& 664&private\_AES\_set\_encrypt\_key&3.9 &DA\\
aes\_core.c& 661&private\_AES\_set\_encrypt\_key&4.0 &DA\\
aes\_core.c& 661&private\_AES\_set\_encrypt\_key&3.9 &DA\\
aes\_core.c& 661&private\_AES\_set\_encrypt\_key&4.0 &DA\\
aes\_core.c& 661&private\_AES\_set\_encrypt\_key&4.0 &DA\\
aes\_core.c& 665&private\_AES\_set\_encrypt\_key&4.0 &DA\\
aes\_core.c& 658&private\_AES\_set\_encrypt\_key&3.8 &DA\\
aes\_core.c& 659&private\_AES\_set\_encrypt\_key&4.1 &DA\\
aes\_core.c& 661&private\_AES\_set\_encrypt\_key&4.1 &DA\\
aes\_core.c& 665&private\_AES\_set\_encrypt\_key&3.9 &DA\\
aes\_core.c& 661&private\_AES\_set\_encrypt\_key&4.1 &DA\\
aes\_core.c& 661&private\_AES\_set\_encrypt\_key&4.1 &DA\\
aes\_core.c& 661&private\_AES\_set\_encrypt\_key&4.0 &DA\\
aes\_core.c& 661&private\_AES\_set\_encrypt\_key&4.1 &DA\\
aes\_core.c& 661&private\_AES\_set\_encrypt\_key&4.0 &DA\\
aes\_core.c& 658&private\_AES\_set\_encrypt\_key&3.8 &DA\\
aes\_core.c& 661&private\_AES\_set\_encrypt\_key&4.3 &DA\\
aes\_core.c& 661&private\_AES\_set\_encrypt\_key&3.8 &DA\\
aes\_core.c& 661&private\_AES\_set\_encrypt\_key&3.8 &DA\\
aes\_core.c& 658&private\_AES\_set\_encrypt\_key&3.9 &DA\\
aes\_core.c& 661&private\_AES\_set\_encrypt\_key&4.0 &DA\\
aes\_core.c& 661&private\_AES\_set\_encrypt\_key&4.3 &DA\\
aes\_core.c& 661&private\_AES\_set\_encrypt\_key&4.1 &DA\\
aes\_core.c& 658&private\_AES\_set\_encrypt\_key&3.9 &DA\\
aes\_core.c& 661&private\_AES\_set\_encrypt\_key&4.1 &DA\\
aes\_core.c& 663&private\_AES\_set\_encrypt\_key&4.1 &DA\\
aes\_core.c& 801&AES\_encrypt&6.2 &DA\\
aes\_core.c& 801&AES\_encrypt&5.7 &DA\\
aes\_core.c& 802&AES\_encrypt&10.3 &DA\\
aes\_core.c& 802&AES\_encrypt&10.6 &DA\\
aes\_core.c& 878&AES\_encrypt&6.1 &DA\\
aes\_core.c& 878&AES\_encrypt&11.0 &DA\\
aes\_core.c& 910&AES\_encrypt&5.1 &DA\\
aes\_core.c& 910&AES\_encrypt&10.0 &DA\\
aes\_core.c& 910&AES\_encrypt&11.4 &DA\\
aes\_core.c& 926&AES\_encrypt&12.0 &DA\\
aes\_core.c& 916&AES\_encrypt&6.1 &DA\\
aes\_core.c& 918&AES\_encrypt&5.0 &DA\\
aes\_core.c& 918&AES\_encrypt&11.7 &DA\\
aes\_core.c& 916&AES\_encrypt&10.7 &DA\\
aes\_core.c& 916&AES\_encrypt&6.5 &DA\\
aes\_core.c& 922&AES\_encrypt&6.3 &DA\\
aes\_core.c& 922&AES\_encrypt&4.6 &DA\\
aes\_core.c& 922&AES\_encrypt&6.3 &DA\\
aes\_core.c& 928&AES\_encrypt&10.1 &DA\\
aes\_core.c& 884&AES\_encrypt&10.2 &DA\\
aes\_core.c& 881&AES\_encrypt&5.6 &DA\\
aes\_core.c& 880&AES\_encrypt&11.1 &DA\\
aes\_core.c& 880&AES\_encrypt&5.7 &DA\\
aes\_core.c& 886&AES\_encrypt&10.0 &DA\\
aes\_core.c& 888&AES\_encrypt&11.0 &DA\\
aes\_core.c& 889&AES\_encrypt&9.6 &DA\\
aes\_core.c& 901&AES\_encrypt&6.4 &DA\\
aes\_core.c& 892&AES\_encrypt&6.2 &DA\\
aes\_core.c& 892&AES\_encrypt&6.2 &DA\\
aes\_core.c& 892&AES\_encrypt&10.2 &DA\\
aes\_core.c& 894&AES\_encrypt&4.6 &DA\\
aes\_core.c& 895&AES\_encrypt&10.3 &DA\\
aes\_core.c& 898&AES\_encrypt&4.3 &DA\\
aes\_core.c& 898&AES\_encrypt&4.0 &DA\\
aes\_core.c& 941&AES\_encrypt&3.9 &DA\\
aes\_core.c& 940&AES\_encrypt&3.8 &DA\\
aes\_core.c& 940&AES\_encrypt&3.9 &DA\\
aes\_core.c& 942&AES\_encrypt&4.2 &DA\\
aes\_core.c& 946&AES\_encrypt&6.3 &DA\\
aes\_core.c& 946&AES\_encrypt&4.3 &DA\\
aes\_core.c& 946&AES\_encrypt&4.2 &DA\\
aes\_core.c& 949&AES\_encrypt&3.9 &DA\\
aes\_core.c& 947&AES\_encrypt&3.9 &DA\\
aes\_core.c& 947&AES\_encrypt&3.9 &DA\\
aes\_core.c& 953&AES\_encrypt&4.5 &DA\\
aes\_core.c& 953&AES\_encrypt&3.9 &DA\\
aes\_core.c& 958&AES\_encrypt&3.6 &DA\\
aes\_core.c& 954&AES\_encrypt&4.3 &DA\\
\hline
\end{tabular}
%}
\renewcommand{\baselinestretch}{1.0}\selectfont
\end{table}

%% \begin{table}[!ht]
\centering\tiny\scriptsize
\caption{Leakages in AES implemented by OpenSSL 1.0.2k}\label{tab:AESOpenSSL1.0.2k}
%\resizebox{\columnwidth}{!}{
\begin{tabular}{lrlrr}
\hline
\textbf{File} & \textbf{Line No.} & \textbf{Function} & \textbf{\# Leaked Bits} & \textbf{Type} \\\hline
aes\_core.c& 654&private\_AES\_set\_encrypt\_key&4.1 &DA\\
aes\_core.c& 654&private\_AES\_set\_encrypt\_key&3.9 &DA\\
aes\_core.c& 672&private\_AES\_set\_encrypt\_key&3.9 &DA\\
aes\_core.c& 673&private\_AES\_set\_encrypt\_key&4.0 &DA\\
aes\_core.c& 694&private\_AES\_set\_encrypt\_key&4.1 &DA\\
aes\_core.c& 694&private\_AES\_set\_encrypt\_key&4.3 &DA\\
aes\_core.c& 695&private\_AES\_set\_encrypt\_key&3.9 &DA\\
aes\_core.c& 695&private\_AES\_set\_encrypt\_key&4.1 &DA\\
aes\_core.c& 643&private\_AES\_set\_encrypt\_key&3.9 &DA\\
aes\_core.c& 644&private\_AES\_set\_encrypt\_key&4.1 &DA\\
aes\_core.c& 661&private\_AES\_set\_encrypt\_key&4.0 &DA\\
aes\_core.c& 661&private\_AES\_set\_encrypt\_key&4.1 &DA\\
aes\_core.c& 661&private\_AES\_set\_encrypt\_key&4.1 &DA\\
aes\_core.c& 657&private\_AES\_set\_encrypt\_key&4.1 &DA\\
aes\_core.c& 664&private\_AES\_set\_encrypt\_key&4.0 &DA\\
aes\_core.c& 661&private\_AES\_set\_encrypt\_key&4.0 &DA\\
aes\_core.c& 661&private\_AES\_set\_encrypt\_key&3.9 &DA\\
aes\_core.c& 661&private\_AES\_set\_encrypt\_key&3.9 &DA\\
aes\_core.c& 661&private\_AES\_set\_encrypt\_key&4.0 &DA\\
aes\_core.c& 665&private\_AES\_set\_encrypt\_key&3.5 &DA\\
aes\_core.c& 658&private\_AES\_set\_encrypt\_key&3.9 &DA\\
aes\_core.c& 659&private\_AES\_set\_encrypt\_key&4.0 &DA\\
aes\_core.c& 661&private\_AES\_set\_encrypt\_key&4.0 &DA\\
aes\_core.c& 665&private\_AES\_set\_encrypt\_key&3.9 &DA\\
aes\_core.c& 661&private\_AES\_set\_encrypt\_key&3.8 &DA\\
aes\_core.c& 661&private\_AES\_set\_encrypt\_key&4.3 &DA\\
aes\_core.c& 661&private\_AES\_set\_encrypt\_key&3.9 &DA\\
aes\_core.c& 661&private\_AES\_set\_encrypt\_key&3.9 &DA\\
aes\_core.c& 661&private\_AES\_set\_encrypt\_key&4.2 &DA\\
aes\_core.c& 658&private\_AES\_set\_encrypt\_key&4.1 &DA\\
aes\_core.c& 661&private\_AES\_set\_encrypt\_key&4.2 &DA\\
aes\_core.c& 661&private\_AES\_set\_encrypt\_key&4.0 &DA\\
aes\_core.c& 661&private\_AES\_set\_encrypt\_key&3.8 &DA\\
aes\_core.c& 658&private\_AES\_set\_encrypt\_key&4.0 &DA\\
aes\_core.c& 661&private\_AES\_set\_encrypt\_key&4.0 &DA\\
aes\_core.c& 661&private\_AES\_set\_encrypt\_key&4.0 &DA\\
aes\_core.c& 661&private\_AES\_set\_encrypt\_key&3.9 &DA\\
aes\_core.c& 658&private\_AES\_set\_encrypt\_key&3.7 &DA\\
aes\_core.c& 661&private\_AES\_set\_encrypt\_key&4.4 &DA\\
aes\_core.c& 663&private\_AES\_set\_encrypt\_key&4.0 &DA\\
aes\_core.c& 801&AES\_encrypt&6.1 &DA\\
aes\_core.c& 801&AES\_encrypt&5.5 &DA\\
aes\_core.c& 802&AES\_encrypt&11.4 &DA\\
aes\_core.c& 802&AES\_encrypt&11.6 &DA\\
aes\_core.c& 878&AES\_encrypt&5.8 &DA\\
aes\_core.c& 878&AES\_encrypt&11.0 &DA\\
aes\_core.c& 910&AES\_encrypt&5.0 &DA\\
aes\_core.c& 910&AES\_encrypt&9.2 &DA\\
aes\_core.c& 910&AES\_encrypt&11.4 &DA\\
aes\_core.c& 926&AES\_encrypt&12.5 &DA\\
aes\_core.c& 916&AES\_encrypt&6.4 &DA\\
aes\_core.c& 918&AES\_encrypt&5.6 &DA\\
aes\_core.c& 918&AES\_encrypt&9.8 &DA\\
aes\_core.c& 916&AES\_encrypt&10.1 &DA\\
aes\_core.c& 916&AES\_encrypt&6.0 &DA\\
aes\_core.c& 922&AES\_encrypt&6.2 &DA\\
aes\_core.c& 922&AES\_encrypt&4.8 &DA\\
aes\_core.c& 922&AES\_encrypt&5.9 &DA\\
aes\_core.c& 928&AES\_encrypt&9.8 &DA\\
aes\_core.c& 884&AES\_encrypt&9.4 &DA\\
aes\_core.c& 881&AES\_encrypt&5.8 &DA\\
aes\_core.c& 880&AES\_encrypt&11.0 &DA\\
aes\_core.c& 880&AES\_encrypt&5.3 &DA\\
aes\_core.c& 886&AES\_encrypt&11.4 &DA\\
aes\_core.c& 888&AES\_encrypt&10.6 &DA\\
aes\_core.c& 889&AES\_encrypt&10.2 &DA\\
aes\_core.c& 901&AES\_encrypt&5.9 &DA\\
aes\_core.c& 892&AES\_encrypt&6.3 &DA\\
aes\_core.c& 892&AES\_encrypt&5.5 &DA\\
aes\_core.c& 892&AES\_encrypt&9.5 &DA\\
aes\_core.c& 894&AES\_encrypt&4.1 &DA\\
aes\_core.c& 895&AES\_encrypt&9.6 &DA\\
aes\_core.c& 898&AES\_encrypt&4.0 &DA\\
aes\_core.c& 898&AES\_encrypt&4.4 &DA\\
aes\_core.c& 941&AES\_encrypt&3.8 &DA\\
aes\_core.c& 940&AES\_encrypt&4.2 &DA\\
aes\_core.c& 940&AES\_encrypt&3.9 &DA\\
aes\_core.c& 942&AES\_encrypt&3.8 &DA\\
aes\_core.c& 946&AES\_encrypt&6.6 &DA\\
aes\_core.c& 946&AES\_encrypt&3.9 &DA\\
aes\_core.c& 946&AES\_encrypt&4.2 &DA\\
aes\_core.c& 949&AES\_encrypt&4.3 &DA\\
aes\_core.c& 947&AES\_encrypt&3.9 &DA\\
aes\_core.c& 947&AES\_encrypt&3.9 &DA\\
aes\_core.c& 953&AES\_encrypt&4.2 &DA\\
aes\_core.c& 953&AES\_encrypt&3.9 &DA\\
aes\_core.c& 958&AES\_encrypt&4.2 &DA\\
aes\_core.c& 954&AES\_encrypt&3.9 &DA\\
\hline
\end{tabular}
%}
\renewcommand{\baselinestretch}{1.0}\selectfont
\end{table}

%%\begin{table}[h!]
\centering\tiny\scriptsize
\renewcommand{\baselinestretch}{0.96}\selectfont
\caption{Leakages in AES implemented by OpenSSL 1.1.0f}\label{tab:AESOpenSSL1.1.0f}
%\resizebox{\columnwidth}{!}{
\begin{tabular}{lrlrr}
\hline
\textbf{File} & \textbf{Line No.} & \textbf{Function} & \textbf{\# Leaked Bits} & \textbf{Type} \\\hline
aes\_core.c& 676&AES\_set\_encrypt\_key&4.1 &DA\\
aes\_core.c& 676&AES\_set\_encrypt\_key&5.6 &DA\\
aes\_core.c& 676&AES\_set\_encrypt\_key&4.2 &DA\\
aes\_core.c& 677&AES\_set\_encrypt\_key&3.9 &DA\\
aes\_core.c& 698&AES\_set\_encrypt\_key&4.0 &DA\\
aes\_core.c& 698&AES\_set\_encrypt\_key&4.3 &DA\\
aes\_core.c& 699&AES\_set\_encrypt\_key&3.7 &DA\\
aes\_core.c& 700&AES\_set\_encrypt\_key&4.2 &DA\\
aes\_core.c& 648&AES\_set\_encrypt\_key&4.1 &DA\\
aes\_core.c& 664&AES\_set\_encrypt\_key&3.9 &DA\\
aes\_core.c& 665&AES\_set\_encrypt\_key&4.0 &DA\\
aes\_core.c& 662&AES\_set\_encrypt\_key&4.2 &DA\\
aes\_core.c& 665&AES\_set\_encrypt\_key&3.9 &DA\\
aes\_core.c& 665&AES\_set\_encrypt\_key&4.1 &DA\\
aes\_core.c& 668&AES\_set\_encrypt\_key&3.8 &DA\\
aes\_core.c& 665&AES\_set\_encrypt\_key&4.0 &DA\\
aes\_core.c& 665&AES\_set\_encrypt\_key&4.1 &DA\\
aes\_core.c& 665&AES\_set\_encrypt\_key&3.8 &DA\\
aes\_core.c& 665&AES\_set\_encrypt\_key&3.9 &DA\\
aes\_core.c& 665&AES\_set\_encrypt\_key&3.9 &DA\\
aes\_core.c& 665&AES\_set\_encrypt\_key&4.0 &DA\\
aes\_core.c& 665&AES\_set\_encrypt\_key&3.8 &DA\\
aes\_core.c& 665&AES\_set\_encrypt\_key&4.2 &DA\\
aes\_core.c& 668&AES\_set\_encrypt\_key&4.2 &DA\\
aes\_core.c& 665&AES\_set\_encrypt\_key&3.9 &DA\\
aes\_core.c& 662&AES\_set\_encrypt\_key&4.0 &DA\\
aes\_core.c& 665&AES\_set\_encrypt\_key&3.8 &DA\\
aes\_core.c& 661&AES\_set\_encrypt\_key&4.4 &DA\\
aes\_core.c& 665&AES\_set\_encrypt\_key&3.8 &DA\\
aes\_core.c& 662&AES\_set\_encrypt\_key&4.0 &DA\\
aes\_core.c& 665&AES\_set\_encrypt\_key&4.0 &DA\\
aes\_core.c& 665&AES\_set\_encrypt\_key&3.9 &DA\\
aes\_core.c& 665&AES\_set\_encrypt\_key&4.1 &DA\\
aes\_core.c& 665&AES\_set\_encrypt\_key&3.8 &DA\\
aes\_core.c& 665&AES\_set\_encrypt\_key&3.8 &DA\\
aes\_core.c& 665&AES\_set\_encrypt\_key&4.2 &DA\\
aes\_core.c& 665&AES\_set\_encrypt\_key&4.3 &DA\\
aes\_core.c& 662&AES\_set\_encrypt\_key&4.0 &DA\\
aes\_core.c& 665&AES\_set\_encrypt\_key&3.8 &DA\\
aes\_core.c& 661&AES\_set\_encrypt\_key&4.0 &DA\\
aes\_core.c& 805&AES\_encrypt&6.2 &DA\\
aes\_core.c& 805&AES\_encrypt&6.5 &DA\\
aes\_core.c& 806&AES\_encrypt&11.6 &DA\\
aes\_core.c& 806&AES\_encrypt&5.6 &DA\\
aes\_core.c& 882&AES\_encrypt&11.6 &DA\\
aes\_core.c& 882&AES\_encrypt&11.4 &DA\\
aes\_core.c& 797&AES\_encrypt&7.3 &DA\\
aes\_core.c& 797&AES\_encrypt&11.3 &DA\\
aes\_core.c& 916&AES\_encrypt&5.4 &DA\\
aes\_core.c& 917&AES\_encrypt&12.6 &DA\\
aes\_core.c& 924&AES\_encrypt&10.9 &DA\\
aes\_core.c& 914&AES\_encrypt&11.1 &DA\\
aes\_core.c& 920&AES\_encrypt&7.5 &DA\\
aes\_core.c& 920&AES\_encrypt&12.6 &DA\\
aes\_core.c& 920&AES\_encrypt&6.7 &DA\\
aes\_core.c& 920&AES\_encrypt&6.9 &DA\\
aes\_core.c& 929&AES\_encrypt&6.6 &DA\\
aes\_core.c& 932&AES\_encrypt&6.5 &DA\\
aes\_core.c& 934&AES\_encrypt&10.7 &DA\\
aes\_core.c& 884&AES\_encrypt&12.5 &DA\\
aes\_core.c& 885&AES\_encrypt&7.0 &DA\\
aes\_core.c& 886&AES\_encrypt&9.0 &DA\\
aes\_core.c& 891&AES\_encrypt&5.9 &DA\\
aes\_core.c& 890&AES\_encrypt&9.2 &DA\\
aes\_core.c& 897&AES\_encrypt&7.5 &DA\\
aes\_core.c& 897&AES\_encrypt&11.1 &DA\\
aes\_core.c& 896&AES\_encrypt&9.8 &DA\\
aes\_core.c& 898&AES\_encrypt&5.2 &DA\\
aes\_core.c& 896&AES\_encrypt&10.3 &DA\\
aes\_core.c& 899&AES\_encrypt&5.8 &DA\\
aes\_core.c& 902&AES\_encrypt&7.3 &DA\\
aes\_core.c& 902&AES\_encrypt&10.5 &DA\\
aes\_core.c& 910&AES\_encrypt&3.9 &DA\\
aes\_core.c& 910&AES\_encrypt&4.2 &DA\\
aes\_core.c& 944&AES\_encrypt&3.8 &DA\\
aes\_core.c& 944&AES\_encrypt&4.4 &DA\\
aes\_core.c& 944&AES\_encrypt&4.4 &DA\\
aes\_core.c& 944&AES\_encrypt&3.8 &DA\\
aes\_core.c& 950&AES\_encrypt&4.1 &DA\\
aes\_core.c& 952&AES\_encrypt&4.7 &DA\\
aes\_core.c& 951&AES\_encrypt&4.0 &DA\\
aes\_core.c& 951&AES\_encrypt&4.1 &DA\\
aes\_core.c& 961&AES\_encrypt&3.9 &DA\\
aes\_core.c& 957&AES\_encrypt&3.9 &DA\\
aes\_core.c& 957&AES\_encrypt&3.9 &DA\\
aes\_core.c& 958&AES\_encrypt&5.4 &DA\\
aes\_core.c& 958&AES\_encrypt&3.9 &DA\\
aes\_core.c& 960&AES\_encrypt&3.9 &DA\\
\hline
\end{tabular}
%}
\renewcommand{\baselinestretch}{1.0}\selectfont
\end{table}

%%\begin{table}[h!]
\centering\tiny\scriptsize
\renewcommand{\baselinestretch}{0.96}\selectfont
\caption{Leakages in AES implemented by OpenSSL 1.1.1}\label{tab:AESOpenSSL1.1.1}
%\resizebox{\columnwidth}{!}{
\begin{tabular}{lrlrr}
\hline
\textbf{File} & \textbf{Line No.} & \textbf{Function} & \textbf{\# Leaked Bits} & \textbf{Type} \\\hline
aes\_core.c& 665&AES\_set\_encrypt\_key&4.2 &DA\\
aes\_core.c& 665&AES\_set\_encrypt\_key&3.9 &DA\\
aes\_core.c& 665&AES\_set\_encrypt\_key&3.9 &DA\\
aes\_core.c& 665&AES\_set\_encrypt\_key&4.0 &DA\\
aes\_core.c& 665&AES\_set\_encrypt\_key&4.2 &DA\\
aes\_core.c& 665&AES\_set\_encrypt\_key&3.9 &DA\\
aes\_core.c& 665&AES\_set\_encrypt\_key&3.7 &DA\\
aes\_core.c& 665&AES\_set\_encrypt\_key&4.1 &DA\\
aes\_core.c& 665&AES\_set\_encrypt\_key&4.1 &DA\\
aes\_core.c& 665&AES\_set\_encrypt\_key&3.7 &DA\\
aes\_core.c& 665&AES\_set\_encrypt\_key&4.1 &DA\\
aes\_core.c& 665&AES\_set\_encrypt\_key&4.3 &DA\\
aes\_core.c& 665&AES\_set\_encrypt\_key&3.8 &DA\\
aes\_core.c& 665&AES\_set\_encrypt\_key&4.1 &DA\\
aes\_core.c& 665&AES\_set\_encrypt\_key&3.8 &DA\\
aes\_core.c& 665&AES\_set\_encrypt\_key&4.2 &DA\\
aes\_core.c& 665&AES\_set\_encrypt\_key&3.8 &DA\\
aes\_core.c& 665&AES\_set\_encrypt\_key&4.1 &DA\\
aes\_core.c& 665&AES\_set\_encrypt\_key&4.1 &DA\\
aes\_core.c& 665&AES\_set\_encrypt\_key&4.1 &DA\\
aes\_core.c& 665&AES\_set\_encrypt\_key&4.2 &DA\\
aes\_core.c& 665&AES\_set\_encrypt\_key&4.2 &DA\\
aes\_core.c& 665&AES\_set\_encrypt\_key&4.0 &DA\\
aes\_core.c& 665&AES\_set\_encrypt\_key&4.4 &DA\\
aes\_core.c& 665&AES\_set\_encrypt\_key&4.1 &DA\\
aes\_core.c& 665&AES\_set\_encrypt\_key&4.2 &DA\\
aes\_core.c& 665&AES\_set\_encrypt\_key&4.1 &DA\\
aes\_core.c& 665&AES\_set\_encrypt\_key&3.9 &DA\\
aes\_core.c& 665&AES\_set\_encrypt\_key&4.2 &DA\\
aes\_core.c& 665&AES\_set\_encrypt\_key&3.9 &DA\\
aes\_core.c& 665&AES\_set\_encrypt\_key&3.7 &DA\\
aes\_core.c& 665&AES\_set\_encrypt\_key&4.0 &DA\\
aes\_core.c& 665&AES\_set\_encrypt\_key&3.9 &DA\\
aes\_core.c& 665&AES\_set\_encrypt\_key&4.0 &DA\\
aes\_core.c& 665&AES\_set\_encrypt\_key&4.1 &DA\\
aes\_core.c& 665&AES\_set\_encrypt\_key&3.9 &DA\\
aes\_core.c& 665&AES\_set\_encrypt\_key&4.3 &DA\\
aes\_core.c& 665&AES\_set\_encrypt\_key&3.9 &DA\\
aes\_core.c& 665&AES\_set\_encrypt\_key&4.0 &DA\\
aes\_core.c& 665&AES\_set\_encrypt\_key&4.1 &DA\\
aes\_core.c& 884&AES\_encrypt&4.2 &DA\\
aes\_core.c& 884&AES\_encrypt&8.1 &DA\\
aes\_core.c& 884&AES\_encrypt&4.2 &DA\\
aes\_core.c& 884&AES\_encrypt&8.1 &DA\\
aes\_core.c& 890&AES\_encrypt&3.8 &DA\\
aes\_core.c& 890&AES\_encrypt&4.0 &DA\\
aes\_core.c& 890&AES\_encrypt&8.1 &DA\\
aes\_core.c& 890&AES\_encrypt&7.7 &DA\\
aes\_core.c& 896&AES\_encrypt&7.9 &DA\\
aes\_core.c& 896&AES\_encrypt&4.0 &DA\\
aes\_core.c& 896&AES\_encrypt&7.3 &DA\\
aes\_core.c& 896&AES\_encrypt&4.3 &DA\\
aes\_core.c& 902&AES\_encrypt&7.3 &DA\\
aes\_core.c& 902&AES\_encrypt&7.9 &DA\\
aes\_core.c& 902&AES\_encrypt&4.1 &DA\\
aes\_core.c& 902&AES\_encrypt&3.9 &DA\\
aes\_core.c& 914&AES\_encrypt&4.3 &DA\\
aes\_core.c& 914&AES\_encrypt&8.1 &DA\\
aes\_core.c& 914&AES\_encrypt&4.2 &DA\\
aes\_core.c& 914&AES\_encrypt&7.3 &DA\\
aes\_core.c& 920&AES\_encrypt&4.1 &DA\\
aes\_core.c& 920&AES\_encrypt&4.0 &DA\\
aes\_core.c& 920&AES\_encrypt&8.2 &DA\\
aes\_core.c& 920&AES\_encrypt&7.6 &DA\\
aes\_core.c& 926&AES\_encrypt&8.2 &DA\\
aes\_core.c& 926&AES\_encrypt&3.9 &DA\\
aes\_core.c& 926&AES\_encrypt&8.2 &DA\\
aes\_core.c& 926&AES\_encrypt&3.9 &DA\\
aes\_core.c& 932&AES\_encrypt&8.1 &DA\\
aes\_core.c& 932&AES\_encrypt&7.6 &DA\\
aes\_core.c& 932&AES\_encrypt&4.0 &DA\\
aes\_core.c& 932&AES\_encrypt&4.0 &DA\\
aes\_core.c& 944&AES\_encrypt&3.9 &DA\\
aes\_core.c& 944&AES\_encrypt&4.3 &DA\\
aes\_core.c& 944&AES\_encrypt&3.9 &DA\\
aes\_core.c& 944&AES\_encrypt&4.0 &DA\\
aes\_core.c& 951&AES\_encrypt&3.8 &DA\\
aes\_core.c& 951&AES\_encrypt&3.9 &DA\\
aes\_core.c& 951&AES\_encrypt&3.9 &DA\\
aes\_core.c& 951&AES\_encrypt&3.9 &DA\\
aes\_core.c& 958&AES\_encrypt&4.2 &DA\\
aes\_core.c& 958&AES\_encrypt&3.7 &DA\\
aes\_core.c& 958&AES\_encrypt&3.8 &DA\\
aes\_core.c& 958&AES\_encrypt&4.0 &DA\\
aes\_core.c& 965&AES\_encrypt&4.0 &DA\\
aes\_core.c& 965&AES\_encrypt&4.0 &DA\\
aes\_core.c& 965&AES\_encrypt&3.7 &DA\\
aes\_core.c& 965&AES\_encrypt&3.7 &DA\\
\hline
\end{tabular}
%}
\renewcommand{\baselinestretch}{1.0}\selectfont
\end{table}

