\documentclass[10pt,conference]{IEEEtran}

\usepackage{adjustbox}
\usepackage{booktabs}

\usepackage{listings}
\usepackage{epsfig}
\usepackage{url}
\usepackage{cite}
\usepackage{fancybox}
\usepackage{amsmath}
\usepackage{amssymb}
\usepackage{amsfonts}
\usepackage{amsthm}
\usepackage{tikz}
\usepackage{multirow}
\usepackage{balance}
\usepackage{graphicx}
\usepackage{pdfpages}
\usepackage{subfig}
\usepackage[hidelinks]{hyperref}
\usepackage{xcolor}
\usepackage[ruled,linesnumbered]{algorithm2e}
\usepackage[para,online,flushleft]{threeparttable}
\usepackage{xspace}
\usepackage[scaled]{beramono}
\usepackage[T1]{fontenc}

\usepackage[english]{babel} % handle hyphenation

\usepackage{amssymb}
\usepackage{ulem}
\normalem

\lstset{
%numbers=left,
%frame=single,
%language=C,
basicstyle=\fontfamily{fvm}\scriptsize,
showlines=true
xleftmargin=.1\textwidth, xrightmargin=.1\textwidth,
}

\newcommand{\tool}{\textsf{Abacus}}
\renewcommand{\baselinestretch}{0.97}

\begin{document}

\title{\tool{}: A Tool for Precise Side-channel Analysis}
\author{\IEEEauthorblockN{
Qinkun Bao\IEEEauthorrefmark{1},
Zihao Wang\IEEEauthorrefmark{1},
James R. Larus\IEEEauthorrefmark{2}, and
Dinghao Wu\IEEEauthorrefmark{1}}
\IEEEauthorblockA{\IEEEauthorrefmark{1}The Pennsylvania State University}
\IEEEauthorblockA{\IEEEauthorrefmark{2}EPFL}
}


\maketitle

\begin{abstract}
Side-channel vulnerabilities can leak sensitive information unconsciously. 
In this paper, we introduce the usage of Abacus. Abacus is a tool that can analyze secret-dependent control-flow and secret-dependent data-access leakages in binary programs. Unlike previous tools that can only identify leakages, it can also estimate the amount of leaked information for each leakage site. A severe vulnerability usually leaks more information, allowing developers to triage the patching effort for side channel vulnerabilities. This paper is to help users make use of Abacus and reproduce our previous
results. Abacus is available at \url{https://github.com/s3team/Abacus}.
\end{abstract}

\IEEEpeerreviewmaketitle
\pagenumbering{arabic}
\pagestyle{plain}
\section{DESCRIPTION}


Abacus~\cite{bao2021abacus} is an address-based side-channel vulnerability analysis tool. Different from previous tools~\cite{203878,236338,182946}, it can also give a precise estimation of the amount of the leaked information for each leakage site.

We quantify the amount of leaked information for each leakage site based on the search space. For example, if the length of the key is 128 bits, an attacker needs to brute force $2^{128}$ possible keys without any domain knowledge. However, suppose the attacker observes some information and can reduce the size of the search space to $2^{120}$. Then we can conclude 8 bits of the information are leaked.

\tool{} takes a binary executable as the input. It uses dynamic binary instrumentation tools to collect execution traces. After that, \tool{} analyzes those traces and produces the vulnerability report. While \tool{} can work on the stripped binary executable, \tool{} can also read the symbol and debugging information to give a more fine-grained (e.g., line numbers) report. Table~\ref{tab:DESOpenSSL1.1.1} shows an example of the report. If you are interested in how \tool{} works, please refer to the technical paper~\cite{bao2021abacus}.

\begin{table}[h!]
\centering\tiny\scriptsize
\caption{Leakages in DES (OpenSSL 1.1.1)}\label{tab:DESOpenSSL1.1.1}
%\resizebox{\columnwidth}{!}{
\begin{tabular}{lrlrr}
\hline
\textbf{File} & \textbf{Line No.} & \textbf{Function} & \textbf{\# Leaked Bits} & \textbf{Type} \\\hline
set\_key.c& 350&DES\_set\_key\_unchecked&5.8 &DA\\
set\_key.c& 350&DES\_set\_key\_unchecked&6.6 &DA\\
set\_key.c& 350&DES\_set\_key\_unchecked&7.5 &DA\\
set\_key.c& 350&DES\_set\_key\_unchecked&6.4 &DA\\
set\_key.c& 355&DES\_set\_key\_unchecked&1.9 &DA\\
set\_key.c& 355&DES\_set\_key\_unchecked&3.1 &DA\\
\hline
\end{tabular}
%}
\renewcommand{\baselinestretch}{1.0}\selectfont
\end{table}

\section{Requirements}
We have tested \tool{} on both macOS and Ubuntu. You can refer to the continuous integration scripts to build Abacus on your operating system. However, we strongly recommend you to build \tool{} inside a container to avoid any dependency problems. To simplify the illustration, we only include the instructions of installations within the docker in this paper.
\begin{itemize}
\item Supported OS: Ubuntu 18.04
\item Memory: 32 GB (If you want to run experiments concurrently, update the size of RAM
     accordingly. Otherwise the program may be terminated by the system.)
\end{itemize}

\section{Installation}

\tool{} can be built within a docker, simply run the following command:

\begin{lstlisting}[language=bash]
$ git clone https://github.com/s3team/Abacus.git
$ cd Abacus
$ ./docker.sh
\end{lstlisting}

The ``docker.sh'' script creates a docker image automatically and enters the container that includes all dependencies. After that, run the following command to build \tool{}:

\begin{lstlisting}[language=bash]
$ ./build.sh
\end{lstlisting}
\section{Run the hello world example}
In this section, we walk you through the steps to test a simple function with \tool{}.

\begin{figure}[h]
\begin{lstlisting}[xleftmargin=.07\textwidth, xrightmargin=.07\textwidth,numbers=left, frame=single, language=C]
void is_odd(uint16_t secret) {
  int res = secret % 2;        
  if (res) {             
    printf("Odd Number\n");
  } else {
    printf("Even Number\n");
  }
}
\end{lstlisting}
\caption{A simple example}
\label{fig:example0}
\end{figure}

As shown in Figure~\ref{fig:example0}, the function takes a 32-bit integer as the secret input and checks the last digit of the integer. An attacker can know the last bit of the input integer by observing which branch is actually executed. So in the above function, we think the code has one secret-dependent control-flow vulnerability, and it can leak one bit of the secret information.

\subsection{Marking secret data as symbolic}
In order to test this function with \tool{}, we need to mark the variable that representing the secret data as a symbolic variable. We use the function \textsf{abacus\_make\_symbolic}. The function takes three arguments: the type of the symbol, the address of the secret, and the length of the secret input. In the below example, the secret input is the variable \textsf{secret}, and its length is two bytes. We add the \textsf{main} function in Figure~\ref{fig:example1} and compile the source code into an executable.

\begin{figure}[h]
\begin{lstlisting}[firstnumber=10, xleftmargin=.07\textwidth, xrightmargin=.07\textwidth, numbers=left, frame=single, language=C]
int main() {
  uint16_t secret = 6;
  char *type = "1";
  abacus_make_symbolic(type, &secret, 2); 
  is_odd(secret);
  return 0;
}
\end{lstlisting}
\caption{A simple function that marks a variable \textsf{secret} as symbolic}
\label{fig:example1}
\end{figure}


\subsection{Build the example}

Abacus analyzes vulnerabilities on the binary executable. Here we build it into  a 32-bit ELF executable. Note that while Abacus can work on stripped binaries without the source code. However, in the example, we use debug information to get a more detailed result (e.g., the line number in the source code).
\begin{lstlisting}[language=bash]
$ cd examples
$ gcc -m32 -g example1.c
\end{lstlisting}


\subsection{Collect the trace}
We use the pin tool to collect the execution trace. The tool can automatically collect the trace and other necessary runtime information.
\begin{lstlisting}[language=bash]
$ cd /abacus/Pintools
$ make PIN_ROOT=/abacus/Intel-Pin-Archive/ TARGET=ia32
$ cd /abacus
$ /abacus/Intel-Pin-Archive/pin \ 
  -t Pintools/obj-ia32/MyPinToolLinux.so \
  -- ./examples/a.out 
\end{lstlisting}

You will get two files \textsf{Function.txt} and \textsf{Inst\_data.txt}. \textsf{Inst\_data.txt} is the mandatory input of \tool{}. \textsf{Function.txt} is optional. 

\subsection{Quantify the leakage}
To analyze the execution trace and generate the report, run the below command:
\begin{lstlisting}[language=bash]

$ ./build/App/QIF/QIF ./Inst_data.txt -f Function.txt \ 
  -d ./examples/a.out -o result.txt
\end{lstlisting}

You should see the following output:

\begin{lstlisting}[language=bash]
Start Computing Constraints
Total Constraints: 1
Control Transfer: 1
Data Access: 0
Information Leak for each address:
Address: 5664259b Leaked:1.0 bits Type: CF  
Source code: example1.c line number: 3
Function Name:  is_odd Module Name:  a.out Offset: 30
...
\end{lstlisting}

As expected, it shows that the function is\_odd has one secret-dependent control-flows at line 3. Also, Abacus show the vulnerability leaks 1 bit information.

\section{Analyze Cryptography Function}
We have applied~tool{} on the following libraries:
\begin{itemize}
\item OpenSSL: 0.9.7, 1.0.2f, 1.0.2k, 1.1.0f, 1.1.1, 1.1.1g
\item MbedTLS: 2.5, 2.15
\item Libgcrypt: 1.8.5
\item Monocyper: 3.0
\end{itemize}

The results can be reproduced by running the simple command after you build Abacus successfully in the docker (step 1). We have prepared scripts to analyze each cryptography algorithm automatically.

For example, if you want to test AES in mbedTLS 2.5, you can try the 
following command.

\begin{lstlisting}[language=bash]
$ cd /abacus/script/AES_MBEDTLS_2.5
$ ./start.sh
\end{lstlisting}

\section{Command-line Options}
~\tool{} takes the trace file as the input (\textsf{Inst\_data.txt}). After the trace file, \tool{} has the following command-line options:
\vspace{3pt}
\\\textsf{-d {executable file}}
\\ Read an elf \textit{executable file}. \tool{} can parse the debug information inside the file. With the optional input, \tool{} is able to output which line in the original source code actually leaks the sensitive information.
\vspace{3pt}
\\\textsf{-f {function file}}
\\Read a function file that was generated by the Pin tool. The command is optional. With the optional input, \tool{} is able to output which function leaks the information and call sites from the sensitive buffer to the leaked site.
\vspace{3pt}
\\\textsf{-n {Monte Carlo times}}
\\ Set the times of Monte Carlo Sampling when \tool{} estimates the amount of leakage information. If you do not specify the option, \tool{} can automatically terminate the sampling when the tool has 95\% confidence that the error of estimated leaked information is less than 1 bit. Please refer to the paper if you want to learn more about the details.
\vspace{3pt}
\\\textsf{-a {The address of the secret buffer} \\-s {The size of the buffer}}
\\The two input options must be used together. In the previous example, we use abacus\_make\_symbolic to mark the secret buffer. For a raw binary, we use the two options to tell \tool{} which buffer is the secret and its length.

\section{CONTRIBUTING}
\tool{} is open source with an Apache 2.0 license. Any pull requests and feature requests are welcome.

\section*{Acknowledgment}
We thank anonymous reviewers for their valuable feedback. The work was supported in part by the National Science Foundation (NSF) under grant CNS-1652790, and the Office of Naval Research (ONR) under grants N00014-16-1-2912, N00014-16-1-2265, and N00014-17-1-2894. 


\bibliographystyle{IEEEtran}
\bibliography{refs}
\end{document}
