\section{Introduction} 
%% side channels are important
Side channels are inevitable in modern computer systems as the sensitive
information may be leaked through many kinds of inadvertent behaviors, such as power,
electromagnetic radiation, and even
sound~\cite{agrawal2002side,kar20178,chari1999towards,217605,genkin2014rsa}.
Among them, software-based side-channel attacks, such as cache attacks, memory page
attacks, and controlled-channel attacks, are especially common and have been
studied for
years~\cite{7163052,217543,217589,lee2017inferring,191010,liu2015last}. These
attacks result from vulnerable software and shared hardware components.
By observing program outputs or hardware behaviors, attackers can infer program
execution flows that manipulate secrets and guess secrets such as encryption
keys~\cite{Osvik2006,Gullasch:2011:CGB:2006077.2006784,203878,10.1007/978-3-540-45238-6_6}.

%% to deal with side channels, we can protect or detect them and detection is better
%%arious countermeasures have been proposed to defend against software-based
%%side-channel attacks. Hardware-level solutions, such as reducing shared
%%resources, adopting oblivious RAM, and using transnational
%%memory~\cite{203878,217537,shih2017t,Zhang:2015:HDL:2775054.2694372}, need new
%%hardware features and changes in modern complex computer systems, which are
%%impractical and hard to adopt in reality. Therefore, a promising and
%%universal direction is software countermeasures, detecting and eliminating
%%side-channel vulnerabilities from code base.

Regarding the root cause of software-based side channel attacks, many of them originate
from two specific circumstances: data flow from secrets to load
addresses and data flow from secrets to branch conditions. We refer to them as
 secret-dependent memory-accesses and control-flows, respectively. A
central problem is eliminating these two code patterns. 
Recent
works~\cite{203878,217537,Wichelmann:2018:MFF:3274694.3274741,Brotzman19Casym,236338,182946},
find plenty of side-channel vulnerabilities. For example,
DATA~\cite{217537} reports 2,246 potential leakage sites for the RSA
implementation in OpenSSL\@. 
%After some inspections, 1,510 leakages are dismissed. But it
%still leaves 460 data-access leakages and 278 control-flow leakages. 
However, we find most of the reported vulnerabilities are not fixed because
of the following reasons.
First, many vulnerable implementations have better performance and 
are well-known for years. For example,
symmetric encryptions like AES and DES use lookup tables (T-tables), which
is fast but notoriously known to be vulnerable to side channels.
As for asymmetric encryptions, many implementations of RSA, adopt the CRT optimization,
which is faster but vulnerable to fault attacks~\cite{aumuller2002fault}.
Second, side-channels are numerous and it is hard to fix all these vulnerabilities,
let alone the majority of them are negligible. 
That is, some vulnerabilities can result in the key being 
entirely compromised~\cite{184415, aumuller2002fault}, but many other vulnerabilities prove to be less
severe in reality. Therefore, we need a proper quantification metric to 
assess the sensitive level of side-channel vulnerabilities.

Previous attempts like static methods~\cite{182946,5207642}, usually with
abstract interpretations, can give a leakage upper bound, which is useful to
justify the implementation is secure if they report zero leakage.
However, they cannot indicate how severe the leakage is because of the
over-approximation method they apply. For example, CacheAudit~\cite{182946} 
reports that the upper
bound leakage of AES-128 exceeds the original key size. The dynamic methods take
another approach with a concrete input and run the program in a real
environment. Although they are very precise in terms of true leakages, no
existing tool can precisely assess the severity of vulnerabilities in production
software. 

To overcome these limitations, we propose a novel method to quantify information
leakages precisely. Unlike previous works that only consider the
``average'' information leakage, we study the problem based on real attack
scenarios. The average information leakage model assumes the target program has
\emph{variable} or \emph{random} sensitive information as inputs when an attack is
launched. However, for real-world attacks, an adversary may run the target
problem with the \emph{fixed} unknown sensitive information
as the input. Therefore, the previous threat model cannot model real attack
scenarios. In contrast, our method is more precise and fine-grained. 
We quantify
the amount of leaked information as the cardinality of the set of possible
inputs based on an attacker's observation to each leakage site.
Before an attack, an adversary has a large but finite input space. Every time
when the adversary observes a leakage site, he can eliminate some potential
inputs and reduce the size of the input space. The smaller the input space is,
the more information is obtained. In an extreme case, if the size of the
input space reduces to one, the adversary can uniquely determine the input, 
which means all the secret information (e.g., the whole secret key) is
leaked. By counting the number of distinct inputs~\cite{10.1007/11499107_24}, we can quantify the
information leakage precisely.

We use constraints generated from symbolic execution to model the relation 
between the original sensitive input
and the attacker's observations. Symbolic execution can
provide fine-grained information but is usually believed to be an expensive
operation in terms of performance. Therefore, existing dynamic symbolic
execution works~\cite{203878,236338,Brotzman19Casym} either only analyze
small programs or apply some domain knowledge~\cite{203878} to simplify the analysis. We
examine the bottleneck of the trace-oriented symbolic execution and optimize it
to be scalable to real-world cryptosystems.

We apply the above technique and build a tool called \tool{}, 
to discover potential information leakage sites as well as estimate how
many bits they can leak for each leakage site. 
First, we collect dynamic execution traces for each target
libraries and then run symbolic execution on each instruction. In this way, we model
each side-channel leakage as a logic formula. The sensitive input is divided into
several independent bytes, and each byte is regarded as a unique and free symbol. Those
formulas can precisely model side-channel vulnerabilities. Then we use the conjunction
of those formulas to model the same leaks in the source code but appears in the different location of
the execution trace file (e.g., leakages inside a loop).
Finally, we introduce Monte Carlo
sampling method to estimate the single and combined information leakage. 


%Based on the fixed attack target, we classify the software-based side-channel
%vulnerabilities into two categories: 1.\textit{secret-dependent control-flow
%transfers} and 2.\textit{secret-dependent data accesses} and model them with
%math formulas which constrain the value of sensitive information. We quantify
%the amount of leaked information as the number of possible solutions that are
%reduced after applying each constrains.


%Our method can identify and quantify address-based sensitive information
%leakage sites in real-world applications automatically. Adversaries can exploit
%different control-flow transfers and data-access patterns when the program
%processes different sensitive data. We refer them as the potential information
%leakage sites. Our tool can discover and estimate those potential information
%leakage sites as well as how many bits they can leak. We are also able to
%report precisely how many bits can be leaked in total if an attacker observes
%more than one site. We run symbolic execution on execution traces. We model
%each side-channel leakage as a math formula. The sensitive input is divided
%into several independent bytes and each byte is regarded as a unique symbol.
%Those formulas can precisely model every the side-channel vulnerability. In
%other words, if the application has a different sensitive input but still
%satisfies the formula, the code can still leak the same information.  
%Those information leakage sites may spread in the whole program and their
%leakages may not be dependent. Simply adding them up can only get a coarse
%upper bound estimate. In order to accurately calculate the total information
%leakage, we must know the dependent relationships among those multiple leakages
%sites. Therefore, we introduce a monte carlo sampling method to estimate the
%total information leakage.

We apply \tool{} on both symmetric and asymmetric ciphers from real-world crypto
libraries, including OpenSSL, mbedTLS and Monocypher\@. The experimental result confirms
that \tool{} can precisely identify previously known vulnerabilities, reports
how much information is leaked and which byte in the original sensitive buffer
is leaked. We also test \tool{} on side-channel free algorithms. \tool{} has no
false positives.
Also, we show the
widely deployed software countermeasures can mitigate side channels.
Newer version of crypto libraries leak less amount of information 
compared to previous versions.
\tool\ also discovers new vulnerabilities. With the help of \tool{}, we confirm
that those vulnerabilities are severe.

In summary, we make the following contributions:

\begin{itemize}
      \item We propose a novel method that can quantify fine-grained leaked
            information from side-channel vulnerabilities to match real attack
            scenarios. Our method is different from previous ones in that we
            model real attack scenarios more precisely, while the previous
            research only models the ``average'' or ``random'' case. 
            We transfer the information quantification problem into a counting
            problem and use the Monte Carlo sampling method to estimate the
            information leakage.
   
      \item We implement the proposed method into a tool and apply it
            on several real-world software. \tool{} successfully identifies
            memory-related side-channel vulnerabilities and calculates the
            corresponding information leakage. 
            Our results are surprisingly different, much more useful in practice.
            The information leakage results
            provide detailed information that can help developers to fix the
            reported vulnerabilities.
\end{itemize}
