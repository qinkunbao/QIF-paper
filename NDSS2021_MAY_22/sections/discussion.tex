\section{Discussions and Limitations}
While recent works have reported lots
of potential side-channel vulnerabilities, we find most of them are not patched by
developers because they do not think those vulnerabilities are severe enough.
However, side-channels are inevitable in software and it is hard to fix all of them. 
Addressing old vulnerabilities may also introduce new leakage
sites. They need a
tool that can automatically estimate the sensitive level of each vulnerability.
So software engineers can focus on
``severe'' leakages. For example, our tool will report that 
the modular exponentiation using square and multiply algorithms can
leak much information than a key validation function.

\tool{} can be used by software developers to find sensitive vulnerabilities
and reason about countermeasures.
\tool{} estimates the amount of leaked information for each side-channel leakage
in one particular execution. \tool{} is useful for software
engineers to test programs and fix vulnerabilities.
\tool{} works on native x86 execution traces. The design, which is very
precise in terms of true leakages compared to other static source code
method~\cite{197207,BacelarAlmeida:2013:FVS:2483313.2483334}, can omit some
leakages on other traces. The amount of leaked information also depends on the secret key.
However, as the tool is intended for debugging and testing.
we think it is software engineers' responsibility to select the input key and trigger the 
path in which they are interested.

We use the amount of leaked information to represent the sensitive level of each
side-channel vulnerability. Although imperfect, \tool{} produces a reasonable
measurement for each leak. For example, the simple modular exponentiation is notoriously
famous for various side-channel attacks~\cite{kocher1996timing}. During the execution, the single 
leak points will be executed multiple times and each times it leak one different bit.
Therefore, \tool{} reports that the vulnerability can leak the whole key. However, not every
leak point inside a loop is serious. Because very often the leakages may leaks the same
bit in the original key and those leaks are not independent. \tool{} can capture the most 
fine-grained information by
modeling each leak during the execution 
as a math formula and uses the conjunction of those formulas to describe the total effect.

Abacus reaches full precision if the number of estimated leaked bits 
equals to Definition~\ref{def}. According to the threat model, the 
length of the secret is known, so the only estimated value is the number
 of possible secrets under the attacker's observation ($|K^o|$). 
 During the symbolic execution,Abacus may lose precision from the 
 memory model it uses in theory. However, we do not find false positives 
 caused by the imprecise memory model during the evaluation. 
 During the symbolic execution, for each constraint it generates,
 the tool will check the correctness of the constraint by giving 
 the constraint real values (the value of sensitive keys).Samplings 
 can also introduce imprecision but with a probabilistic guarantee, 
 as long as the sampling can terminate under the CLT condition. 
 However,during the evaluation, we find that \tool{} cannot estimate 
 the amount of leakage for some leakage sitesin a reasonable time, 
 which means the number of $K^o$ is very small. According to Definition~\ref{def}, 
 means the leakage is very severe. However, the problem is not that simple. 
 Let's say $|K^o|$ equals to 1.If we have an efficient algorithm to 
 find the item in $K^o$,then it is a truly severe leakage. 
 We can recover the whole key. If the constraint is very complex
 (e.g., a constraint that maps the plaintext and ciphertext for AES-128), 
 then it is not a leak at all.