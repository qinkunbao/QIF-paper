\section{Conclusion}
In this paper, we present a novel information leakage %definition and
method to
quantify memory-based side-channel leakages. We implement the method in
a prototype called \tool{} and show that it is effective in finding
and quantifying the side-channel leakages. With the new definition of
information leakage that imitates real side-channel attackers, the number of
leaked bits is useful in practice to justify and understand the severity level
of side-channel vulnerabilities. The evaluation results confirm our design goal
and show \tool{} is useful in estimating the amount of leaked information in
real-world applications.
