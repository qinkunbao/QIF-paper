
\newcommand{\vvv}{\vspace*{-3pt}}


\section{Algorithm to Compute the Maximum Independent Partition}
\label{appendix:partition}
~
{\small
\IncMargin{1em}
\begin{algorithm}[h]\small
    \DontPrintSemicolon
    \SetKwInOut{Input}{input}\SetKwInOut{Output}{output}
    \Input{$c_t(\addr{1},\addr{2},\ldots,\addr{n}) = c_{\addr{1}} \land c_{\addr{2}} \land \ldots \land c_{\addr{m}}$}
    \Output{The Maximum Independent Partition of $G = \{g_{1}, g_{2}  , \ldots,  g_{m} \}$ }
    \For{$i\leftarrow 1$ \KwTo $n$}
    {
        $S_{c_{\addr{i}}}$ $\leftarrow$ $\pi(c_{\addr{i}})$ \;
        \For{$g_{i} \in G$}
        {
            $S_{g_j}$ $\leftarrow$ $\pi(g_{j})$ \;
            $S$ $\leftarrow$ $S_{c_{\addr{i}}} \cap S_{g_j}$  \;
            \If{$S \neq \emptyset$}
            {
                $g_{j} \leftarrow g_{i} \land g_{\addr{i}}$ \;
                \textbf{break} \;
            }
            Insert $c_{\addr{i}}$ to $G$
        }
    }
    \caption{The Maximum Independent Partition}
    \label{algo:max-inde}
\end{algorithm}
\DecMargin{1em}
}
~
%\newpage
\section{Algorithm to Compute the Number of Satisfying Assignments}
\label{appendix:montecarlo}
~
{\small
\IncMargin{1em}
\begin{algorithm}\small
    \SetAlgoLined
    \DontPrintSemicolon

    \KwIn{{The constraint $g_{i}= c_{i_1} \land c_{i_2}
                    \land \ldots \land c_{i_m}$}}
    \KwOut{{The number of assignments that satisfy $g_{i}$ $|K_{g_{i}}|$}}

    $n$: the number of sampling times \;
    $S_{c_i}$: the set contains input variables for $c_{i}$ \;
    $n_{s}$: the number of satisfying assignments \;
    $N_{c_t}$: the set contains all solution for $c_t$ \;
    $r$: times of reducing $g$\;
    $k$: the input variable \;
    $R$: a function that produces a random point from $S_{c_i}$\;
    %$\#k$: the satisfying number of k \fixme{this number is not used syntactically} \;
    %Initialization: \;
    $r$ $\leftarrow$ $1$,
    $n$ $\leftarrow$ $0$ \;
    \For{$t$ $\leftarrow$ $1$ \KwTo $m$} {
        $S_{c_t}$ $\leftarrow$ $\pi(c_t)$ \;
        \If{$|S_{c_t}| = 1$}
        {
            $N_{c_t}$ $\leftarrow$ Compute all solutions of $c_i$ \;
            $N_{c_t} = \{n_1, \ldots, n_m\},\ S_{c_t} = \{k\}  $ \;
            $g_{i} = $ $g_i(k=n_1) \land \ldots \land g_i(k=n_m)$ \;
            $r \leftarrow r+1$ \;

        }
    }
    \While{$n \leq \frac{6p}{1-p}$} {
        $S_{g_i}$ $\leftarrow$ $\pi(g_i)$ \;
        $v \leftarrow R(S_{g_i})$ 
        \If{$v$ satisfies $g_i$}
        {
           $n_s \leftarrow n_s + 1$
        }
        $n \leftarrow n +1,\ p = \frac{n_s}{n}$
    }

    $|K_{g_{i}}|$ $\leftarrow$ $n_s|K| / (n * r * range(k))$
    \caption{Multiple Step Monte Carlo Sampling}
\end{algorithm}
\DecMargin{1em}
}
~\vvv

\clearpage
\section{Mitigation Method Evaluation}
%\section{SBOX Example of Bit-slicing}
\label{appendix:SBOX}
~
\vvv
\vvv
\vvv
\vvv
\vvv
\vvv
\begin{figure}[h!]
    \centering
    \begin{lstlisting}[xleftmargin=.02\textwidth,xrightmargin=.01\textwidth]
uint8_t password = input();

a = *password & 0b001;
b = (*password & 0b010) >> 1;
c = (*password & 0b100) >> 2;

na = ~a & 1;
nb = ~b & 1;
nc = ~c & 1;

t0 = (b & nc);
t1 = (b | nc);

l = (a & nb) | t0;
r = (na & t1) | t0;

ret = l << 1 + r;
      \end{lstlisting}
    \caption{SBOX with BitSlicing}
    \label{fig:SBOX_bitslicing}
\end{figure}
~\vvv
\vvv
\vvv
\vvv
\begin{figure}[h!]
    \centering
    \begin{lstlisting}[xleftmargin=.02\textwidth,xrightmargin=.01\textwidth]
uint8_t password = input();

uint8_t SBOX[] = {1, 0, 3, 1, 2, 2, 3, 0};

if (password <= 0b111)      \\Leaks 5 bits of password
    ret = SBOX[password];   \\Leaks 4 bits of password
      \end{lstlisting}
\vvv
\vvv
    \caption{SBOX without BitSlicing}
    \label{fig:SBOX_da}
\end{figure}
~\vvv
\vvv
\vvv
\vvv
%\section{Example for Lookup Table}
\begin{figure}[h!]
    \centering
    \begin{lstlisting}[xleftmargin=.02\textwidth,xrightmargin=.01\textwidth]
static const uint8_t T[16] = {
      0x63U, 0x7cU, 0x77U, 0x7bU, 0xf2U, 0x6bU, 0x6fU, 0xc5U,
      0x30U, 0x01U, 0x67U, 0x2bU, 0xfeU, 0xd7U, 0xabU, 0x76U};
      \end{lstlisting}
    \caption{One-byte Entry Lookup Table}
    \label{fig:one_byte_table}
\end{figure}
~\vvv
\vvv
\vvv
\vvv
\begin{figure}[h!]
    \centering
    \begin{lstlisting}[xleftmargin=.02\textwidth,xrightmargin=.01\textwidth]
static const uint32_t T[16] = {
    0xc66363a5U, 0xf87c7c84U, 0xee777799U, 0xf67b7b8dU,
    0xfff2f20dU, 0xd66b6bbdU, 0xde6f6fb1U, 0x91c5c554U,
    0x60303050U, 0x02010103U, 0xce6767a9U, 0x562b2b7dU,
    0xe7fefe19U, 0xb5d7d762U, 0x4dababe6U, 0xec76769aU};
    \end{lstlisting}
\vvv
\vvv
    \caption{Four-byte Entry Lookup Table}
    \label{fig:four_byte_table}
\end{figure}
~\vvv
\vvv
\vvv
\vvv
\vvv
\vvv
\begin{figure}[h!]
    \centering
    \begin{lstlisting}[xleftmargin=.02\textwidth,xrightmargin=.01\textwidth]
void encrypt_one(uint32_t *o, uint32_t *key, uint32_t l) {
    for (int i = 0; i < l; i+=4)
        output[i] = (T[(key[i]>>24)] << 24) ^
                    (T[(key[i+1]>>16) & 0xff] << 16) ^
                    (T[(key[i+2]>>8) & 0xff] << 8) ^
                    (T[(key[i+3]) & 0xff]);
}
    \end{lstlisting}
\vvv
\vvv
    \caption{Encrypt with One-byte Entry Lookup Table}
    \label{fig:one_byte_table_lookup}
\end{figure}
~\vvv
\vvv
\vvv
\vvv
\vvv
\vvv
\begin{figure}[h!]
    \centering
    \begin{lstlisting}[xleftmargin=.02\textwidth,xrightmargin=.01\textwidth]
void encrypt_four(uint32_t *o, uint32_t *key, uint32_t l) {
    for (int i = 0; i < l; i+=4)
        output[i] = (T[(key[i]>>24)] & 0xff000000) ^
                    (T[(key[i+1]>>16) & 0xff] & 0x00ff0000) ^
                    (T[(key[i+2]>>8) & 0xff] & 0x0000ff00) ^
                    (T[(key[i+3]) & 0xff] & 0x000000ff);
}
    \end{lstlisting}
\vvv
\vvv
    \caption{Encrypt with One-byte Entry Lookup Table}
    \label{fig:four_byte_table_lookup}
\end{figure}



\newpage
\section{Detailed Experimental Results}
\label{sec:result-table}

Here we present the detailed experimental results.
Due to space limitation, we select the representative implementations of
AES, DES, and RSA in
mbed TLS 2.5,
OpenSSL 1.1.0f,  and
OpenSSL 1.1.1.  
%For RSA, we also include OpenSSL 1.0.2f and OpenSSL 1.0.2k.
The results are representative to other versions.
All the results will be made available in electronic format online
when the paper is published. %at \fixme{http://tinyurl}.

In all the tables presented in this appendix, the mark ``$*$'' means timeout,
which indicates more severe leakages. See \S\ref{loc:timeout} for the details.
Also note that we round the calculated numbers of leaked bits to include one digit
after the decimal point, so $0.0$ really means very small amount of leakage, but not exactly zero. See \S\ref{sssec:errest} for the details of error estimate.

  %as an integer.
  %Hence the `0' leakage does not equals to no leakage but some leakage size
 % between 0 and 0.5 bit, see \S\ref{sssec:errest} for the details.


\input{figures/form/latex/DES-mbedTLS-2-5.tex}
%% \input{figures/form/latex/DES-mbedTLS-2-15-1.tex}
%% \input{figures/form/latex/DES-openssl-0-9-7.tex}
%% \input{figures/form/latex/DES-openssl-1-0-2f.tex}
%% \input{figures/form/latex/DES-openssl-1-0-2k.tex}
\input{figures/form/latex/DES-openssl-1-1-0f.tex}
\input{figures/form/latex/DES-openssl-1-1-1.tex}


\input{figures/form/latex/RSA-mbedTLS-2-5.tex}
%% \input{figures/form/latex/RSA-mbedTLS-2-15-1.tex}
%% \input{figures/form/latex/RSA-openssl-0-9-7.tex}
%% \input{figures/form/latex/RSA-openssl-1-0-2f.tex}
%% \input{figures/form/latex/RSA-openssl-1-0-2k.tex}
\input{figures/form/latex/RSA-openssl-1-1-0f.tex}
\begin{table}[h!]
\centering\tiny\scriptsize
\caption{Leakages in RSA implemented by OpenSSL 1.1.1}\label{tab:RSAOpenSSL1.1.1}
%\resizebox{\columnwidth}{!}{
\begin{tabular}{l@{~~}rlr@{~~}r}
\hline
\textbf{File} & \textbf{Line No.} & \textbf{Function} & \hspace*{-20em}\textbf{\# Leaked Bits} & \textbf{Type} \\\hline
%\input{figures/form/latex/RSA-openssl-1-1-1-data.tex}
rsa\_ossl.c& 399&rsa\_ossl\_private\_decrypt&0.0 &CF\\
bn\_lib.c& 555&BN\_ucmp&*&CF\\
bn\_gcd.c& 199&int\_bn\_mod\_inverse&1.0 &CF\\
bn\_gcd.c& 247&int\_bn\_mod\_inverse&14.9 &CF\\
bn\_gcd.c& 225&int\_bn\_mod\_inverse&12.3 &CF\\
ct\_b64.c& 168&\_\_udivdi3&0.1 &CF\\
bn\_div.c& 374&bn\_div\_fixed\_top&*&CF\\
bn\_lib.c& 955&bn\_correct\_top&2.6 &CF\\
ct\_b64.c& 168&\_\_memset\_sse2\_rep&0.0 &CF\\
ct\_b64.c& 168&\_\_memset\_sse2\_rep&0.0 &CF\\
ct\_b64.c& 168&\_\_memset\_sse2\_rep&0.0 &DA\\
ct\_b64.c& 168&\_\_memset\_sse2\_rep&0.0 &DA\\
bn\_exp.c& 317&BN\_mod\_exp\_mont&1.0 &CF\\
bn\_asm.c& 592&bn\_mul\_comba8&2.1 &CF\\
bn\_exp.c& 383&BN\_mod\_exp\_mont&0.9 &CF\\
bn\_lib.c& 453&BN\_bn2binpad&0.0 &DA\\
bn\_lib.c& 450&BN\_bn2binpad&0.0 &CF\\
rsa\_oaep.c& 180&RSA\_padding\_check\_PKCS1\_OAEP\_mgf1&0.0 &DA\\
rsa\_oaep.c& 180&RSA\_padding\_check\_PKCS1\_OAEP\_mgf1&0.0 &DA\\
rsa\_oaep.c& 176&RSA\_padding\_check\_PKCS1\_OAEP\_mgf1&0.0 &CF\\
string3.h& 90&SHA1\_Final&0.0 &CF\\
rsa\_oaep.c& 200&RSA\_padding\_check\_PKCS1\_OAEP\_mgf1&0.0 &CF\\
rsa\_oaep.c& 209&RSA\_padding\_check\_PKCS1\_OAEP\_mgf1&0.0 &CF\\
rsa\_oaep.c& 250&RSA\_padding\_check\_PKCS1\_OAEP\_mgf1&0.0 &CF\\
rsa\_oaep.c& 253&RSA\_padding\_check\_PKCS1\_OAEP\_mgf1&0.0 &CF\\
ct\_b64.c& 168&\_\_memset\_sse2\_rep&0.0 &DA\\
ct\_b64.c& 168&\_\_memset\_sse2\_rep&0.0 &DA\\
\hline
\end{tabular}
%}
\renewcommand{\baselinestretch}{1.0}\selectfont
\end{table}



\input{figures/form/latex/AES-mbedTLS-2-5.tex}
%% \input{figures/form/latex/AES-mbedTLS-2-15-1.tex}
%% \input{figures/form/latex/AES-openssl-0-9-7.tex}
%% \input{figures/form/latex/AES-openssl-1-0-2f.tex}
%% \input{figures/form/latex/AES-openssl-1-0-2k.tex}
\input{figures/form/latex/AES-openssl-1-1-0f.tex}
\input{figures/form/latex/AES-openssl-1-1-1.tex}
